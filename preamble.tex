\usepackage[utf8]{inputenc}
\usepackage[english]{babel}
\usepackage[margin=1in]{geometry}
\usepackage[T1]{fontenc}
\usepackage{graphicx}
\graphicspath{ {../Figures/} }
\setlength\parindent{0pt}
\usepackage{hyperref}
\hypersetup{colorlinks=true,linkcolor=blue,filecolor=magenta,urlcolor=cyan,}
\urlstyle{same}
\usepackage{amsthm}
\usepackage{amsmath}
\theoremstyle{definition}
\usepackage{pgfplots}
\usepackage{caption}
\usepackage{subcaption}
\usepackage{makecell}
\usepackage[table]{colortbl}
\usepackage{enumitem}
\usepackage{siunitx}
\usepackage{amssymb}
\usepackage{tikz-cd}
\tikzset{>=latex}
\usepackage{tkz-euclide}
\usepackage{tikz,bm}
\usepackage{mwe,tikz}
\usetikzlibrary{arrows}
\pgfplotsset{compat=1.11}
\usepackage{moresize}
\usepackage{bohr}
\usetikzlibrary{patterns}
\usepackage{wrapfig}
\usepackage{mdframed}
\usepackage{dashrule}
\usepackage{tikzsymbols}
\usepackage{fontawesome}
\usepackage{linearb} %for \BPwheel symbol in Unit 6
\usepackage[clock]{ifsym} %for clock and stopwatch symbols
\usepackage{multicol}
\usepackage{glossaries}
\usepackage{cancel}
% \usepackage{circuitikz}
\sisetup{group-separator = {,}}
\usepgfplotslibrary{fillbetween}
\usetikzlibrary{math}
\numberwithin{equation}{section}
\numberwithin{figure}{section}




\DeclareSIUnit{\nothing}{\relax}
\def\mymu{\SI{}{\micro\nothing} }

\newtheorem{example}{Example}[section]
\newtheorem{exercise}{}[section]
\newtheorem{regla}{Rule}

\newenvironment{warmup}
    {\begin{mdframed}[backgroundcolor=cpBlue,fontcolor=white]
        \begin{center}
            \texttt[csOrange]{WARM-UP}
        \end{center}
        
        }
        {
    \end{mdframed}
    }

\newenvironment{hook}
    {\begin{mdframed}[backgroundcolor=cpViolet,fontcolor=white]
        \begin{center}
            \texttt[csOrange]{OPENING HOOK}
        \end{center}
        
        }
        {
    \end{mdframed}
    }


\newenvironment{cfu}
    {\begin{mdframed}
        \begin{center}
            \texttt[cpOrange]{CHECK FOR UNDERSTANDING}
        \end{center}
        
        }
        {
    \end{mdframed}
    }

\newcommand{\Solution}{{\footnotesize \color{cyan} SOLUTION }}


\def\endsolution{{\footnotesize \color{cyan} \hfill END OF SOLUTION }}

\newcommand{\cyanhrule}{{\color{cyan} \hrule }}

\def\redplus{\mathbin{\color{red} +}}
\def\redminus{\mathbin{\color{red} -}}
\def\redtimes{\mathbin{\color{red} \times}}


\def\openstax{https://openstax.org/books/physics/pages/1-introduction}
\def\openstaxfooter{\fancyfoot[C]{Access for free at \href{\openstax}{\openstax} \hfill \thepage}}


% The following needs to be added to each individual main.tex file so it doesn't interfere with exam headers:

%\usepackage{fancyhdr}
% \pagestyle{fancy}
% \renewcommand{\headrulewidth}{0pt}
% \renewcommand{\headruleskip}{0mm}
% \fancyhead{}
% \def\openstax{https://openstax.org/books/physics/pages/1-introduction}
% \def\openstaxfooter{\fancyfoot[C]{Access for free at \href{\openstax}{\openstax} \hfill \thepage}}

\newcommand\myboxa[2][]{\tikz[overlay]\node[fill=gray!20,inner sep=4pt, anchor=text, rectangle, rounded corners=1mm,#1] {#2};\phantom{#2}}
\newcommand{\hgraydashline}{{\color{lightgray} \hdashrule{0.99\textwidth}{1pt}{0.8mm}}}

\let\oldtexttt\texttt% Store \texttt
\renewcommand{\texttt}[2][black]{\textcolor{#1}{\ttfamily #2}}% 

\newcommand\mybox[2][]{\tikz[overlay]\node[fill=black!20,inner sep=2pt, anchor=text, rectangle, rounded corners=1mm,#1] {#2};\phantom{#2}}

\setlength{\columnsep}{1cm}
\setlength{\columnseprule}{1pt}
\def\columnseprulecolor{\color{cyan}}

\pgfdeclarehorizontalshading{visiblelight}{50bp}{
color(0.00000000000000bp)=(red);
color(8.33333333333333bp)=(orange);
color(16.66666666666670bp)=(yellow);
color(25.00000000000000bp)=(green);
color(33.33333333333330bp)=(cyan);
color(41.66666666666670bp)=(blue);
color(50.00000000000000bp)=(violet)
}

\def\myfillin{\rule{2cm}{0.15mm}}

\def\phet{\texttt[red]{PhET} }

\usepackage{circuitikz}

\usepackage{utfsym} %to get symbol of car
\def\mycar{\reflectbox{\huge\usym{1F697}} } %modiying symbol of car
\def\mycarleft{\huge\usym{1F697}}%modiying symbol of car

\def\mytrain{\reflectbox{\huge\usym{1F682}} } %modiying symbol of train
\def\mytrainleft{\huge\usym{1F682}}%modiying symbol of train


\usepackage{nameref}

\setenumerate{itemsep=-2pt,topsep=-2mm,leftmargin=4em}
\setitemize{itemsep=-2pt,topsep=0pt,leftmargin=4em}

\def\xydirection{
        \begin{axis}[width=2.4cm,
            height=2.4cm,
            ticks=none,
            axis lines=center,
            ylabel=$y$,
            xlabel=$x$,
            xmin=0,xmax=1,
            ymin=0,ymax=1
        ]
        \end{axis}
}

\tikzset{
    declare function = {patheq(\x,\yi,\vi,\thetai)= \yi + tan(\thetai)*\x - 9.8*\x^2/(2*(\vi*cos(\thetai))^2);},
    declare function = {patheqten(\x,\yi,\vi,\thetai)= \yi + tan(\thetai)*\x - 10*\x^2/(2*(\vi*cos(\thetai))^2);} %like patheq but with gravity = 10
}

\def\mystep{\rotatebox[origin=c]{-90}{\faEject}}

%%%%% CFISD Primary Colors
\definecolor{cpBlue}{RGB}{29,118,187}
\definecolor{cpViolet}{RGB}{122,84,161}
\definecolor{cpOrange}{RGB}{250,173,61}

%%%%% CFISD Secondary Colors
\definecolor{csBlue}{RGB}{27,46,112}
\definecolor{csViolet}{RGB}{69,58,147}
\definecolor{csOrange}{RGB}{251,197,116}

\usetikzlibrary{fadings}

\newcommand{\gradientstart}[1]{%
    \vspace{1em}
    \begin{tikzpicture}
        \fill[gray,path fading=south] (0,1em) rectangle (\linewidth,2em);
    \end{tikzpicture}

    \vspace{-1em}

    \begin{center}
        \texttt{#1}
    \end{center}

    \vspace{-1ex}%
}

\def\gradientend{%
    \vspace{1em}
    
    \begin{tikzpicture}
        \fill[gray,path fading=north] (0,-1em) rectangle (\linewidth,-2em);
    \end{tikzpicture}%
}

\newenvironment{gradient}[1]
    {\vspace{1em}
    
    \begin{tikzpicture}
        \fill[RoyalBlue,path fading=south] (0,1em) rectangle (\linewidth,2em);
    \end{tikzpicture}
    
    \vspace{-1em}

    \begin{center}
        \texttt[RoyalBlue]{#1}
    \end{center}

    \vspace{-1ex}
        }
        {

        \vspace{1ex}
        
    \begin{tikzpicture}
        \fill[RoyalBlue,path fading=north] (0,-1em) rectangle (\linewidth,-2em);
    \end{tikzpicture}

    \vspace{1em}
    }






