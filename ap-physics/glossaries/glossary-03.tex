\makenoidxglossaries

% \newglossaryentry{vector}{
%     name=vector,
%     description={a quantity that has both magnitude and direction; an arrow used to represent quantities with both magnitude and direction}
% }

\newglossaryentry{head-to-tail method}{
    name={head-to-tail method},
    description={a method of adding vectors in which the tail of each vector is placed at the head of the previous vector}
}

\newglossaryentry{tail}{
    name={tail},
    description={the start point of a vector; opposite to the head or tip of the arrow}
}

\newglossaryentry{head}{
    name={head},
    description={the end point of a vector; the location of the tip of the vector's arrowhead; also referred to as the ``tip''}
}

\newglossaryentry{resultant}{
    name={resultant},
    description={the sum of two or more vectors}
}

\newglossaryentry{magnitude}{
    name={magnitude},
    description={the length or size of a vector; magnitude is a scalar quantity}
}

\newglossaryentry{direction}{
    name={direction},
    description={the orientation of a vector in space}
}

% \newglossaryentry{scalar}{
%     name={scalar},
%     description={a quantity with magnitude but no direction}
% }

\newglossaryentry{component}{
    name={component},
    description={a piece of a vector that points in either the vertical or the horizontal direction; every 2-d vector can be expressed as a sum of two vertical and horizontal vector components}
}

\newglossaryentry{commutative}{
    name={commutative},
    description={refers to the interchangeability of order in a function; vector addition is commutative because the order in which vectors are added together does not affect the final sum}
}

\newglossaryentry{resultant vector}{
    name=resultant vector,
    description={the vector sum of two or more vectors}
}

\newglossaryentry{analytical method}{
    name={analytical method},
    description={}
}

\newglossaryentry{projectile motion}{
    name={projectile motion},
    description={}
}

\newglossaryentry{motion}{
    name={motion},
    description={}
}

\newglossaryentry{projectile}{
    name={projectile},
    description={}
}

\newglossaryentry{trajectory}{
    name={trajectory},
    description={}
}

\newglossaryentry{air resistance}{
    name={air resistance},
    description={}
}

\newglossaryentry{kinematics}{
    name={kinematics},
    description={}
}

\newglossaryentry{range}{
    name={range},
    description={}
}



