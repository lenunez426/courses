\documentclass[dvipsnames]{article}
\usepackage[utf8]{inputenc}
\usepackage[english]{babel}
\usepackage{geometry}
\usepackage[T1]{fontenc}
\usepackage{graphicx}
\graphicspath{ {../Figures/} }
\setlength\parindent{0pt}
\usepackage{hyperref}
\hypersetup{colorlinks=true,linkcolor=blue,filecolor=magenta,urlcolor=cyan,}
\urlstyle{same}
\usepackage{amsthm}
\usepackage{amsmath}
\theoremstyle{definition}
\usepackage{pgfplots}
\usepackage{caption}
\usepackage{subcaption}
\usepackage{makecell}
\usepackage[table]{colortbl}
\usepackage{enumitem}
\usepackage{siunitx}
\usepackage{amssymb}
\usepackage{tikz-cd}
\tikzset{>=latex}
\usepackage{tkz-euclide}
\usepackage{tikz,bm}
\usepackage{mwe,tikz}
\usetikzlibrary{arrows}
\pgfplotsset{compat=1.11}
\usepackage{moresize}
\usepackage{bohr}
\usetikzlibrary{patterns}
\usepackage{wrapfig}
\usepackage{mdframed}
\usepackage{dashrule}
\usepackage{tikzsymbols}
\usepackage{fontawesome}
\usepackage{linearb} %for \BPwheel symbol in Unit 6
\usepackage{multicol}
\usepackage{glossaries}
\usepackage{cancel}
% \usepackage{circuitikz}
\sisetup{group-separator = {,}}
\usepgfplotslibrary{fillbetween}
\usetikzlibrary{math}
\numberwithin{equation}{section}
\numberwithin{figure}{section}




\DeclareSIUnit{\nothing}{\relax}
\def\mymu{\SI{}{\micro\nothing} }

\newtheorem{example}{Example}[section]
\newtheorem{exercise}{}[section]
\newtheorem{regla}{Rule}


\newcommand{\Solution}{{\footnotesize \color{cyan} SOLUTION }}


\def\endsolution{{\footnotesize \color{cyan} \hfill END OF SOLUTION }}

\newcommand{\cyanhrule}{{\color{cyan} \hrule }}

\def\redplus{\mathbin{\color{red} +}}
\def\redminus{\mathbin{\color{red} -}}
\def\redtimes{\mathbin{\color{red} \times}}


\def\openstax{https://openstax.org/books/physics/pages/1-introduction}
\def\openstaxfooter{\fancyfoot[C]{Access for free at \href{\openstax}{\openstax} \hfill \thepage}}


% The following needs to be added to each individual main.tex file so it doesn't interfere with exam headers:

%\usepackage{fancyhdr}
% \pagestyle{fancy}
% \renewcommand{\headrulewidth}{0pt}
% \renewcommand{\headruleskip}{0mm}
% \fancyhead{}
% \def\openstax{https://openstax.org/books/physics/pages/1-introduction}
% \def\openstaxfooter{\fancyfoot[C]{Access for free at \href{\openstax}{\openstax} \hfill \thepage}}

\newcommand\myboxa[2][]{\tikz[overlay]\node[fill=gray!20,inner sep=4pt, anchor=text, rectangle, rounded corners=1mm,#1] {#2};\phantom{#2}}
\newcommand{\hgraydashline}{{\color{lightgray} \hdashrule{0.99\textwidth}{1pt}{0.8mm}}}

\let\oldtexttt\texttt% Store \texttt
\renewcommand{\texttt}[2][black]{\textcolor{#1}{\ttfamily #2}}% 

\newcommand\mybox[2][]{\tikz[overlay]\node[fill=black!20,inner sep=2pt, anchor=text, rectangle, rounded corners=1mm,#1] {#2};\phantom{#2}}

\setlength{\columnsep}{1cm}
\setlength{\columnseprule}{1pt}
\def\columnseprulecolor{\color{cyan}}

\pgfdeclarehorizontalshading{visiblelight}{50bp}{
color(0.00000000000000bp)=(red);
color(8.33333333333333bp)=(orange);
color(16.66666666666670bp)=(yellow);
color(25.00000000000000bp)=(green);
color(33.33333333333330bp)=(cyan);
color(41.66666666666670bp)=(blue);
color(50.00000000000000bp)=(violet)
}

\def\myfillin{\rule{2cm}{0.15mm}}

\def\phet{\texttt[red]{PhET} }

\usepackage{circuitikz}

\usepackage{utfsym} %to get symbol of car
\def\mycar{\reflectbox{\huge\usym{1F697}} } %modiying symbol of car
\def\mycarleft{\huge\usym{1F697}}%modiying symbol of car

\def\mytrain{\reflectbox{\huge\usym{1F682}} } %modiying symbol of train
\def\mytrainleft{\huge\usym{1F682}}%modiying symbol of train


\usepackage{nameref}

\setenumerate{itemsep=-2pt,topsep=0pt,leftmargin=4em}







\begin{document}

\subsubsection*{\ref{iThIct} \nameref{iThIct}}

\begin{center}
    \begin{tikzpicture}
    \begin{axis}[width=8cm,height=5cm,
        axis lines = left,
        axis y line=none,
        xlabel = {Position (m)},
        ymin=0, ymax=6, 
        xmin=0, xmax=12,
        xtick={0,2,...,12},
        clip=false,
        ]
        \draw[->,thick] (0,4) node[above] {A} -- ++(7,0) ;
        \draw[->,thick] (12,3) node[above] {B} -- ++(-5,0) ;
        \draw[->,rounded corners=1pt,thick] (2,2) node[above] {C} -- ++(8,0) -- ++(0,0.2) -- ++(-2,0) -- ++(0,0.2) -- ++(2,0);
        \draw[->,rounded corners=2pt,thick] (9,1) node[above] {D} -- ++(-6,0) -- ++(0,0.2) -- ++(2,0);
    \end{axis}
    \end{tikzpicture}
\captionsetup{type=figure,margin=1in,font=scriptsize}
\captionof{figure}{}
\label{ljFwNL}
\end{center}

\begin{exercise}
    Find the following for path A in Figure \ref{ljFwNL}: (a) The distance traveled. (b) The magnitude of the displacement from start to finish. (c) The displacement from start to finish.
\end{exercise}

\begin{exercise}
    Find the following for path B in Figure \ref{ljFwNL}: (a) The distance traveled. (b) The magnitude of the displacement from start to finish. (c) The displacement from start to finish.
\end{exercise}

\begin{exercise}
    Find the following for path C in Figure \ref{ljFwNL}: (a) The distance traveled. (b) The magnitude of the displacement from start to finish. (c) The displacement from start to finish.
\end{exercise}

\begin{exercise}
    Find the following for path D in Figure \ref{ljFwNL}: (a) The distance traveled. (b) The magnitude of the displacement from start to finish. (c) The displacement from start to finish.
\end{exercise}

\subsubsection*{\ref{3hdww6} \nameref{3hdww6}}

\begin{exercise}
    (a) Calculate Earth's average speed relative to the Sun. (b) What is its average velocity over a period of one year?
\end{exercise}

\begin{exercise}
    A helicopter blade spins at exactly 100 revolutions per minute. Its tip is \SI{5.00}{m} from the center of rotation. (a) Calculate the average speed of the blade tip in the helicopter’s frame of reference. (b) What is its average velocity over one revolution?
\end{exercise}

\begin{exercise}
    The North American and European continents are moving apart at a rate of about \SI{3}{cm/y}. At this rate how long will it take them to drift \SI{500}{km} farther apart than they are at present?
\end{exercise}

\begin{exercise}
    Land west of the San Andreas fault in southern California is moving at an average velocity of about \SI{6}{cm/y} northwest relative to land east of the fault. Los Angeles is west of the fault and may thus someday be at the same latitude as San Francisco, which is east of the fault. How far in the future will this occur if the displacement to be made is \SI{590}{km} northwest, assuming the motion remains constant?
\end{exercise}

\begin{exercise}
    On May 26, 1934, a streamlined, stainless steel diesel train called the Zephyr set the world’s nonstop long-distance speed record for trains. Its run from Denver to Chicago took 13 hours, 4 minutes, 58 seconds, and was witnessed by more than a million people along the route. The total distance traveled was \SI{1633.8}{km}. What was its average speed in km/h and m/s?
\end{exercise}

\begin{exercise}
    Tidal friction is slowing the rotation of the Earth. As a result, the orbit of the Moon is increasing in radius at a rate of approximately \SI{4}{cm/year}. Assuming this to be a constant rate, how many years will pass before the radius of the Moon’s orbit increases by \SI{3.84e6}{m} (1\%)?
\end{exercise}

\begin{exercise}
    A student drove to the university from their home and noted that the odometer reading of their car increased by \SI{12.0}{km}. The trip took \SI{18.0}{min}. (a) What was their average speed? (b) If the straight-line distance from their home to the university is \SI{10.3}{km} in a direction \ang{25} south of east, what was their average velocity? (c) If they returned home by the same path \SI{7}{h} \SI{30}{min} after they left, what were their average speed and velocity for the entire trip?
\end{exercise}

\begin{exercise}
    The speed of propagation of the action potential (an electrical signal) in a nerve cell depends (inversely) on the diameter of the axon (nerve fiber). If the nerve cell connecting the spinal cord to your feet is \SI{1.1}{m} long, and the nerve impulse speed is \SI{18}{m/s}, how long does it take for the nerve signal to travel this distance?
\end{exercise}

\begin{exercise}
    Conversations with astronauts on the lunar surface were characterized by a kind of echo in which the earthbound person's voice was so loud in the astronaut's space helmet that it was picked up by the astronaut's microphone and transmitted back to Earth. It is reasonable to assume that the echo time equals the time necessary for the radio wave to travel from the Earth to the Moon and back (that is, neglecting any time delays in the electronic equipment). Calculate the distance from Earth to the Moon given that the echo time was \SI{2.56}{s} and that radio waves travel at the speed of light (\SI{3.00e8}{m/s}).
\end{exercise}

\begin{exercise}
    A football quarterback runs \SI{15.0}{m} straight down the playing field in \SI{2.50}{s}. He is then hit and pushed \SI{3.00}{m} straight backward in \SI{1.75}{s}. He breaks the tackle and runs straight forward another \SI{21.0}{m} in \SI{5.20}{s}. Calculate his average velocity (a) for each of the three intervals and (b) for the entire motion.
\end{exercise}

\begin{exercise}
    The planetary model of the atom pictures electrons orbiting the atomic nucleus much as planets orbit the Sun. In this model you can view hydrogen, the simplest atom, as having a single electron in a circular orbit \SI{1.06e-10}{m} in diameter. (a) If the average speed of the electron in this orbit is known to be \SI{2.20e6}{m/s}, calculate the number of revolutions per second it makes about the nucleus. (b) What is the electron's average velocity per revolution?
\end{exercise}


\end{document}



