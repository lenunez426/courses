\documentclass[../../main-ap-physics.tex]{subfiles}

\begin{document}

\section{Dynamics: Force and Newton's Laws of Motion}

\subsection{Development of Force Concept}

\Gls{dynamics} is the study of the forces that cause objects and systems to move. To understand this, we need a working definition of \gls{force}. Our intuitive definition of force---that is, a push or a pull---is a good place to start. We know that a push or pull has both magnitude and direction (therefore, it is a vector quantity) and can vary considerably in each regard. For example, a cannon exerts a strong force on a cannonball that is launched into the air. In contrast, Earth exerts only a tiny downward pull on a flea. Our everyday experiences also give us a good idea of how multiple forces add. If two people push in different directions on a third person, as illustrated in Figure \ref{KPYGZu}, we might expect the total force to be in the direction shown. Since force is a vector, it adds just like other vectors, as illustrated in Figure \ref{KPYGZu}(a) for two ice skaters. Forces, like other vectors, are represented by arrows and can be added using the familiar head-to-tail method or by trigonometric methods. These ideas were developed in ``Two-Dimensional Kinematics.''

\begin{center}
    \begin{tikzpicture}
        \begin{axis}[width=6cm,
            xmin=0,xmax=10,
            ymin=0,ymax=10,
            ticks=none,
            axis line style={draw=none},
            clip=false,
        ]
        \draw[->,red,thick] (0,5) node[below,black] {$F_1$} -- ++(5,0);
        \draw[->,red,thick] (5,0) node[left,black] {$F_2$} -- ++(0,5);
        \begin{scope}[shift={(5.25,0.25)}]
            \draw[->,red,dashed] (0,5) -- ++(5,0) node[below,pos=0.5,black] {$F_1$};
            \draw[->,red,dashed] (5,5) -- ++(0,5) node[right,pos=0.5,black] {$F_2$};
            \draw[->,red,thick] (0,5) -- ++(5,5) node[above left,pos=0.5,black] {$F_{\text{tot}}$};
        \end{scope}
        \node at (1,9) {(a)};
        \end{axis}
    \end{tikzpicture}%
    \hspace{1cm}
    \begin{tikzpicture}
        \begin{axis}[width=6cm,
            xmin=0,xmax=10,
            ymin=0,ymax=10,
            ticks=none,
            axis line style={draw=none},
            clip=false,
        ]
        \draw[->,thick] (5pt,0) -- ++(5,0) node[below,pos=0.5] {$F_1$};
        \draw[->,thick] (0,5pt) -- ++(0,5) node[left,pos=0.5] {$F_2$};
        \fill[red] (0,0) circle (3pt);
        \node[right] at (1,6) {Free-body diagram};
        \node at (1,8) {(b)};
        \end{axis}
    \end{tikzpicture}
    \captionsetup{type=figure,margin=1in,font=scriptsize}
    \captionof{figure}{Part (a) shows an overhead view of two ice skaters pushing on a third. Forces are vectors and add like other vectors, so the total force on the third skater is in the direction shown. In part (b), we see a free-body diagram representing the forces acting on the third skater.}
    \label{KPYGZu}
\end{center}

Figure \ref{KPYGZu}(b) is our first example of a \gls{free-body diagram}, which is a technique used to illustrate all the \gls{external force}s acting on a body. The body is represented by a single isolated point (or free body), and only those forces acting on the body from the outside (external forces) are shown. (These forces are the only ones shown, because only external forces acting on the body affect its motion. We can ignore any internal forces within the body.) Free-body diagrams are very useful in analyzing forces acting on a system and are employed extensively in the study and application of Newton's laws of motion.

\vspace{1em}

% A more quantitative definition of force can be based on some standard force, just as distance is measured in units relative to a standard distance. One possibility is to stretch a spring a certain fixed distance, as illustrated in Figure ?.??, and use the force it exerts to pull itself back to its relaxed shape---called a restoring force---as a standard. The magnitude of all other forces can be stated as multiples of this standard unit of force. Many other possibilities exist for standard forces. (One that we will encounter in Magnetism is the magnetic force between two wires carrying electric current.) Some alternative definitions of force will be given later in this chapter.

% Skip Figure 4.4

\subsection{Newton's First Law of Motion: Inertia}

Experience suggests that an object at rest will remain at rest if left alone, and that an object in motion tends to slow down and stop unless some effort is made to keep it moving. What \gls{Newton's first law of motion} states, however, is the following:

\begin{gradient}{NEWTON'S FIRST LAW OF MOTION}
    A body at rest remains at rest, or, if in motion, remains in motion at a constant velocity unless acted on by a net external force.
\end{gradient}

Note the repeated use of the verb ``remains.'' We can think of this law as preserving the status quo of motion.

\vspace{1em}

Rather than contradicting our experience, Newton's first law of motion states that there must be a cause (which is a net external force) \textit{for there to be any change in velocity (either a change in magnitude or direction)}. We will define net external force in the next section. An object sliding across a table or floor slows down due to the net force of friction acting on the object. If friction disappeared, would the object still slow down?

\vspace{1em}

The idea of cause and effect is crucial in accurately describing what happens in various situations. For example, consider what happens to an object sliding along a rough horizontal surface. The object quickly grinds to a halt. If we spray the surface with talcum powder to make the surface smoother, the object slides farther. If we make the surface even smoother by rubbing lubricating oil on it, the object slides farther yet. Extrapolating to a frictionless surface, we can imagine the object sliding in a straight line indefinitely. Friction is thus the cause of the slowing (consistent with Newton's first law). The object would not slow down at all if friction were completely eliminated. Consider an air hockey table. When the air is turned off, the puck slides only a short distance before friction slows it to a stop. However, when the air is turned on, it creates a nearly frictionless surface, and the puck glides long distances without slowing down. Additionally, if we know enough about the friction, we can accurately predict how quickly the object will slow down. Friction is an external force.

\vspace{1em}

Newton's first law is completely general and can be applied to anything from an object sliding on a table to a satellite in orbit to blood pumped from the heart. Experiments have thoroughly verified that any change in velocity (speed or direction) must be caused by an external force. The idea of generally applicable or universal laws is important not only here---it is a basic feature of all laws of physics. Identifying these laws is like recognizing patterns in nature from which further patterns can be discovered. The genius of Galileo, who first developed the idea for the first law, and Newton, who clarified it, was to ask the fundamental question, ``What is the cause?'' Thinking in terms of cause and effect is a worldview fundamentally different from the typical ancient Greek approach when questions such as ``Why does a tiger have stripes?'' would have been answered in Aristotelian fashion, ``That is the nature of the beast.'' True perhaps, but not a useful insight.

\subsubsection*{Mass}

The property of a body to remain at rest or to remain in motion with constant velocity is called \gls{inertia}. Newton's first law is often called the \gls{law of inertia}. As we know from experience, some objects have more inertia than others. It is obviously more difficult to change the motion of a large boulder than that of a basketball, for example. The inertia of an object is measured by its mass. Roughly speaking, mass is a measure of the amount of ``stuff'' (or matter) in something. The quantity or amount of matter in an object is determined by the numbers of atoms and molecules of various types it contains. Unlike weight, mass does not vary with location. The mass of an object is the same on Earth, in orbit, or on the surface of the Moon. In practice, it is very difficult to count and identify all of the atoms and molecules in an object, so masses are not often determined in this manner. Operationally, the masses of objects are determined by comparison with the standard kilogram.

\begin{cfu}
    Which has more mass: a kilogram of cotton balls or a kilogram of gold?
\end{cfu}

\Solution They are equal. A kilogram of one substance is equal in mass to a kilogram of another substance. The quantities that might differ between them are volume and density.

\endsolution

\end{document}