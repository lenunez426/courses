\documentclass[main-ap-physics.tex]{subfiles}

\renewcommand\theadfont{\bfseries}

\begin{document}

\section{Introduction: The Nature of Science and Physics}

The physical universe is enormously complex in its detail. Every day, each of us observes a great variety of objects and phenomena. Over the centuries, the curiosity of the human race has led us collectively to explore and catalog a tremendous wealth of information. From the flight of birds to the colors of flowers, from lightning to gravity, from quarks to clusters of galaxies, from the flow of time to the mystery of the creation of the universe, we have asked questions and assembled huge arrays of facts. In the face of all these details, we have discovered that a surprisingly small and unified set of physical laws can explain what we observe. As humans, we make generalizations and seek order. We have found that nature is remarkably cooperative---it exhibits the underlying order and simplicity we so value.

\subsection{Physics: An Introduction}

\subsubsection*{Science and the Realm of Physics}

Science consists of the theories and laws that are the general truths of nature as well as the body of knowledge they encompass. Scientists are continually trying to expand this body of knowledge and to perfect the expression of the laws that describe it. Physics is concerned with describing the interactions of energy, matter, space, and time, and it is especially interested in what fundamental mechanisms underlie every phenomenon. The concern for describing the basic phenomena in nature essentially defines the realm of physics.

\subsection{Physical Quantities and Units}

The range of objects and phenomena studied in physics is immense. From the incredibly short lifetime of a nucleus to the age of the Earth, from the tiny sizes of sub-nuclear particles to the vast distance to the edges of the known universe, from the force exerted by a jumping flea to the force between Earth and the Sun, there are enough factors of 10 to challenge the imagination of even the most experienced scientist. Giving numerical values for physical quantities and equations for physical principles allows us to understand nature much more deeply than does qualitative description alone. To comprehend these vast ranges, we must also have accepted units in which to express them. And we shall find that (even in the potentially mundane discussion of meters, kilograms, and seconds) a profound simplicity of nature appears---most physical quantities can be expressed as combinations of only four fundamental physical quantities: length, mass, time, and electric current.

\vspace{1em}

We define a \gls{physical quantity} either by \textit{specifying how it is measured or by stating how it is calculated} from other measurements. For example, we define distance and time by specifying methods for measuring them, whereas we define average speed by stating that it is calculated as distance traveled divided by time of travel.

\vspace{1em}

Measurements of physical quantities are expressed in terms of \gls{units}, which are standardized values. For example, the length of a race, which is a physical quantity, can be expressed in units of meters (for sprinters) or kilometers (for distance runners). Without standardized units, it would be extremely difficult for scientists to express and compare measured values in a meaningful way. 

\vspace{1em} %Figure

There are two major systems of units used in the world: \gls{SI units} (also known as the metric system) and \gls{English units} (also known as the customary or imperial system). English units were historically used in nations once ruled by the British Empire and are still widely used in the United States. Virtually every other country in the world now uses SI units as the standard; the metric system is also the standard system agreed upon by scientists and mathematicians. The acronym ``SI'' is derived from the French \textit{Syst\`{e}me International}.

\subsubsection*{SI Units: Fundamental and Derived Units}

Table \ref{dkKzqd} gives the fundamental SI units that are used throughout this textbook. This text uses non-SI units in a few applications where they are in very common use, such as the measurement of blood pressure in millimeters of mercury (mm Hg). Whenever non-SI units are discussed, they will be tied to SI units through conversions.

\begin{table}[!htbp]
    \centering
    \begin{tabular}{|l|l|l|l|}
        \hline
        \thead{Length} & \thead{Mass} & \thead{Time} & \thead{Electric current}\\
        \hline
        meter (m) & kilogram (kg) & second (s) & ampere (A)\\
        \hline
    \end{tabular}
    \captionsetup{type=table,margin=1in,font=scriptsize}
    \captionof{table}{Fundamental SI Units}
    \label{dkKzqd}
\end{table}

It is an intriguing fact that some physical quantities are more fundamental than others and that the most fundamental physical quantities can be defined only in terms of the procedure used to measure them. The units in which they are measured are thus called \gls{fundamental units}. In this textbook, the fundamental physical quantities are taken to be length, mass, time, and electric current. (Note that electric current will not be introduced 
in AP Physics 1.)
%until much later in this text.) 
All other physical quantities, such as force and electric charge, can be expressed as algebraic combinations of length, mass, time, and current (for example, speed is length divided by time); these units are called \gls{derived units}.

\subsubsection*{Units of Time, Length, and Mass: The Second, Meter, and Kilogram}

\textbf{The Second}: The SI unit for time, the \gls{second} (abbreviated s), has a long history. For many years it was defined as 1/\num{86400} of a mean solar day. More recently, a new standard was adopted to gain greater accuracy and to define the second in terms of a non-varying, or constant, physical phenomenon (because the solar day is getting longer due to very gradual slowing of the Earth's rotation). Cesium atoms can be made to vibrate in a very steady way, and these vibrations can be readily observed and counted. In 1967 the second was redefined as the time required for \num{9192631770} of these vibrations. (See Figure ?.??.) Accuracy in the fundamental units is essential, because all measurements are ultimately expressed in terms of fundamental units and can be no more accurate than are the fundamental units themselves.

\vspace{1em}

\textbf{The Meter}: The SI unit for length is the \gls{meter} (abbreviated m); its definition has also changed over time to become more accurate and precise. The meter was first defined in 1791 as 1/\num{10000000} of the distance from the equator to the North Pole. This measurement was improved in 1889 by redefining the meter to be the distance between two engraved lines on a platinum-iridium bar now kept near Paris. By 1960, it had become possible to define the meter even more accurately in terms of the wavelength of light, so it was again redefined as \num{1650763.73} wavelengths of orange light emitted by krypton atoms. In 1983, the meter was given its present definition (partly for greater accuracy) as the distance light travels in a vacuum in 1/\num{299792458} of a second. (See Figure ?.??.) This change defines the speed of light to be exactly \num{299792458} meters per second. The length of the meter will change if the speed of light is someday measured with greater accuracy.

\vspace{1em}

\textbf{The Kilogram}: The SI unit for mass is the \gls{kilogram} (abbreviated kg); it was previously defined to be the mass of a platinum-iridium cylinder kept with the old meter standard at the International Bureau of Weights and Measures near Paris. Exact replicas of the previously defined kilogram are also kept at the United States' National Institute of Standards and Technology, or NIST, located in Gaithersburg, Maryland outside of Washington D.C., and at other locations around the world. The determination of all other masses could be ultimately traced to a comparison with the standard mass. Even though the platinum-iridium cylinder was resistant to corrosion, airborne contaminants were able to adhere to its surface, slightly changing its mass over time. In May 2019, the scientific community adopted a more stable definition of the kilogram. The kilogram is now defined in terms of the second, the meter, and Planck's constant, $h$ (a quantum mechanical value that relates a photon's energy to its frequency).

%Skip a paragraph

\subsubsection*{Metric Prefixes}

SI units are part of the \gls{metric system}. The metric system is convenient for scientific and engineering calculations because the units are categorized by factors of 10. Table ?.? gives metric prefixes and symbols used to denote various factors of 10.

\vspace{1em}

Metric systems have the advantage that conversions of units involve only powers of 10. There are 100 centimeters in a meter, 1000 meters in a kilometer, and so on. In non-metric systems, such as the system of U.S. customary units, the relationships are not as simple---there are 12 inches in a foot, 5280 feet in a mile, and so on. Another advantage of the metric system is that the same unit can be used over extremely large ranges of values simply by using an appropriate metric prefix. For example, distances in meters are suitable in construction, while distances in kilometers are appropriate for air travel, and the tiny measure of nanometers are convenient in optical design. With the metric system there is no need to invent new units for particular applications.

\vspace{1em}

The term \gls{order of magnitude} refers to the scale of a value expressed in the metric system. Each power of 10, and so forth are all different orders of magnitude. All quantities that can be expressed as a product of a specific power of 10 are said to be of the same order of magnitude. For example, the number 800 can be written as \num{8e2}, and number 450 can be written as \num{4.5e2}. Thus, the numbers 800 and 450 are of the same order of magnitude: $10^2$. Order of magnitude can be thought of as a ballpark estimate for the scale of a value. The diameter of an atom is on the order of $10^{-10}\,\text{m}$, while the diameter of the Sun is on the order of $10^9\,\text{m}$.

\begin{table}[!htbp]
    \centering
    \begin{tabular}{|l|l|l|l|l|l|l|}
        \hline
        \thead{Prefix} & \thead{Symbol} & \thead{Value} & \thead{Example} &&& \\
        \hline
        exa & E & $10^{18}$ & exameter & Em & $10^{18}\,\text{m}$& distance light travels in a century\\ 
        peta & P & $10^{15}$ & petasecond & Ps & $10^{15}\,\text{s}$ & 300 million years\\ 
        tera & T & $10^{12}$ & terawatt & TW & $10^{12}\,\text{W}$  & powerful laswer output\\ 
        giga & G & $10^9$ & gigahertz & GHz & $10^9\,\text{Hz}$ & a microwave frequency \\ 
        mega & M & $10^6$ & megacurie & MCi & $10^6\,\text{Ci}$ & high radioactivity \\ 
        kilo & k & $10^3$ & kilometer & km & $10^3\,\text{m}$ & about 6/10 mile\\ 
        hecto & h & $10^2$ & hectoliter & hL & $10^2\,\text{L}$ & 26 gallons\\ 
        deka & da & $10^1$ & dekagram & dag & $10^1\,\text{g}$ & teaspoon of butter\\ 
        - & - & $10^0 = 1$ & & & &\\ 
        deci & d & $10^{-1}$ & deciliter & dL & $10^{-1}\,\text{L}$ & less than half a soda\\ 
        centi & c & $10^{-2}$ & centimeter & cm & $10^{-2}\,\text{m}$ & fingertip thickness\\ 
        milli & m & $10^{-3}$ & millimeter & mm & $10^{-3}\,\text{m}$ & flea at its shoulders\\ 
        micro & $\mymu$ & $10^{-6}$ & micrometer & $\mymu$m & $10^{-6}\,\text{m}$ & detail in microscope\\
        nano & n & $10^{-9}$ & nanogram & ng & $10^{-9}\,\text{g}$ & small speck of dust\\ 
        pico & p & $10^{-12}$ & picofarad & pF & $10^{-12}\,\text{F}$ & small capacitor in radio\\ 
        femto & f & $10^{-15}$ & femtometer & fm & $10^{-15}\,\text{m}$ & size of a proton\\ 
        atto & a & $10^{-18}$ & attosecond & as & $10^{-18}\,\text{s}$ & time light crosses an atom\\ \hline
    \end{tabular}
    \captionsetup{type=table,margin=1in,font=scriptsize}
    \captionof{table}{Metric Prefixes for Powers of 10 and their Symbols}
\end{table}

\subsubsection*{Unit Conversion and Dimensional Analysis}

It is often necessary to convert from one type of unit to another. For example, if you are reading a European cookbook, some quantities may be expressed in units of liters and you need to convert them to cups. Or, perhaps you are reading walking directions from one location to another and you are interested in how many miles you will be walking. In this case, you will need to convert units of feet to miles.

\vspace{1em}

Let us consider a simple example of how to convert units. Let us say that we want to convert 80 meters (m) to kilometers (km).

\vspace{1em}

The first thing to do is to list the units that you have and the units that you want to convert to. In this case, we have units in \textit{meters} and we want to convert to \textit{kilometers}.

\vspace{1em}

Next, we need to determine a \gls{conversion factor} relating meters to kilometers. A conversion factor is a ratio expressing how many of one unit are equal to another unit. For example, there are 12 inches in 1 foot, 100 centimeters in 1 meter, 60 seconds in 1 minute, and so on. In this case, we know that there are \num{1000} meters in 1 kilometer.

\vspace{1em}

Now we can set up our unit conversion. We will write the units that we have and then multiply them by the conversion factor so that the units cancel out, as shown:

\begin{equation}
    \SI{80}{\cancel\meter} \times \frac{\SI{1}{km}}{\SI{1000}{\cancel\meter}} = \SI{0.080}{km}
\end{equation}

Note that the unwanted m unit cancels, leaving only the desired km unit. You can use this method to convert between any types of unit.

% Click Appendix C for a more complete list of conversion factors.

\begin{example} \label{W2DilL}
    Suppose that you drive the \SI{10.0}{km} from your school to home in \SI{20.0}{min}. Calculate your average speed in kilometers per hour (km/h), using the formula

    \begin{equation}
        \text{average speed} = \frac{\text{distance}}{\text{time}}
    \end{equation}
\end{example}

\Solution First we calculate the average speed using the given units. Then we can get the average speed into the desired units by picking the correct conversion factor and multiplying by it. The correct conversion factor is the one that cancels the unwanted unit and leaves the desired unit in its place.

\vspace{1em}

We are given distance and time: $\text{distance} = \SI{10.0}{km}$ and $\text{time} = \SI{20.0}{min}$. Therefore, average speed is

\begin{equation*}
    \text{average speed} = \frac{\SI{10.0}{km}}{\SI{20.0}{min}} = \SI[per-mode=fraction]{0.5}{\kilo\meter\per\minute}
\end{equation*}

Next, convert km/min to km/h: multiply by the conversion factor that will cancel minutes and leave hours. That conversion factor comes from the fact that there are 60 minutes in 1 hour:

\begin{equation*}
    \SI{1}{hr} = \SI{60}{min}
\end{equation*}

Therefore, in kilometers per hour, the average speed is

\begin{equation*}
    \SI[per-mode=fraction]{0.5}{\kilo\meter\per\cancel\minute} \times \frac{\SI{60}{\cancel\minute}}{\SI{1}{hr}} = \SI[per-mode=fraction]{30}{\kilo\meter\per\hour}
\end{equation*}

Therefore, the answer is 30 kilometers per hour. To check your answer, consider the following three points: 

\vspace{1em}

(1) Be sure that you have properly cancelled the units in the unit conversion. If you have written the unit conversion factor upside down, the units will not cancel properly in the equation. If you accidentally get the ratio upside down, then the units will not cancel; rather, they will give you the wrong units as follows:

\begin{equation*}
    \frac{\text{km}}{\text{min}} \times \frac{\SI{1}{h}}{\SI{60}{min}} = \frac{1}{60} \frac{\text{km} \cdot \text{hr}}{\text{min}^2}
\end{equation*}

which are obviously not the desired units of km/h.

\vspace{1em}

(2) Check that the units of the final answer are the desired units. The problem asked us to solve for average speed in units of km/h and we have indeed obtained these units.

\vspace{1em}

(3) Next, check whether the answer is reasonable. Let us consider some information from the problem---if you travel \SI{10}{km} in a third of an hour (\SI{20}{min}), you would travel three times that far in an hour. The answer does seem reasonable.

\endsolution

\begin{example}
    Solve Example \ref{W2DilL} in units of meters per second (m/s).
\end{example}

\Solution In Example \ref{W2DilL}, we arrived at the answer

\begin{equation*}
    \text{average speed} = \SI[per-mode=fraction]{30}{\kilo\meter\per\hour}
\end{equation*}

There are several ways to convert this average speed into meters per second. Two conversion factors are needed---one to convert hours to seconds, and another to convert kilometers to meters. Since there are 3600 seconds in 1 hour, and 1000 meters in 1 kilometer, we may covert average speed as

\begin{equation*}
    \text{average speed} = \SI[per-mode=fraction]{30}{\kilo\meter\per\hour} \times \frac{\SI{1}{hr}}{\SI{3600}{s}} \times \frac{\SI{1000}{m}}{\SI{1}{km}} = \SI[per-mode=fraction]{8.33}{\meter\per\second}
\end{equation*}

If we had started with \SI{0.500}{km/min}, we would have needed different conversion factors, but the answer would have been the same: \SI{8.33}{m/s}.

\endsolution








\end{document}