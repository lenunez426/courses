\makenoidxglossaries

\newglossaryentry{cardinal directions}{
    name=cardinal directions,
    description={the four directions---north, south, east, and west---that point to northern, southern, eastern, or western geographical directions}
}

\newglossaryentry{planet}{
    name=planet,
    description={today, any of the larger objects revolving about the Sun or any similar objects that orbit other stars; in ancient times, any object that moved regularly among the fixed stars}
}

\newglossaryentry{constellation}{
    name=constellation,
    description={one of the 88 sections into which astronomers divide the sky, each named after a prominent star pattern within it}
}

\newglossaryentry{apparent magnitude}{
    name=apparent magnitude,
    description={a measure of how bright a star looks in the sky; the larger the number, the dimmer the star appears to us}
}

\newglossaryentry{geocentric}{
    name=geocentric,
    description={centered on Earth}
}

\newglossaryentry{heliocentric}{
    name=heliocentric,
    description={centered on the Sun}
}

\newglossaryentry{horizon}{
    name=horizon,
    description={a great circle on the celestial sphere \SI{90}{\degree} from the zenith; more popularly, the circle around us where the dome of the sky meets Earth},
}

\newglossaryentry{zenith}{
    name=zenith,
    description={the point on the celestial sphere opposite the direction of gravity; point directly above the observer}
}

\newglossaryentry{year}{
    name=year,
    description={the period of revolution of Earth around the Sun}
}

\newglossaryentry{celestial equator}{
    name=celestial equator,
    description={a great circle on the celestial sphere \SI{90}{\degree} from the celestial poles; where the celestial sphere intersects the plane of Earth’s equator}
}

\newglossaryentry{celestial sphere}{
    name=celestial sphere,
    description={the apparent sphere of the sky; a sphere of large radius centered on the observer; the dome on which the stars are fixed; directions of objects in the sky can be denoted by their position on the celestial sphere}
}

\newglossaryentry{celestial poles}{
    name=celestial poles,
    description={points about which the celestial sphere appears to rotate; intersections of the celestial sphere with Earth’s polar axis}
}

\newglossaryentry{ecliptic}{
    name=ecliptic,
    description={the apparent annual path of the Sun on the celestial sphere}
}

\newglossaryentry{circumpolar zone}{
    name=circumpolar zone,
    description={those portions of the celestial sphere near the celestial poles that are either always above or always below the horizon}
}