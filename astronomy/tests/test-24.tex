\documentclass{exam}
\usepackage{preamble}
\sisetup{group-separator = {,}}
\usepackage{exam-randomizechoices}

\def\one{10}
\def\two{2}
\def\three{3}
\def\four{4}

\def\testVersion{\one} % Edit Version number here only.

\setrandomizerseed{\testVersion}

\pagestyle{headandfoot}
\runningheadrule

\firstpagefooter{Access for free at \href{\openstax}{\openstax}}{}{}
\runningfooter{Access for free at \href{\openstax}{\openstax}}{}{}
\firstpageheader{Astronomy}{Study Guide for Test on Unit 8: Curved Spacetime}{}

\CorrectChoiceEmphasis{\color{red}\bfseries}
\SolutionEmphasis{\color{red}}

\printanswers

\begin{document}
\begin{questions}

\question
What is the general theory of relativity?

\begin{randomizechoices}
    \correctchoice the theory relating gravity and the structure of space and time
    \choice the theory that aims to describe the motion of subatomic particles, like electrons
    \choice the theory about how to travel back in time
    \choice the theory about gravitational attract proposed by Newton in 1687
\end{randomizechoices}


\question
Which scientist first developed the theory of general relativity?

\begin{randomizechoices}
    \correctchoice Albert Einstein (1915)
    \choice Isaac Newton (1687)
    \choice Annie Jump Cannon (1899)
    \choice Stephen Hawking (2001)
\end{randomizechoices}

\question
Which of the following best exemplifies the principle of equivalence?

\begin{randomizechoices}
    \correctchoice For a weightless observer in a windowless frame, being in free fall feels the same as being in zero gravity.
    \choice Energy and mass are equivalent in accordance with the equation $E = mc^2$.
    \choice In flat space, the distance between three equally space points in equivalent.
    \choice All black holes have equivalent volume and density.
\end{randomizechoices}

\question
Which of the following is a direct conclusion of the principle of equivalence?

\begin{randomizechoices}
    \correctchoice Experiments conducted in zero gravity must yield the same results as experiments conducted in free fall where gravity is present.
    \choice Knowing the speed of light ($c=\SI{3e8}{m/s}$) and and an object's mass in kilograms may lead to calculation of its energy in joules.
    \choice Black holes have zero volume and infinite density.
    \choice In flat space, one may use the Pythagorean Theorem to calculate the long side of a right triangle.
\end{randomizechoices}

\question
According to Einstein's theory of general relativity, mass associated with large concentrations of matter distorts spacetime, such that light near such massive objects does not travel in straight lines but in \fillin\ .

\begin{randomizechoices}
\correctchoice curves
\choice circles
\choice ellipses
\choice rays
\end{randomizechoices}

\question
Why is the distortion of spacetime much greater near the surface of a white dwarf or neutron star than near the surface of the Sun, even when these objects have comparable masses?

\begin{randomizechoices}
\correctchoice white dwarfs and neutron stars are more compact
\choice white dwarfs and neutron stars are dimmer
\choice white dwarfs and neutron stars are hotter
\choice white dwarfs and neutron stars are larger
\end{randomizechoices}

\question
The general theory of relativity states that \fillin\ determines the curvature of spacetime.

\begin{randomizechoices}
\correctchoice mass
\choice radius
\choice equivalence
\choice empty space
\end{randomizechoices}

\question
When the distorting mass is small, the predictions of Einstein's theory of general relativity agrees with predictions resulting from

\begin{randomizechoices}
\correctchoice Newton's law of universal gravitation 
\choice Darwin's theory of evolution
\choice quantum mechanics
\choice string theory
\end{randomizechoices}

\question
Of all the planets, why is Mercury the most affected by the distortion of spacetime produced by the Sun's mass?

\begin{randomizechoices}
\correctchoice Mercury is very close to the Sun.
\choice Mercury is the smallest.
\choice Mercury is extremely compact.
\choice Mercury is near a black hole.
\end{randomizechoices}

\question
According to Einstein, what happens to Mercury as it orbits the curved spacetime near the Sun?

\begin{randomizechoices}
\correctchoice It experiences a noticeable difference in motion that is not predicted by Newton's laws.
\choice It travels in a straight line.
\choice It orbits in an perfect circle.
\choice It behaves exactly as predicted by Newton's laws.
\end{randomizechoices}

\question
As early as 1859, way before Einstein, astronomers observed that Mercury's orbit slightly varied from Newtonian prediction. They believed this variation was caused by an undiscovered inner planet (which was never actually found). What did they name this imaginary planet?

\begin{randomizechoices}
\correctchoice Vulcan 
\choice Nibiru
\choice X
\choice Fitness
\end{randomizechoices}

\question
According to the general theory of relativity, light passing very close to the Sun is expected to

\begin{randomizechoices}
\correctchoice follow a curved path
\choice travel a straight line
\choice orbit the Sun
\choice disappear beyond the Sun's event horizon
\end{randomizechoices}

\question
Einstein calculated from general relativity theory that starlight just grazing the Sun’s surface should be deflected by an angle of

\begin{randomizechoices}
\correctchoice 1.75 arcseconds
\choice 56 arcseconds
\choice 1 million arcseconds
\choice 45 arcseconds
\end{randomizechoices}

\question
In order to reveal the deflection of light passing near the Sun, Einstein suggested\ldots 

\begin{randomizechoices}
\correctchoice taking photographic observation during a solar eclipse
\choice building a spacecraft that would travel close to the Sun
\choice using special telescopes to safely observe the Sun
\choice use a prism to study the Sun's spectrum
\end{randomizechoices}

\question
Who organized the expeditions to Brazil and West Africa in 1919, for the purpose of photographing the solar eclipse with the shifting of background stars, that would validate Einstein's general relativity?

\begin{randomizechoices}
    \correctchoice Arthur Eddington
    \choice Albert Einstein
    \choice Albert Einstein's cousin
    \choice Cecilia Payne
\end{randomizechoices}

\question
One of the remarkable predictions of Einstein's general relativity is the existence of black holes in the universe. What is a black hole?

\begin{randomizechoices}
    \correctchoice A region in spacetime where gravity is so strong that nothing---not even light---can escape
    \choice A large region of space where no stars, galaxies, or other light sources exist
    \choice A collection of dust, gas, and other space debris that absorb light and prevent light particles from reflecting
    \choice A wormhole in the fabric of space and time
\end{randomizechoices}

\question
What happens when an object or a photon of light crosses a black hole's event horizon?

\begin{randomizechoices}
    \correctchoice It has entered a region where it can no longer escape the gravitational pull of the black hole.
    \choice It will continue in a curved path until it escapes the other end of the event horizon.
    \choice It will bounce back as it is denied entry to the black hole region.
    \choice It will orbit around the perimeter of the event horizon.
\end{randomizechoices}



\end{questions}
\end{document}