\documentclass[addpoints]{exam}
\usepackage{preamble}
\sisetup{group-separator = {,}}
%\extrawidth{-2.5in}

\pagestyle{headandfoot}
\runningheadrule


\firstpagefooter{Access for free at \href{https://openstax.org/books/astronomy-2e/pages/1-introduction}{https://openstax.org/books/astronomy-2e/pages/1-introduction}}{}{}
\runningfooter{Access for free at \href{https://openstax.org/books/astronomy-2e/pages/1-introduction}{https://openstax.org/books/astronomy-2e/pages/1-introduction}}{}{}


\firstpageheader{Astronomy}{Chapter 3: {\small Orbits and Gravity}}{Problem Set}


\CorrectChoiceEmphasis{\color{red}\bfseries}
\SolutionEmphasis{\color{red}}
\printanswers

\begin{document}


\begin{questions}

\question
In what year was Tycho Brahe born?

\begin{choices}
    \choice 1609
    \choice 1543
    \choice 1776
    \correctchoice 1546
\end{choices}

\question
What was Tycho Brahe's greatest contribution to astronomy?

\begin{choices}
    \choice He discovered and wrote the 3 laws of planetary motion.
    \choice He was a gifted mathematician who developed of Law of Universal Gravitation.
    \correctchoice He made excellent measurements of solar and planetary positions for 20 years. 
    \choice He observed the Moon and planets with a telescope.
\end{choices}

\question
True or False? Tycho Brahe was skilled at using a telescope to record planetary motions.

\begin{choices}
    \choice True
    \correctchoice False
\end{choices}

% \question
% What prevented Brahe from creating a better model of planetary motion than Ptolemy's ancient model?

\question
Who did Brahe hire to analyze his data?

\begin{choices}
    \choice Isaac Newton
    \choice Galileo Galilei
    \choice Thomas Jefferson
    \correctchoice Johannes Kepler
\end{choices}

\question
In what year was Johannes Kepler born?

\begin{choices}
    \choice 1609
    \choice 1546
    \choice 1630
    \correctchoice 1571
\end{choices}

\question
Kepler studied \fillin[theology] at the University of Tubingen.

\begin{choices}
    \correctchoice theology 
    \choice biology
    \choice chemistry
    \choice anatomy
\end{choices}

\question
What idea or system did Kepler encounter during his time in the university?

\begin{choices}
    \choice Newton's Law of Universal Gravitation.
    \choice The 3 Laws of Planetary Motion.
    \choice Darwin's theory of evolution.
    \correctchoice Copernicus's heliocentric (Sun-centered) hypothesis of the solar system.
\end{choices}

% \question
% Who hired Kepler? For what purpose was Kepler hired?

\question
What did Kepler inherit from Brahe?

\begin{choices}
    \choice A lot of money.
    \choice His observatory.
    \correctchoice Two decades' worth of accurate data on planetary positions in the sky. 
    \choice A manual on how to use the telescope for astronomical observation.
\end{choices}

\clearpage
\question
What is an orbit?

\begin{choices}
    \choice the amount of time it takes a planet to go around the Sun
    \choice the distance between the Sun and a planet
    \correctchoice the path a planet takes around the Sun
    \choice the Houston Astros mascot
\end{choices}


\question
Initially Kepler assumed the shape of a planet's orbit was \fillin\ .

\begin{choices}
    \correctchoice a circle
    \choice an ellipse
    \choice a parabola
    \choice a square
\end{choices}

\question
What made Kepler change his mind about his original assumption of the shape of orbits?

\begin{choices}
    \choice Brahe explained to him the correct answer. 
    \choice He read about it in Copernicus's book.
    \correctchoice The shape of the orbit of Mars refuted his hypothesis.
    \choice He had a dream in which angels revealed to him the truth.
\end{choices}

\question
What is an ellipse?

\begin{choices}
    \choice a perfect circle
    \correctchoice a flattened circle
    \choice when the Moon blocks the Sun (or vice-versa)
    \choice a flattened square
\end{choices}

\question
Kepler's First Law of motion states the shape of the orbit of a planet is \fillin\ .

\begin{choices}
    \correctchoice an ellipse 
    \choice a circle
    \choice a rectangle
    \choice a triangle
\end{choices}


% \question
% Kepler's Second Law of Motion
% states that when a planet is closer to the Sun, it moves \fillin[faster], and when it's farther away, it
% moves \fillin[slower], but it sweeps out equal areas in an equal times.

\question
What is Kepler's Third Law of Motion?

\begin{choices}
    \choice Every action has an equal and opposite reaction.
    \choice Net force equals mass times acceleration ($F_{\text{net}} = m a$)
    \correctchoice Orbital period squared is proportional to semimajor axis cubed ($P^2 \propto a^3$)
    \choice The square of the hypotenuse equals the sum of the square of sides ($a^2 + b^2 = c^2$).
\end{choices}

% \question
% In Kepler's Third Law, what does the $P$ stand for? What units should be used?

% \question
% In Kepler's Third Law, what does the $a$ stand for? What units should be used?

\question
Saturn has a semimajor axis of \SI{9.54}{AU}. What is Saturn's orbital period?

\begin{choices}
    \choice \SI{11.8}{yr}
    \choice \SI{27.65}{yr}
    \correctchoice \SI{29.46}{yr}
    \choice \SI{84.5}{yr}
\end{choices}

\question
Imagine a new planet was discovered, beyond the orbit of Neptune, with a semimajor axis of \SI{34}{AU}. How long would it take the planet to orbit the Sun? (Find its orbital period.)

\begin{choices}
    \correctchoice 198 years
    \choice 102 years
    \choice 460 years
    \choice 34 years
\end{choices}

\question
Calculate the magnitude of the gravitational force exerted by the Sun on Jupiter. Use the information below.

\begin{center}
    \begin{tabular}{c|c|c}
        \textbf{Jupiter mass} & \textbf{Sun mass} & \textbf{Sun-to-Jupiter Distance}\\
        \hline
        \SI{1898e24}{kg} & \SI{1.99e30}{kg} & \SI{778.5e9}{m}\\
    \end{tabular}
\end{center}

\question
Calculate the magnitude of the gravitational force exerted by the Sun on the Moon. Use the information below.

\begin{center}
    \begin{tabular}{c|c|c}
        \textbf{Moon mass} & \textbf{Sun mass} & \textbf{Sun-to-Moon Distance}\\
        \hline
        \SI{0.073e24}{kg} & \SI{1.99e30}{kg} & \SI{149.6e9}{m}\\
    \end{tabular}
\end{center}

\begin{choices}
    \choice 
    \choice 
    \choice 
    \correctchoice \SI{4.3e20}{N} 
\end{choices}

\question
How does the gravitational force exerted by the Sun on \textbf{Jupiter} compare to the force by the Sun on Earth? Jupiter's semimajor axis is \SI{5.2}{AU}

\question
How does the gravitational force exerted by the Sun on the \textbf{Moon} compare to the force by the Sun on Earth? Assume the Moon is \SI{1}{AU} from the Sun. The Moon's mass is \SI{0.073e24}{kg}.






% \begin{choices}
%     \correctchoice He recorded 20 years' worth of precise planetary motions.
%     \choice He used the telescope to observe the Jupiter's moons.
%     \choice He discovered that sunlight is made of the colors of the rainbow.
%     \choice He wrote 3 laws of planetary motion.
% \end{choices}

% \question
% True or False? Brahe created a new system for planetary motion that overthrew Ptolemy's outdated system.

% \begin{choices}
%     \choice True
%     \correctchoice False
% \end{choices}

    
\end{questions}
\end{document}