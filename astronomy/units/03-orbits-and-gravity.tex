\documentclass[../../main-astronomy.tex]{subfiles}

\begin{document}

\section{Orbits and Gravity}

\subsection{The Laws of Planetary Motion}

At about the time that Galileo was beginning his experiments with falling bodies, the efforts of two other scientists dramatically advanced our understanding of the motions of the planets. These two astronomers were the observer Tycho Brahe and the mathematician Johannes Kepler. Together, they placed the speculations of Copernicus on a sound mathematical basis and paved the way for the work of Isaac Newton in the next century.

\subsubsection*{Tycho Brahe's Observatory}

Three years after the publication of Copernicus' \textit{De Revolutionibus}, Tycho Brahe was born to a family of Danish nobility. He developed an early interest in astronomy and, as a young man, made significant astronomical observations. Among these was a careful study of what we now know was an exploding star that flared up to great brilliance in the night sky. His growing reputation gained him the patronage of the Danish King Frederick II, and at the age of 30, Brahe was able to establish a fine astronomical observatory on the North Sea island of Hven (Figure ?.??). Brahe was the last and greatest of the pre-telescopic observers in Europe.

\vspace{1em} %insert figure

At Hven, Brahe made a continuous record of the positions of the Sun, Moon, and planets for almost 20 years. His extensive and precise observations enabled him to note that the positions of the planets varied from those given in published tables, which were based on the work of Ptolemy. These data were extremely valuable, but Brahe didn't have the ability to analyze them and develop a better model than what Ptolemy had published. He was further inhibited because he was an extravagant and cantankerous fellow, and he accumulated enemies among government officials. When his patron, Frederick II, died in 1597, Brahe lost his political base and decided to leave Denmark. He took up residence in Prague, where he became court astronomer to Emperor Rudolf of Bohemia. There, in the year before his death, Brahe found a most able young mathematician, Johannes Kepler, to assist him in analyzing his extensive planetary data.

\vspace{1em}

\subsubsection*{Johannes Kepler}

Johannes Kepler was born into a poor family in the German province of W\"{u}rttemberg and lived much of his life amid the turmoil of the Thirty Years' War (see Figure ?.??). He attended university at Tubingen and studied for a theological career. There, he learned the principles of the Copernican system and became converted to the heliocentric hypothesis. Eventually, Kepler went to Prague to serve as an assistant to Brahe, who set him to work trying to find a satisfactory theory of planetary motion---one that was compatible with the long series of observations made at Hven. Brahe was reluctant to provide Kepler with much material at any one time for fear that Kepler would discover the secrets of the universal motion by himself, thereby robbing Brahe of some of the glory. Only after Brahe's death in 1601 did Kepler get full possession of the priceless records. Their study occupied most of Kepler's time for more than 20 years.

\vspace{1em}

Through his analysis of the motions of the planets, Kepler developed a series of principles, now known as \textit{Kepler's three laws}, which described the behavior of planets based on their paths through space. The first two laws of planetary motion were published in 1609 in \textit{The New Astronomy}. Their discovery was a profound step in the development of modern science.


\subsubsection{The First Two Laws of Planetary Motion}

The path of an object through space is called its \gls{orbit}. Kepler initially assumed that the orbits of planets were circles, but doing so did not allow him to find orbits that were consistent with Brahe's observations. Working with the data for Mars, he eventually discovered that the orbit of that planet had the shape of a somewhat flattened circle, or \gls{ellipse}. Next to the circle, the ellipse is the simplest kind of closed curve, belonging to a family of curves known as conic sections (see Figure ?.??)

\vspace{1em} %Insert figure

You might recall from math classes that in a circle, the center is a special point. The distance from the center to anywhere on the circle is exactly the same. In an ellipse, the sum of the distance from two special points inside the ellipse to any point on the ellipse is always the same. These two points inside the ellipse are called its foci (singular: \gls{focus}), a word invented for this purpose by Kepler.

\vspace{1em}

This property suggests a simple way to draw an ellipse (Figure ?.??). We wrap the ends of a loop of string around two tacks pushed through a sheet of paper into a drawing board, so that the string is slack. If we push a pencil against the string, making the string taut, and then slide the pencil against the string all around the tacks, the curve that results is an ellipse. At any point where the pencil may be, the sum of the distances from the pencil to the two tacks is a constant length---the length of the string. The tacks are at the two foci of the ellipse.

\vspace{1em}

The widest diameter of the ellipse is called its \gls{major axis}. Half this distance---that is, the distance from the center of the ellipse to one end---is the \gls{semimajor axis}, which is usually used to specify the size of the ellipse. For example, the semimajor axis of the orbit of Mars, which is also the planet's average distance from the Sun, is 228 million kilometers.

\vspace{1em}

The shape (roundness) of an ellipse depends on how close together the two foci are, compared with the major axis. The ratio of the distance between the foci to the length of the major axis is called the \gls{eccentricity} of the ellipse.

\vspace{1em}

If the foci (or tacks) are moved to the same location, then the distance between the foci would be zero. This means that the eccentricity is zero and the ellipse is just a circle; thus, a circle can be called an ellipse of zero eccentricity. In a circle, the semimajor axis would be the radius.

\vspace{1em}

Next, we can make ellipses of various elongations (or extended lengths) by varying the spacing of the tacks (as long as they are not farther apart than the length of the string). The greater the eccentricity, the more elongated is the ellipse, up to a maximum eccentricity of 1.0, when the ellipse becomes ``flat,'' the other extreme from a circle.

\vspace{1em}

The size and shape of an ellipse are completely specified by its semimajor axis and its eccentricity. Using Brahe's data, Kepler found that Mars has an elliptical orbit, with the Sun at one focus (the other focus is empty). The eccentricity of the orbit of Mars is only about 0.1; its orbit, drawn to scale, would be practically indistinguishable from a circle, but the difference turned out to be critical for understanding planetary motions.

\vspace{1em}

Kepler generalized this result in his first law and said that the orbits of all the planets are ellipses. Here was a decisive moment in the history of human thought: it was not necessary to have only circles in order to have an acceptable cosmos. The universe could be a bit more complex than the Greek philosophers had wanted it to be.

\vspace{1em}

Kepler's second law deals with the speed with which each planet moves along its ellipse, also known as its \gls{orbital speed}. Working with Brahe's observations of Mars, Kepler discovered that the planet speeds up as it comes closer to the Sun and slows down as it pulls away from the Sun. He expressed the precise form of this relationship by imagining that the Sun and Mars are connected by a straight, elastic line. When Mars is closer to the Sun (positions 1 and 2 in Figure ?.??), the elastic line is not stretched as much, and the planet moves rapidly. Farther from the Sun, as in positions 3 and 4, the line is stretched a lot, and the planet does not move so fast. As Mars travels in its elliptical orbit around the Sun, the elastic line sweeps out areas of the ellipse as it moves (the colored regions in our figure). Kepler found that in equal intervals of time ($t$), the areas swept out in space by this imaginary line are always equal; that is, the area of the region B from 1 to 2 is the same as that of region A from 3 to 4.

\vspace{1em}

If a planet moves in a circular orbit, the elastic line is always stretched the same amount and the planet moves at a constant speed around its orbit. But, as Kepler discovered, in most orbits that speed of a planet orbiting its star (or moon orbiting its planet) tends to vary because the orbit is elliptical.

\begin{center}
\begin{tikzpicture}[
    declare function={f(\x)=\a*(1-\e^2)/(1+\e*cos(\x));}
  ]
\tikzmath{
        \a = 1;
        \e = 0.6;
        \b = \a*sqrt(1-\e^2);
        \c = sqrt(\a^2 - \b^2);
        % The following four angles came from Rebound using same elapsed time. Use mathematica to check that areas swept are equal.
        \fone = 5.6859; 
        \ftwo = 0.59715;
        \fthree = 3.1025;
        \ffour = 3.18068;
    }
  \begin{polaraxis}[width=15cm,
  axis lines=none,
  axis equal image,
  clip=false
  ]
    \addplot[domain=0:2*pi r,samples=300,color=RoyalBlue,thick] {f(\x)};
    \fill[thick,domain=\fone r:2*pi r,fill=yellow!80] (0,0) -- plot (\x,{f(\x)}); 
    \fill[thick,domain=0:\ftwo r,fill=yellow!80] (0,0) -- plot (\x,{f(\x)});
    \fill[thick,domain=\fthree r:\ffour r,fill=yellow!80] (0,0) -- plot (\x,{f(\x)});
    \draw[fill=yellow!80] (0,0) circle (1.5mm) node[above=1.5mm] {Sun};

    % Annotations
    \node[below] at (\fone r,{f(\fone)*1.2}) {1};
    \draw[domain=\fone r:2*pi r] plot (\x,{f(\x)*1.2});
    \draw[domain=0:\ftwo r,->] plot (\x,{f(\x)*1.2});
    \node[right=2pt] at (0,{f(0)*1.2}) {$t$};
    \node[above] at (\ftwo r,{f(\ftwo)*1.2}) {2};
    \node[left=2mm] at (0,{f(0)}) {B};
    
    \node[above] at (\fthree r,{f(\fthree r)*1.05}) {3};
    \draw[domain=\fthree r:\ffour r,->] plot (\x,{f(\x)*1.05});
    \node[below] at (\ffour r,{f(\ffour r)*1.05}) {4};
    \node[left=2pt] at (pi r,{f(pi r)*1.05}) {$t$};
    \node[right=2mm] at (pi r,{f(pi r)}) {A};
  \end{polaraxis}
\end{tikzpicture}
\captionsetup{type=figure,margin=1in,font=scriptsize}
\captionof{figure}{\textbf{Kepler's Second Law: The Law of Equal Areas}. The orbital speed of a planet traveling around the Sun (the circular object inside the ellipse) varies in such a way that in equal intervals of time (t), a line between the Sun and a planet sweeps out equal areas (A and B). Note that the eccentricities of the planets' orbits in our solar system are substantially less than shown here.}
\end{center}

\begin{gradient}{LINK TO LEARNING}
    The \href{https://openstax.org/l/30kepsecond}{Kepler's Second Law demonstrator} from CCNY's ScienceSims project shows how an orbiting planet sweeps out the same area in the same time.
\end{gradient}

\subsubsection*{Kepler's Third Law}

Kepler's first two laws of planetary motion describe the shape of a planet's orbit and allow us to calculate the speed of its motion at any point in the orbit. Kepler was pleased to have discovered such fundamental rules, but they did not satisfy his quest to fully understand planetary motions. He wanted to know why the orbits of the planets were spaced as they are and to find a mathematical pattern in their movements---a ``harmony of the spheres'' as he called it. For many years he worked to discover mathematical relationships governing planetary spacing and the time each planet took to go around the Sun.

\vspace{1em}

In 1619, Kepler discovered a basic relationship to relate the planets' orbits to their relative distances from the Sun. We define a planet's \gls{orbital period}, $P$, as the time it takes a planet to travel once around the Sun. Also, recall that a planet's semimajor axis, $a$, is equal to its average distance from the Sun. The relationship, now known as Kepler's third law, says that a planet's orbital period squared is proportional to the semimajor axis of its orbit cubed, or

\begin{equation*}
    P^2 \propto a^3
\end{equation*}

When $P$ (the orbital period) is measured in years, and $a$ is expressed in a quantity known as an \gls{astronomical unit} (AU), the two sides of the formula are not only proportional but equal. One AU is the average distance between Earth and the Sun and is approximately equal to \SI{1.5e8}{} kilometers. In these units,

\begin{equation*}
    P^2 = a^3
\end{equation*}

Kepler's third law applies to all objects orbiting the Sun, including Earth, and provides a means for calculating their relative distances from the Sun from the time they take to orbit. Let's look at a specific example to illustrate how useful Kepler's third law is.

\vspace{1em}

For instance, suppose you time how long Mars takes to go around the Sun (in Earth years). Kepler's third law can then be used to calculate Mars' average distance from the Sun. Mars' orbital period (1.88 Earth years) squared, or $P^2$, is  $1.88^2 = 3.53$, and according to the equation for Kepler's third law, this equals the cube of its semimajor axis, or $a^3$. So what number must be cubed to give 3.53? The answer is 1.52 (since $1.52 \times 1.52 \times 1.52 = 3.53$). Thus, Mars' semimajor axis in astronomical units must be \SI{1.52}{AU}. In other words, to go around the Sun in a little less than two years, Mars must be about 50\% (half again) as far from the Sun as Earth is.

\begin{example}
    \textit{Calculating Periods}\\
    Imagine an object is traveling around the Sun. What would be the orbital period of the object if its orbit has a semimajor axis of \SI{50}{AU}?
\end{example}

\Solution From Kepler's third law, we know that (when we use units of years and AU)

\begin{equation*}
    P^2 = a^3
\end{equation*}

If the object's orbit has a semimajor axis of \SI{50}{AU} ($a = 50$), we can cube 50 and then take the square root of the result to get $P$:

\begin{align*}
    P &= \sqrt{a^3} \\[1ex]
    P &= \sqrt{50 \times 50 \times 50} = \sqrt{\num{125000}} = \SI{353.6}{years}
\end{align*}

Therefore, the orbital period of the object is about 350 years. This would place our hypothetical object beyond the orbit of Pluto.

\endsolution

\begin{cfu}
    What would be the orbital period of an asteroid (a rocky chunk between Mars and Jupiter) with a semimajor axis of \SI{3}{AU}?
\end{cfu}

\Solution $P = \sqrt{3 \times 3 \times 3 \times} = \sqrt{27} = \SI{5.2}{years}$

\endsolution

Kepler's three laws of planetary motion can be summarized as follows:

\begin{itemize}
    \item \gls{Kepler's first law}: Each planet moves around the Sun in an orbit that is an ellipse, with the Sun at one focus of the ellipse.
    \item \gls{Kepler's second law}: The straight line joining a planet and the Sun sweeps out equal areas in space in equal intervals of time.
    \item \gls{Kepler's third law}: The square of a planet's orbital period is directly proportional to the cube of the semimajor axis of its orbit.
\end{itemize}

Kepler's three laws provide a precise geometric description of planetary motion within the framework of the Copernican system. With these tools, it was possible to calculate planetary positions with greatly improved precision. Still, Kepler's laws are purely descriptive: they do not help us understand what forces of nature constrain the planets to follow this particular set of rules. That step was left to Isaac Newton.

\begin{example}
    \textit{Applying Kepler's Law}\\
    Using the orbital periods and semimajor axes for Venus and Earth that are provided here, calculate $P^2$ and $a^3$, and verify that they obey Kepler's third law. Venus' orbital period is \SI{0.62}{year}, and its semimajor axis is \SI{0.72}{AU}. Earth's orbital period is \SI{1.00}{year}, and its semimajor axis is \SI{1.00}{AU}.
\end{example}

\Solution We can use the equation for Kepler’s third law, $P^2 \propto a^3$. For Venus, $P^2= 0.62 \times 0.62 = 0.38$ and $a^3 = 0.72 \times 0.72 \times 0.72 =0.37$ (rounding numbers sometimes causes minor discrepancies like this). The square of the orbital period (0.38) approximates the cube of the semimajor axis (0.37). Therefore, Venus obeys Kepler's third law. For Earth,  $P^2= 1.00 \times 1.00 = 1.00$ and $a^3 = 1.00 \times 1.00 \times 1.00 = 1.00$. The square of the orbital period (1.00) approximates (in this case, equals) the cube of the semimajor axis (1.00). Therefore, Earth obeys Kepler's third law.

\endsolution

\begin{cfu}
    Using the orbital periods and semimajor axes for Saturn and Jupiter that are provided here, calculate $P^2$ and $a^3$, and verify that they obey Kepler's third law. Saturn's orbital period is 29.46 years, and its semimajor axis is \SI{9.54}{AU}. Jupiter's orbital period is 11.86 years, and its semimajor axis is \SI{5.20}{AU}.
\end{cfu}

\Solution For Saturn, $P^2 = 29.46 \times 29.46 = 867.9$ and $a^3 = 9.54 \times 9.54 \times 9.54 = 868.3$. The square of the orbital period (867.9) approximates the cube of the semimajor axis (868.3). Therefore, Saturn obeys Kepler's third law.

\endsolution

\begin{gradient}{LINK TO LEARNING}
    In honor of the scientist who first devised the laws that govern the motions of planets, the team that built the first spacecraft to search for planets orbiting other stars decided to name the probe ``Kepler.'' Visit NASA's Kepler website to learn more about Johannes Kepler's life and his laws of planetary motion. \href{https://openstax.org/l/30nasakepmiss}{NASA's Kepler website} and follow the links that interest you.
\end{gradient}

\subsection{Newton's Great Synthesis}

It was the genius of Isaac Newton that found a conceptual framework that completely explained the observations and rules assembled by Galileo, Brahe, Kepler, and others. Newton was born in Lincolnshire, England, in the year after Galileo's death (Figure ?.??). Against the advice of his mother, who wanted him to stay home and help with the family farm, he entered Trinity College at Cambridge in 1661 and eight years later was appointed professor of mathematics. Among Newton's contemporaries in England were architect Christopher Wren, authors Aphra Behn and Daniel Defoe, and composer G.~F.~Handel.

\vspace{1em} %Skip figure

\subsubsection*{Newton's Laws of Motion}

As a young man in college, Newton became interested in natural philosophy, as science was then called. He worked out some of his first ideas on machines and optics during the plague years of 1665 and 1666, when students were sent home from college. Newton, a moody and often difficult man, continued to work on his ideas in private, even inventing new mathematical tools to help him deal with the complexities involved. Eventually, his friend Edmund Halley (profiled in ``Comets and Asteroids: Debris of the Solar System'') prevailed on him to collect and publish the results of his remarkable investigations on motion and gravity. The result was a volume that set out the underlying system of the physical world, \textit{Philosophiae Naturalis Principia Mathematica}. The \textit{Principia}, as the book is generally known, was published at Halley's expense in 1687.

\vspace{1em}

At the very beginning of the \textit{Principia}, Newton proposes three laws that would govern the motions of all objects:

\begin{itemize}
    \item \gls{Newton's first law}: Every object will continue to be in a state of rest or move at a constant speed in a straight line unless it is compelled to change by an outside force.
    \item \gls{Newton's second law}: The change of motion of a body is proportional to and in the direction of the force acting on it.
    \item \gls{Newton's third law}: For every action there is an equal and opposite reaction (or: the mutual actions of two bodies upon each other are always equal and act in opposite directions).
\end{itemize}

In the original Latin, the three laws contain only 59 words, but those few words set the stage for modern science. Let us examine them more carefully.

\subsubsection*{Interpretation of Newton's Laws}

Newton’s first law is a restatement of one of Galileo’s discoveries, called the conservation of momentum. The law states that in the absence of any outside influence, there is a measure of a body’s motion, called its \gls{momentum}, that remains unchanged. You may have heard the term momentum used in everyday expressions, such as ``This bill in Congress has a lot of momentum; it’s going to be hard to stop.''

\vspace{1em}

Newton’s first law is sometimes called the \textit{law of inertia}, where inertia is the tendency of objects (and legislatures) to keep doing what they are already doing. In other words, a stationary object stays put, and a moving object keeps moving unless some force intervenes.

\vspace{1em}

Let’s define the precise meaning of momentum---it depends on three factors: (1) speed---how fast a body moves (zero if it is stationary), (2) the direction of its motion, and (3) its mass---a measure of the amount of matter in a body, which we will discuss later. Scientists use the term \gls{velocity} to describe the speed and direction of motion. For example, 20 kilometers per hour due south is velocity, whereas 20 kilometers per hour just by itself is speed. Momentum then can be defined as an object’s mass times its velocity.

\vspace{1em}

It’s not so easy to see this rule in action in the everyday world because of the many forces acting on a body at any one time. One important force is friction, which generally slows things down. If you roll a ball along the sidewalk, it eventually comes to a stop because the sidewalk exerts a rubbing force on the ball. But in the space between the stars, where there is so little matter that friction is insignificant, objects can in fact continue to move (to coast) indefinitely.

\vspace{1em}

The momentum of a body can change only under the action of an outside influence. Newton’s second law expresses force in terms of its ability to change momentum with time. A force (a push or a pull) has both size and direction. When a force is applied to a body, the momentum changes in the direction of the applied force. This means that a force is required to change either the speed or the direction of a body, or both---that is, to start it moving, to speed it up, to slow it down, to stop it, or to change its direction.

\vspace{1em}

As you learned in ``Observing the Sky: The Birth of Astronomy,'' the rate of change in an object’s velocity is called acceleration. Newton showed that the acceleration of a body was proportional to the force being applied to it. Suppose that after a long period of reading, you push an astronomy book away from you on a long, smooth table. (We use a smooth table so we can ignore friction.) If you push the book steadily, it will continue to speed up as long as you are pushing it. The harder you push the book, the larger its acceleration will be. How much a force will accelerate an object is also determined by the object’s mass. If you kept pushing a pen with the same force with which you pushed the textbook, the pen---having less mass---would be accelerated to a greater speed.

\vspace{1em}

Newton’s third law is perhaps the most profound of the rules he discovered. Basically, it is a generalization of the first law, but it also gives us a way to define mass. If we consider a system of two or more objects isolated from outside influences, Newton’s first law says that the total momentum of the objects should remain constant. Therefore, any change of momentum within the system must be balanced by another change that is equal and opposite so that the momentum of the entire system is not changed.

\vspace{1em}

This means that forces in nature do not occur alone: we find that in each situation there is always a pair of forces that are equal to and opposite each other. If a force is exerted on an object, it must be exerted by something else, and the object will exert an equal and opposite force back on that something. We can look at a simple example to demonstrate this.

\vspace{1em}

Suppose that a daredevil astronomy student---and avid skateboarder---wants to jump from his second-story dorm window onto his board below (we don’t recommend trying this!). The force pulling him down after jumping (as we will see in the next section) is the force of gravity between him and Earth. Both he and Earth must experience the same total change of momentum because of the influence of these mutual forces. So, both the student and Earth are accelerated by each other’s pull. However, the student does much more of the moving. Because Earth has enormously greater mass, it can experience the same change of momentum by accelerating only a very small amount. Things fall toward Earth all the time, but the acceleration of our planet as a result is far too small to be measured.

\vspace{1em}

A more obvious example of the mutual nature of forces between objects is familiar to all who have batted a baseball. The recoil you feel as you swing your bat shows that the ball exerts a force on it during the impact, just as the bat does on the ball. Similarly, when a rifle you are bracing on your shoulder is discharged, the force pushing the bullet out of the muzzle is equal to the force pushing backward upon the gun and your shoulder.

\vspace{1em}

This is the principle behind jet engines and rockets: the force that discharges the exhaust gases from the rear of the rocket is accompanied by the force that pushes the rocket forward. The exhaust gases need not push against air or Earth; a rocket actually operates best in a vacuum (Figure ?.??).

\begin{gradient}{LINK TO LEARNING}
    For more about Isaac Newton’s life and work, check out this \href{https://openstax.org/l/30IsaacNewTime}{timeline page} with snapshots from his career, produced by the British Broadcasting Corporation (BBC).
\end{gradient}

\subsubsection*{Mass, Volume, and Density}

Before we go on to discuss Newton’s other work, we want to take a brief look at some terms that will be important to sort out clearly. We begin with \gls{mass}, which is a measure of the amount of material within an object.

\vspace{1em}

The \textit{volume} of an object is the measure of the physical space it occupies. Volume is measured in cubic units, such as cubic centimeters or liters. The volume is the ``size'' of an object. A penny and an inflated balloon may both have the same mass, but they have very different volumes. The reason is that they also have very different densities, which is a measure of how much mass there is per unit volume. Specifically, \gls{density} is the mass divided by the volume. Note that in everyday language we often use ``heavy'' and ``light'' as indications of density (rather than weight) as, for instance, when we say that iron is heavy or that whipped cream is light.

\vspace{1em}

The units of density that will be used in this book are grams per cubic centimeter (\SI{}{g/cm^3}). If a block of some material has a mass of 300 grams and a volume of \SI{100}{cm^3}, its density is \SI{3}{g/cm^3}. Familiar materials span a considerable range in density, from artificial materials such as plastic insulating foam (less than \SI{0.1}{g/cm^3}) to gold (\SI{19.3}{g/cm^3}). Table ?.? gives the densities of some familiar materials. In the astronomical universe, much more remarkable densities can be found, all the way from a comet’s tail ($10^{-16}\,\SI{}{g/cm^3}$) to a collapsed ``star corpse'' called a neutron star ($10^{15}\,\SI{}{g/cm^3}$).

\begin{center}
    \begin{tabular}{|c|c|}
    \hline
        \textbf{Material} & \textbf{Density \SI{}{g/cm^3}} \\
        \hline
        Gold & 19.3\\
        Lead & 11.3\\
        Iron & 7.9\\
        Earth (bulk) & 5.5\\
        Rock (typical) & 2.5\\
        Water & 1\\
        Wood (typical) & 0.8\\
        Insulating foam & 0.1\\
        Silica gel & 0.02\\
        \hline
    \end{tabular}
    \captionsetup{type=table,margin=1in,font=scriptsize}
    \captionof{table}{Densities of common materials.}
\end{center}

To sum up, mass is \textit{how much}, volume is \textit{how big}, and density is \textit{how tightly packed}.

\begin{gradient}{LINK TO LEARNING}
    You can play with a \href{https://openstax.org/l/30phetsimdenmas}{simple animation} demonstrating the relationship between the concepts of density, mass, and volume, and find out why objects like wood float in water.
\end{gradient}

\subsubsection*{Angular Momentum}

A concept that is a bit more complex, but important for understanding many astronomical objects, is \gls{angular momentum}, which is a measure of the rotation of a body as it revolves around some fixed point (an example is a planet orbiting the Sun). The angular momentum of an object is defined as the product of its mass, its velocity, and its distance from the fixed point around which it revolves.

\vspace{1em}

If these three quantities remain constant---that is, if the motion of a particular object takes place at a constant velocity at a fixed distance from the spin center---then the angular momentum is also a constant. Kepler’s second law is a consequence of the \textit{conservation of angular momentum}. As a planet approaches the Sun on its elliptical orbit and the distance to the spin center decreases, the planet speeds up to conserve the angular momentum. Similarly, when the planet is farther from the Sun, it moves more slowly.

\vspace{1em}

The conservation of angular momentum is illustrated by figure skaters, who bring their arms and legs in to spin more rapidly, and extend their arms and legs to slow down (Figure ?.??). You can duplicate this yourself on a well-oiled swivel stool by starting yourself spinning slowly with your arms extended and then pulling your arms in. Another example of the conservation of angular momentum is a shrinking cloud of dust or a star collapsing on itself (both are situations that you will learn about as you read on). As material moves to a lesser distance from the spin center, the speed of the material increases to conserve angular momentum.

\vspace{1em} %Skip Figure








\end{document}