\documentclass[main.tex]{subfiles}

\begin{document}

\section{Observing The Sky}

Much to your surprise, a member of the Flat Earth Society moves in next door. He believes that Earth is flat and all the NASA images of a spherical Earth are either faked or simply show the round (but flat) disk of Earth from above. How could you prove to your new neighbor that Earth really is a sphere? (When you've thought about this on your own, you can check later in the chapter for some suggested answers.)

\vspace{1em}

Today, few people really spend much time looking at the night sky. In ancient days, before electric lights robbed so many people of the beauty of the sky, the stars and planets were an important aspect of everyone's daily life. All the records that we have---on paper and in stone---show that ancient civilizations around the world noticed, worshipped, and tried to understand the lights in the sky and fit them into their own view of the world. These ancient observers found both majestic regularity and never-ending surprise in the motions of the heavens. Through their careful study of the planets, the Greeks and later the Romans laid the foundation of the science of astronomy.

\subsection{The Sky Above}

Our senses suggest to us that Earth is the center of the universe---the hub around which the heavens turn. This \gls{geocentric} (Earth-centered) view was what almost everyone believed until the European Renaissance. After all, it is simple, logical, and seemingly self-evident. Furthermore, the geocentric perspective reinforced those philosophical and religious systems that taught the unique role of human beings as the central focus of the cosmos. However, the geocentric view happens to be wrong. One of the great themes of our intellectual history is the overthrow of the geocentric perspective. Let us, therefore, take a look at the steps by which we reevaluated the place of our world in the cosmic order.

\subsubsection*{The Celestial Sphere}

If you go on a camping trip or live far from city lights, your view of the sky on a clear night is pretty much identical to that seen by people all over the world before the invention of the telescope. Gazing up, you get the impression that the sky is a great hollow dome with you at the center (Figure ?.?), and all the stars are an equal distance from you on the surface of the dome. The top of that dome, the point directly above your head, is called the \gls{zenith}, and where the dome meets Earth is called the \gls{horizon}. From the sea or a flat prairie, it is easy to see the horizon as a circle around you, but from most places where people live today, the horizon is at least partially hidden by mountains, trees, buildings, or smog.


%%% INSERT FIGURE HERE %%%%%
\vspace{1em}

If you lie back in an open field and observe the night sky for hours, as ancient shepherds and travelers regularly did, you will see stars rising on the eastern horizon (just as the Sun and Moon do), moving across the dome of the sky in the course of the night, and setting on the western horizon. Watching the sky turn like this night after night, you might eventually get the idea that the dome of the sky is really part of a great sphere that is turning around you, bringing different stars into view as it turns. The early Greeks regarded the sky as just such a celestial sphere (Figure ?.?). Some thought of it as an actual sphere of transparent crystalline material, with the stars embedded in it like tiny jewels.

%%%% INSERT FIGURE HERE %%%%
\vspace{1em}

Today, we know that it is not the celestial sphere that turns as night and day proceed, but rather the planet on which we live. We can put an imaginary stick through Earth's North and South Poles, representing our planet's axis. It is because Earth turns on this axis every 24 hours that we see the Sun, Moon, and stars rise and set with clockwork regularity. Today, we know that these celestial objects are not really on a dome, but at greatly varying distances from us in space. Nevertheless, it is sometimes still convenient to talk about the celestial dome or sphere to help us keep track of objects in the sky. There is even a special theater, called a planetarium, in which we project a simulation of the stars and planets onto a white dome.

\vspace{1em}

As the celestial sphere rotates, the objects on it maintain their positions with respect to one another. A grouping of stars such as the Big Dipper has the same shape during the course of the night, although it turns with the sky. During a single night, even objects we know to have significant motions of their own, such as the nearby planets, seem fixed relative to the stars. Only meteors---brief ``shooting stars'' that flash into view for just a few seconds---move appreciably with respect to other objects on the celestial sphere. (This is because they are not stars at all. Rather, they are small pieces of cosmic dust, burning up as they hit Earth's atmosphere.) We can use the fact that the entire celestial sphere seems to turn together to help us set up systems for keeping track of what things are visible in the sky and where they happen to be at a given time.

\subsubsection*{Celestial Poles and Celestial Equator}

To help orient us in the turning sky, astronomers use a system that extends Earth's axis points into the sky. Imagine a line going through Earth, connecting the North and South Poles. This is Earth's axis, and Earth rotates about this line. If we extend this imaginary line outward from Earth, the points where this line intersects the celestial sphere are called the north celestial pole and the south celestial pole. As Earth rotates about its axis, the sky appears to turn in the opposite direction around those \gls{celestial pole}s (Figure ?.?). We also (in our imagination) throw Earth's equator onto the sky and call this the \gls{celestial equator}. It lies halfway between the celestial poles, just as Earth's equator lies halfway between our planet's poles.

%%% FIGURE

\vspace{1em}

Now let's imagine how riding on different parts of our spinning Earth affects our view of the sky. The apparent motion of the celestial sphere depends on your latitude (position north or south of the equator). First of all, notice that Earth's axis is pointing at the celestial poles, so these two points in the sky do not appear to turn.

\vspace{1em}

If you stood at the North Pole of Earth, for example, you would see the north celestial pole overhead, at your zenith. The celestial equator, \ang{90} from the celestial poles, would lie along your horizon. As you watched the stars during the course of the night, they would all circle around the celestial pole, with none rising or setting. Only that half of the sky north of the celestial equator is ever visible to an observer at the North Pole. Similarly, an observer at the South Pole would see only the southern half of the sky.

\vspace{1em}

If you were at Earth's equator, on the other hand, you see the celestial equator (which, after all, is just an ``extension'' of Earth's equator) pass overhead through your zenith. The celestial poles, being \ang{90} from the celestial equator, must then be at the north and south points on your horizon. As the sky turns, all stars rise and set; they move straight up from the east side of the horizon and set straight down on the west side. During a 24-hour period, all stars are above the horizon exactly half the time. (Of course, during some of those hours, the Sun is too bright for us to see them.)

\vspace{1em}

What would an observer in the latitudes of the United States or Europe see? Remember, we are neither at Earth's pole nor at the equator, but in between them. For those in the continental United States and Europe, the north celestial pole is neither overhead nor on the horizon, but in between. It appears above the northern horizon at an angular height, or altitude, equal to the observer's latitude. In Cypress, Texas, for example, where the latitude is about \ang{30} N, the north celestial pole is \ang{30} above the northern horizon.

\vspace{1em}

For an observer at \ang{30} N latitude, the south celestial pole is \ang{30} below the southern horizon and, thus, never visible. As Earth turns, the whole sky seems to pivot about the north celestial pole. For this observer, stars within \ang{30} of the North Pole can never set. They are always above the horizon, day and night. This part of the sky is called the north \gls{circumpolar zone}. For observers in the continental United States, the Big Dipper, Little Dipper, and Cassiopeia are examples of star groups in the north circumpolar zone. On the other hand, stars within \ang{30} of the south celestial pole never rise. That part of the sky is the south circumpolar zone. To most U.S.~observers, the Southern Cross is in that zone. (Don't worry if you are not familiar with the star groups just mentioned; we will introduce them more formally later on.)

\vspace{1em}

At this particular time in Earth's history, there happens to be a star very close to the north celestial pole. It is called Polaris, the pole star, and has the distinction of being the star that moves the least amount as the northern sky turns each day. Because it moved so little while the other stars moved much more, it played a special role in the mythology of several Native American tribes, for example (some called it the ``fastener of the sky'').

\subsubsection*{Rising and Setting of the Sun}

We described the movement of stars in the night sky, but what about during the daytime? The stars continue to circle during the day, but the brilliance of the Sun makes them difficult to see. (The Moon can often be seen in the daylight, however.) On any given day, we can think of the Sun as being located at some position on the hypothetical celestial sphere. When the Sun rises---that is, when the rotation of Earth carries the Sun above the horizon---sunlight is scattered by the molecules of our atmosphere, filling our sky with light and hiding the stars above the horizon.

\vspace{1em}

For thousands of years, astronomers have been aware that the Sun does more than just rise and set. It changes position gradually on the celestial sphere, moving each day about \ang{1} to the east relative to the stars. Very reasonably, the ancients thought this meant the Sun was slowly moving around Earth, taking a period of time we call 1 \gls{year} to make a full circle. Today, of course, we know it is Earth that is going around the Sun, but the effect is the same: the Sun's position in our sky changes day to day. We have a similar experience when we walk around a campfire at night; we see the flames appear in front of each person seated about the fire in turn.

\vspace{1em}

The path the Sun appears to take around the celestial sphere each year is called the \gls{ecliptic} (Figure ?.??). Because of its motion on the ecliptic, the Sun rises about 4 minutes later each day with respect to the stars. Earth must make just a bit more than one complete rotation (with respect to the stars) to bring the Sun up again.

%%%% Figure
\vspace{1em}

As the months go by and we look at the Sun from different places in our orbit, we see it projected against different places in our orbit, and thus against different stars in the background (Figure ?.?? and Table ?.?)---or we would, at least, if we could see the stars in the daytime. In practice, we must deduce which stars lie behind and beyond the Sun by observing the stars visible in the opposite direction at night. After a year, when Earth has completed one trip around the Sun, the Sun will appear to have completed one circuit of the sky along the ecliptic.

%%%% Figure

\vspace{1em}

\begin{center}
\textbf{Constellations on the Ecliptic}
    \begin{tabular}{c|c}
        \textbf{Constellation on the Ecliptic} & \textbf{Dates When the Sun Crosses It}\\
        \hline
        Capricornus	& January 21--February 16\\
        Aquarius & February 16--March 11\\
        Pisces & March 11--April 18\\
        Aries & April 18--May 13\\
        Taurus & May 13--June 22\\
        Gemini & June 22--July 21\\
        Cancer & July 21--August 10\\
        Leo & August 10--September 16\\
        Virgo & September 16--October 31\\
        Libra & October 31--November 23\\
        Scorpius & November 23--November 29\\
        Ophiuchus & November 29--December 18\\
        Sagittarius & December 18--January 21\\
        \hline
    \end{tabular}
\end{center}

The ecliptic does not lie along the celestial equator but is inclined to it at an angle of about \ang{23.5}. In other words, the Sun's annual path in the sky is not linked with Earth's equator. This is because our planet's axis of rotation is tilted by about \ang{23.5} from a vertical line sticking out of the plane of the ecliptic (Figure ?.??). Being tilted from ``straight up'' is not at all unusual among celestial bodies; Uranus and Pluto are actually tilted so much that they orbit the Sun ``on their side.''

%%%%% Figure

\vspace{1em}

The inclination of the ecliptic is the reason the Sun moves north and south in the sky as the seasons change. In a later unit, we discuss the progression of the seasons in more detail.

\subsubsection*{Fixed and Wandering Stars}

The Sun is not the only object that moves among the fixed stars. The Moon and each of the planets that are visible to the unaided eye---Mercury, Venus, Mars, Jupiter, Saturn, and Uranus (although just barely)---also change their positions slowly from day to day. During a single day, the Moon and planets all rise and set as Earth turns, just as the Sun and stars do. But like the Sun, they have independent motions among the stars, superimposed on the daily rotation of the celestial sphere. Noticing these motions, the Greeks of 2000 years ago distinguished between what they called the fixed stars---those that maintain fixed patterns among themselves through many generations---and the wandering stars, or \gls{planets}. The word ``planet,'' in fact, means ``wanderer'' in ancient Greek.

\vspace{1em}

Today, we do not regard the Sun and Moon as planets, but the ancients applied the term to all seven of the moving objects in the sky. Much of ancient astronomy was devoted to observing and predicting the motions of these celestial wanderers. They even dedicated a unit of time, the week, to the seven objects that move on their own; that's why there are 7 days in a week. The Moon, being Earth's nearest celestial neighbor, has the fastest apparent motion; it completes a trip around the sky in about 1 month (or \textit{moonth}). To do this, the Moon moves about \ang{12}, or 24 times its own apparent width on the sky, each day.

\vspace{1em}

The individual paths of the Moon and planets in the sky all lie close to the ecliptic, although not exactly on it. This is because the paths of the planets about the Sun, and of the Moon about Earth, are all in nearly the same plane, as if they were circles on a huge sheet of paper. The planets, the Sun, and the Moon are thus always found in the sky within a narrow 18-degree-wide belt, centered on the ecliptic, called the \gls{zodiac} (Figure ?.??). (The root of the term ``zodiac'' is the same as that of the word ``zoo'' and means a collection of animals; many of the patterns of stars within the zodiac belt reminded the ancients of animals, such as a fish or a goat.)

\vspace{1em}

How the planets appear to move in the sky as the months pass is a combination of their actual motions plus the motion of Earth about the Sun; consequently, their paths are somewhat complex. As we will see, this complexity has fascinated and challenged astronomers for centuries.

\subsubsection*{Constellations}

The backdrop for the motions of the ``wanderers'' in the sky is the canopy of stars. If there were no clouds in the sky and we were on a flat plain with nothing to obstruct our view, we could see about 3000 stars with the unaided eye. To find their way around such a multitude, the ancients found groupings of stars that made some familiar geometric pattern or (more rarely) resembled something they knew. Each civilization found its own patterns in the stars, much like a modern Rorschach test in which you are asked to discern patterns or pictures in a set of inkblots. The ancient Chinese, Egyptians, and Greeks, among others, found their own groupings---or constellations---of stars. These were helpful in navigating among the stars and in passing their star lore on to their children.

\vspace{1em}

You may be familiar with some of the old star patterns we still use today, such as the Big Dipper, Little Dipper, and Orion the hunter, with his distinctive belt of three stars (Figure ?.???). However, many of the stars we see are not part of a distinctive star pattern at all, and a telescope reveals millions of stars too faint for the eye to see. Therefore, during the early decades of the 20th century, astronomers from many countries decided to establish a more formal system for organizing the sky.

%%%% Figure

\vspace{1em}

Today, we use the term constellation to mean one of 88 sectors into which we divide the sky, much as the United States is divided into 50 states. The modern boundaries between the constellations are imaginary lines in the sky running north-south and east-west, so that each point in the sky falls in a specific constellation, although, like the states, not all constellations are the same size. All the constellations are listed in [Appendix L???]. Whenever possible, we have named each modern constellation after the Latin translations of one of the ancient Greek star patterns that lies within it. Thus, the modern constellation of Orion is a kind of box on the sky, which includes, among many other objects, the stars that made up the ancient picture of the hunter. Some people use the term \textit{asterism} to denote an especially noticeable star pattern within a constellation (or sometimes spanning parts of several constellations). For example, the Big Dipper is an asterism within the constellation of Ursa Major, the Big Bear.

\vspace{1em}

Students are sometimes puzzled because the constellations seldom resemble the people or animals for which they were named. In all likelihood, the Greeks themselves did not name groupings of stars because they looked like actual people or subjects (any more than the outline of Washington state resembles George Washington). Rather, they named sections of the sky in honor of the characters in their mythology and then fit the star configurations to the animals and people as best they could.

\begin{mdframed}[backgroundcolor=black!10]
    \textbf{Link to Learning}
    
    \vspace{1ex}

    This \href{https://openstax.org/l/30heavensabove}{website about objects in the sky} allows users to construct a detailed sky map showing the location and information about the Sun, Moon, planets, stars, constellations, and even satellites orbiting Earth. Begin by setting your observing location using the option in the menu in the upper right corner of the screen. An excellent website called \href{https://openstax.org/l/30constel}{Figures in the Sky} shows the constellation figures imagined by cultures around the world.
\end{mdframed}

\subsection{Ancient Astronomy}

Let us now look briefly back into history. Much of modern Western civilization is derived in one way or another from the ideas of the ancient Greeks and Romans, and this is true in astronomy as well. However, many other ancient cultures also developed sophisticated systems for observing and interpreting the sky.

\subsubsection*{Astronomy Around the World}

Ancient Babylonian, Assyrian, and Egyptian astronomers knew the approximate length of the year. The Egyptians of 3000 years ago, for example, adopted a calendar based on a 365-day year. They kept careful track of the rising time of the bright star Sirius in the predawn sky, which has a yearly cycle that corresponded with the flooding of the Nile River. The Chinese also had a working calendar; they determined the length of the year at about the same time as the Egyptians. The Chinese also recorded comets, bright meteors, and dark spots on the Sun. (Many types of astronomical objects were introduced in Science and the Universe: A Brief Tour. If you are not familiar with terms like comets and meteors, you may want to review that chapter.) Later, Chinese astronomers kept careful records of ``guest stars''---those that are normally too faint to see but suddenly flare up to become visible to the unaided eye for a few weeks or months. We still use some of these records in studying stars that exploded a long time ago.

\vspace{1em}

The Mayan culture in Mexico and Central America developed a sophisticated calendar based on the planet Venus, and they made astronomical observations from sites dedicated to this purpose a thousand years ago. The Polynesians learned to navigate by the stars over hundreds of kilometers of open ocean---a skill that enabled them to colonize new islands far away from where they began.

\vspace{1em}

In Britain, before the widespread use of writing, ancient people used stones to keep track of the motions of the Sun and Moon. We still find some of the great stone circles they built for this purpose, dating from as far back as 2800 BCE. The best known of these is Stonehenge, which is discussed in a later unit\footnote{For resources providing more information about the astronomy of diverse cultures around the world, please see \href{http://bit.ly/astrocultures}{http://bit.ly/astrocultures}}.

\subsubsection*{Early Greek and Roman Cosmology}

Our concept of the cosmos---its basic structure and origin---is called \gls{cosmology}, a word with Greek roots. Before the invention of telescopes, humans had to depend on the simple evidence of their senses for a picture of the universe. The ancients developed cosmologies that combined their direct view of the heavens with a rich variety of philosophical and religious symbolism.

\vspace{1em}

At least 2000 years before Columbus, educated people in the eastern Mediterranean region knew Earth was round. Belief in a spherical Earth may have stemmed from the time of Pythagoras, a philosopher and mathematician who lived 2500 years ago. He believed circles and spheres to be ``perfect forms'' and suggested that Earth should therefore be a sphere. As evidence that the gods liked spheres, the Greeks cited the fact that the Moon is a sphere, using evidence we describe later.

\vspace{1em}

The writings of Aristotle (384--322 BCE), the tutor of Alexander the Great, summarize many of the ideas of his day. They describe how the progression of the Moon's phases---its apparent changing shape---results from our seeing different portions of the Moon's sunlit hemisphere as the month goes by (see Earth, Moon, and Sky). Aristotle also knew that the Sun has to be farther away from Earth than is the Moon because occasionally the Moon passed exactly between Earth and the Sun and hid the Sun temporarily from view. We call this a \textit{solar eclipse}.

\vspace{1em}

Aristotle cited convincing arguments that Earth must be round. First is the fact that as the Moon enters or emerges from Earth's shadow during an eclipse of the Moon, the shape of the shadow seen on the Moon is always round (Figure ?.??). Only a spherical object always produces a round shadow. If Earth were a disk, for example, there would be some occasions when the sunlight would strike it edge-on and its shadow on the Moon would be a line.

%%% Figure

\vspace{1em}

As a second argument, Aristotle explained that travelers who go south a significant distance are able to observe stars that are not visible farther north. And the height of the North Star---the star nearest the north celestial pole---decreases as a traveler moves south. On a flat Earth, everyone would see the same stars overhead. The only possible explanation is that the traveler must have moved over a curved surface on Earth, showing stars from a different angle. (See the How Do We Know Earth Is Round? feature for more ideas on proving Earth is round.)

\vspace{1em}

One Greek thinker, Aristarchus of Samos (310--230 BCE), even suggested that Earth was moving around the Sun, but Aristotle and most of the ancient Greek scholars rejected this idea. One of the reasons for their conclusion was the thought that if Earth moved about the Sun, they would be observing the stars from different places along Earth's orbit. As Earth moved along, nearby stars should shift their positions in the sky relative to more distant stars. In a similar way, we see foreground objects appear to move against a more distant background whenever we are in motion. When we ride on a train, the trees in the foreground appear to shift their position relative to distant hills as the train rolls by. Unconsciously, we use this phenomenon all of the time to estimate distances around us.

\vspace{1em}

The apparent shift in the direction of an object as a result of the motion of the observer is called \gls{parallax}. We call the shift in the apparent direction of a star due to Earth's orbital motion stellar parallax. The Greeks made dedicated efforts to observe stellar parallax, even enlisting the aid of Greek soldiers with the clearest vision, but to no avail. The brighter (and presumably nearer) stars just did not seem to shift as the Greeks observed them in the spring and then again in the fall (when Earth is on the opposite side of the Sun).

\vspace{1em}

This meant either that Earth was not moving or that the stars had to be so tremendously far away that the parallax shift was immeasurably small. A cosmos of such enormous extent required a leap of imagination that most ancient philosophers were not prepared to make, so they retreated to the safety of the Earth-centered view, which would dominate Western thinking for nearly two millennia.

\begin{mdframed}[backgroundcolor=black!10]
    \textbf{Astronomy Basics: How Do We Know Earth Is Round?}
    
    \vspace{1ex}
    
    In addition to the two ways (from Aristotle's writings) discussed in this chapter, you might also reason as follows:
    
    \vspace{1ex}

    \begin{enumerate}
    \setlength\itemsep{0.1ex}
        \item Let's watch a ship leave its port and sail into the distance on a clear day. On a flat Earth, we would just see the ship get smaller and smaller as it sails away. But this isn't what we actually observe. Instead, ships sink below the horizon, with the hull disappearing first and the mast remaining visible for a while longer. Eventually, only the top of the mast can be seen as the ship sails around the curvature of Earth. Finally, the ship disappears under the horizon.
        \item  The International Space Station circles Earth once every 90 minutes or so. Photographs taken from the shuttle and other satellites show that Earth is round from every perspective.
        \item Suppose you made a friend in each time zone of Earth. You call all of them at the same hour and ask, ``Where is the Sun?'' On a flat Earth, each caller would give you roughly the same answer. But on a round Earth you would find that, for some friends, the Sun would be high in the sky whereas for others it would be rising, setting, or completely out of sight (and this last group of friends would be upset with you for waking them up).
    \end{enumerate}
\end{mdframed}

\subsubsection*{Measurement of Earth by Eratosthenes}

The Greeks not only knew Earth was round, but also they were able to measure its size. The first fairly accurate determination of Earth's diameter was made in about 200 BCE by Eratosthenes (276--194 BCE), a Greek living in Alexandria, Egypt. His method was a geometric one, based on observations of the Sun.

\vspace{1em}

The Sun is so distant from us that all the light rays that strike our planet approach us along essentially parallel lines. To see why, look at Figure ?.??. Take a source of light near Earth--say, at position A. Its rays strike different parts of Earth along diverging paths. From a light source at B, or at C (which is still farther away), the angle between rays that strike opposite parts of Earth is smaller. The more distant the source, the smaller the angle between the rays. For a source infinitely distant, the rays travel along parallel lines.

%%% Figure

\vspace{1em}

Of course, the Sun is not infinitely far away, but given its distance of 150 million kilometers, light rays striking Earth from a point on the Sun diverge from one another by an angle far too small to be observed with the unaided eye. As a consequence, if people all over Earth who could see the Sun were to point at it, their fingers would, essentially, all be parallel to one another. (The same is also true for the planets and stars---an idea we will use in our discussion of how telescopes work.)

\vspace{1em}

Eratosthenes was told that on the first day of summer at Syene, Egypt (near modern Aswan), sunlight struck the bottom of a vertical well at noon. This indicated that the Sun was directly over the well---meaning that Syene was on a direct line from the center of Earth to the Sun. At the corresponding time and date in Alexandria, Eratosthenes observed the shadow a column made and saw that the Sun was not directly overhead, but was slightly south of the zenith, so that its rays made an angle with the vertical equal to about 1/50 of a circle (\ang{7}). Because the Sun's rays striking the two cities are parallel to one another, why would the two rays not make the same angle with Earth's surface? Eratosthenes reasoned that the curvature of the round Earth meant that ``straight up'' was not the same in the two cities. And the measurement of the angle in Alexandria, he realized, allowed him to figure out the size of Earth. Alexandria, he saw, must be 1/50 of Earth's circumference north of Syene (Figure ?.??). Alexandria had been measured to be 5000 stadia north of Syene. (The stadium was a Greek unit of length, derived from the length of the racetrack in a stadium.) Eratosthenes thus found that Earth's circumference must be $50 \times 5000$, or \num{250000} stadia.

%%% Figure 

\vspace{1em}

It is not possible to evaluate precisely the accuracy of Eratosthenes solution because there is doubt about which of the various kinds of Greek stadia he used as his unit of distance. If it was the common Olympic stadium, his result is about 20\% too large. According to another interpretation, he used a stadium equal to about 1/6 kilometer, in which case his figure was within 1\% of the correct value of \num{40000} kilometers. Even if his measurement was not exact, his success at measuring the size of our planet by using only shadows, sunlight, and the power of human thought was one of the greatest intellectual achievements in history.

\subsubsection*{Hipparchus and Precession}

Perhaps the greatest astronomer of antiquity was Hipparchus, born in Nicaea in what is present-day Turkey. He erected an observatory on the island of Rhodes around 150 BCE, when the Roman Republic was expanding its influence throughout the Mediterranean region. There he measured, as accurately as possible, the positions of objects in the sky, compiling a pioneering star catalog with about 850 entries. He designated celestial coordinates for each star, specifying its position in the sky, just as we specify the position of a point on Earth by giving its latitude and longitude.

\vspace{1em}

He also divided the stars into \gls{apparent magnitude}s according to their apparent brightness. He called the brightest ones ``stars of the first magnitude''; the next brightest group, ``stars of the second magnitude''; and so forth. This rather arbitrary system, in modified form, still remains in use today (although it is less and less useful for professional astronomers).

\vspace{1em}

By observing the stars and comparing his data with older observations, Hipparchus made one of his most remarkable discoveries: the position in the sky of the north celestial pole had altered over the previous century and a half. Hipparchus deduced correctly that this had happened not only during the period covered by his observations, but was in fact happening all the time: the direction around which the sky appears to rotate changes slowly but continuously. Recall from the section on celestial poles and the celestial equator that the north celestial pole is just the projection of Earth's North Pole into the sky. If the north celestial pole is wobbling around, then Earth itself must be doing the wobbling. Today, we understand that the direction in which Earth's axis points does indeed change slowly but regularly---a motion we call precession. If you have ever watched a spinning top wobble, you observed a similar kind of motion. The top's axis describes a path in the shape of a cone, as Earth's gravity tries to topple it (Figure ?.??).

\vspace{1em}

%% Figure

Because our planet is not an exact sphere, but bulges a bit at the equator, the pulls of the Sun and Moon cause it to wobble like a top. It takes about \num{26000} years for Earth's axis to complete one circle of precession. As a result of this motion, the point where our axis points in the sky changes as time goes on. While Polaris is the star closest to the north celestial pole today (it will reach its closest point around the year 2100), the star Vega in the constellation of Lyra will be the North Star in \num{14000} years.

\subsubsection*{Ptolemy's Model of the Solar System}

The last great astronomer of the Roman era was Claudius Ptolemy (or Ptolemaeus), who flourished in Alexandria in about the year 140. He wrote a mammoth compilation of astronomical knowledge, which today is called by its Arabic name, \textit{Almagest} (meaning ``The Greatest''). \textit{Almages}t does not deal exclusively with Ptolemy's own work; it includes a discussion of the astronomical achievements of the past, principally those of Hipparchus. Today, it is our main source of information about the work of Hipparchus and other Greek astronomers.

\vspace{1em}

Ptolemy's most important contribution was a geometric representation of the solar system that predicted the positions of the planets for any desired date and time. Hipparchus, not having enough data on hand to solve the problem himself, had instead amassed observational material for posterity to use. Ptolemy supplemented this material with new observations of his own and produced a cosmological model that endured more than a thousand years, until the time of Copernicus.

\vspace{1em}

The complicating factor in explaining the motions of the planets is that their apparent wandering in the sky results from the combination of their own motions with Earth's orbital revolution. As we watch the planets from our vantage point on the moving Earth, it is a little like watching a car race while you are competing in it. Sometimes opponents' cars pass you, but at other times you pass them, making them appear to move backward for a while with respect to you.

\vspace{1em}

Figure ?.?? shows the motion of Earth and a planet farther from the Sun---in this case, Mars. Earth travels around the Sun in the same direction as the other planet and in nearly the same plane, but its orbital speed is faster. As a result, it overtakes the planet periodically, like a faster race car on the inside track. The figure shows where we see the planet in the sky at different times. The path of the planet among the stars is illustrated in the star field on the right side of the figure.

%%%% Figure

\vspace{1em}

\begin{mdframed}[backgroundcolor=black!10]
    \textbf{Link to Learning}

    \vspace{1ex}

    The \href{https://openstax.org/l/30planetconfig}{planetary configurations simulator from Foothill AstroSims} allows you to see the usual prograde and occasional retrograde motion of other planets. You can switch back and forth between viewing motion from Earth and Mars (as well as other planets).
\end{mdframed}

\vspace{1em}

Normally, planets move eastward in the sky over the weeks and months as they orbit the Sun, but from positions B to D in Figure ?.??, as Earth passes the planets in our example, it appears to drift backward, moving west in the sky. Even though it is actually moving to the east, the faster-moving Earth has overtaken it and seems, from our perspective, to be leaving it behind. As Earth rounds its orbit toward position E, the planet again takes up its apparent eastward motion in the sky. The temporary apparent westward motion of a planet as Earth swings between it and the Sun is called \gls{retrograde motion}. Such backward motion is much easier for us to understand today, now that we know Earth is one of the moving planets and not the unmoving center of all creation. But Ptolemy was faced with the far more complex problem of explaining such motion while assuming a stationary Earth.

\vspace{1em}

Furthermore, because the Greeks believed that celestial motions had to be circles, Ptolemy had to construct his model using circles alone. To do it, he needed dozens of circles, some moving around other circles, in a complex structure that makes a modern viewer dizzy. But we must not let our modern judgment cloud our admiration for Ptolemy's achievement. In his day, a complex universe centered on Earth was perfectly reasonable and, in its own way, quite beautiful. However, as Alfonso X, the King of Castile, was reported to have said after having the Ptolemaic system of planet motions explained to him, ``If the Lord Almighty had consulted me before embarking upon Creation, I should have recommended something simpler.''

\vspace{1em}

Ptolemy solved the problem of explaining the observed motions of planets by having each planet revolve in a small orbit called an \gls{epicycle}. The center of the epicycle then revolved about Earth on a circle called a deferent (Figure ?.??). When the planet is at position x in Figure ?.?? on the epicycle orbit, it is moving in the same direction as the center of the epicycle; from Earth, the planet appears to be moving eastward. When the planet is at y, however, its motion is in the direction opposite to the motion of the epicycle's center around Earth. By choosing the right combination of speeds and distances, Ptolemy succeeded in having the planet moving westward at the correct speed and for the correct interval of time, thus replicating retrograde motion with his model.

\vspace{1em}

%%% Figure

\begin{mdframed}[backgroundcolor=black!10]
    \textbf{Link to Learning}

    \vspace{1ex}

    Use the \href{https://openstax.org/l/30ptolemaic}{Ptolemaic System simulator from Foothill AstroSims} to explore how Ptolemy's system of deferents and epicycles explained the apparent motion of the planets.
\end{mdframed}

\vspace{1em}

However, we shall see in Orbits and Gravity that the planets, like Earth, travel about the Sun in orbits that are ellipses, not circles. Their actual behavior cannot be represented accurately by a scheme of uniform circular motions. In order to match the observed motions of the planets, Ptolemy had to center the deferent circles, not on Earth, but at points some distance from Earth. In addition, he introduced uniform circular motion around yet another axis, called the equant point. All of these considerably complicated his scheme.

\vspace{1em}

It is a tribute to the genius of Ptolemy as a mathematician that he was able to develop such a complex system to account successfully for the observations of planets. It may be that Ptolemy did not intend for his cosmological model to describe reality, but merely to serve as a mathematical representation that allowed him to predict the positions of the planets at any time. Whatever his thinking, his model, with some modifications, was eventually accepted as authoritative in the Muslim world and (later) in Christian Europe.

\subsection{The Birth of Modern Astronomy}

Astronomy made no major advances in strife-torn medieval Europe. The birth and expansion of Islam after the seventh century led to a flowering of Arabic and Jewish cultures that preserved, translated, and added to many of the astronomical ideas of the Greeks. Many of the names of the brightest stars, for example, are today taken from the Arabic, as are such astronomical terms as ``zenith.''

\vspace{1em}

As European culture began to emerge from its long, dark age, trading with Arab countries led to a rediscovery of ancient texts such as \textit{Almagest} and to a reawakening of interest in astronomical questions. This time of rebirth (in French, ``renaissance'') in astronomy was embodied in the work of Copernicus (Figure ?.??).

\vspace{1em}

%%% Figure

\subsubsection*{Copernicus}

One of the most important events of the Renaissance was the displacement of Earth from the center of the universe, an intellectual revolution initiated by a Polish cleric in the sixteenth century. Nicolaus Copernicus was born in Torun, a mercantile town along the Vistula River. His training was in law and medicine, but his main interests were astronomy and mathematics. His great contribution to science was a critical reappraisal of the existing theories of planetary motion and the development of a new Sun-centered, or \gls{heliocentric}, model of the solar system. Copernicus concluded that Earth is a planet and that all the planets circle the Sun. Only the Moon orbits Earth (Figure ?.??).

\vspace{1em}

% Figure

Copernicus described his ideas in detail in his book \textit{De Revolutionibus Orbium Coelestium} (\textit{On the Revolution of Celestial Orbs}), published in 1543, the year of his death. By this time, the old Ptolemaic system needed significant adjustments to predict the positions of the planets correctly. Copernicus wanted to develop an improved theory from which to calculate planetary positions, but in doing so, he was himself not free of all traditional prejudices.

\vspace{1em}

He began with several assumptions that were common in his time, such as the idea that the motions of the heavenly bodies must be made up of combinations of uniform circular motions. But he did not assume (as most people did) that Earth had to be in the center of the universe, and he presented a defense of the heliocentric system that was elegant and persuasive. His ideas, although not widely accepted until more than a century after his death, were much discussed among scholars and, ultimately, had a profound influence on the course of world history.

\vspace{1em}

One of the objections raised to the heliocentric theory was that if Earth were moving, we would all sense or feel this motion. Solid objects would be ripped from the surface, a ball dropped from a great height would not strike the ground directly below it, and so forth. But a moving person is not necessarily aware of that motion. We have all experienced seeing an adjacent train, bus, or ship appear to move, only to discover that it is we who are moving.

\vspace{1em}

Copernicus argued that the apparent motion of the Sun about Earth during the course of a year could be represented equally well by a motion of Earth about the Sun. He also reasoned that the apparent rotation of the celestial sphere could be explained by assuming that Earth rotates while the celestial sphere is stationary. To the objection that if Earth rotated about an axis it would fly into pieces, Copernicus answered that if such motion would tear Earth apart, the still faster motion of the much larger celestial sphere required by the geocentric hypothesis would be even more devastating.

\subsubsection*{The Heliocentric Model}

The most important idea in Copernicus' \textit{De Revolutionibus} is that Earth is one of six (then-known) planets that revolve about the Sun. Using this concept, he was able to work out the correct general picture of the solar system. He placed the planets, starting nearest the Sun, in the correct order: Mercury, Venus, Earth, Mars, Jupiter, and Saturn. Further, he deduced that the nearer a planet is to the Sun, the greater its orbital speed. With his theory, he was able to explain the complex retrograde motions of the planets without epicycles and to work out a roughly correct scale for the solar system.

\vspace{1em}

Copernicus could not prove that Earth revolves about the Sun. In fact, with some adjustments, the old Ptolemaic system could have accounted, as well, for the motions of the planets in the sky. But Copernicus pointed out that the Ptolemaic cosmology was clumsy and lacking the beauty and symmetry of its successor.

\vspace{1em}

In Copernicus' time, in fact, few people thought there were ways to prove whether the heliocentric or the older geocentric system was correct. A long philosophical tradition, going back to the Greeks and defended by the Catholic Church, held that pure human thought combined with divine revelation represented the path to truth. Nature, as revealed by our senses, was suspect. For example, Aristotle had reasoned that heavier objects (having more of the quality that made them heavy) must fall to Earth faster than lighter ones. This is absolutely incorrect, as any simple experiment dropping two balls of different weights shows. However, in Copernicus' day, experiments did not carry much weight (if you will pardon the expression); Aristotle's reasoning was more convincing.

\vspace{1em}

In this environment, there was little motivation to carry out observations or experiments to distinguish between competing cosmological theories (or anything else). It should not surprise us, therefore, that the heliocentric idea was debated for more than half a century without any tests being applied to determine its validity. (In fact, in the North American colonies, the older geocentric system was still taught at Harvard University in the first years after it was founded in 1636.)

\vspace{1em}

Contrast this with the situation today, when scientists rush to test each new hypothesis and do not accept any ideas until the results are in. For example, when two researchers at the University of Utah announced in 1989 that they had discovered a way to achieve nuclear fusion (the process that powers the stars) at room temperature, other scientists at more than 25 laboratories around the United States attempted to duplicate ``cold fusion'' within a few weeks---without success, as it turned out. The cold fusion theory soon went down in flames.

\vspace{1em}

How would we look at Copernicus' model today? When a new hypothesis or theory is proposed in science, it must first be checked for consistency with what is already known. Copernicus' heliocentric idea passes this test, for it allows planetary positions to be calculated at least as well as does the geocentric theory. The next step is to determine which predictions the new hypothesis makes that differ from those of competing ideas. In the case of Copernicus, one example is the prediction that, if Venus circles the Sun, the planet should go through the full range of phases just as the Moon does, whereas if it circles Earth, it should not (Figure ?.??). Also, we should not be able to see the full phase of Venus from Earth because the Sun would then be between Venus and Earth. But in those days, before the telescope, no one imagined testing these predictions.

\vspace{1em}

% Figure

\begin{mdframed}[backgroundcolor=black!10]
    \textbf{Link to Learning}

    \vspace{1ex}

    This \href{https://openstax.org/l/30venusphases}{animation} shows the phases of Venus. You can also see its distance from Earth as it orbits the Sun.
\end{mdframed}

\subsubsection*{Galileo and the Beginning of Modern Science}

Many of the modern scientific concepts of observation, experimentation, and the testing of hypotheses through careful quantitative measurements were pioneered by a man who lived nearly a century after Copernicus. Galileo Galilei (Figure ?.??), a contemporary of Shakespeare, was born in Pisa. Like Copernicus, he began training for a medical career, but he had little interest in the subject and later switched to mathematics. He held faculty positions at the University of Pisa and the University of Padua, and eventually became mathematician to the Grand Duke of Tuscany in Florence.

\vspace{1em} % Figure

Galileo's greatest contributions were in the field of mechanics, the study of motion and the actions of forces on bodies. It was familiar to all persons then, as it is to us now, that if something is at rest, it tends to remain at rest and requires some outside influence to start it in motion. Rest was thus generally regarded as the natural state of matter. Galileo showed, however, that rest is no more natural than motion.

\vspace{1em}

If an object is slid along a rough horizontal floor, it soon comes to rest because friction between it and the floor acts as a retarding force. However, if the floor and the object are both highly polished, the object, given the same initial speed, will slide farther before stopping. On a smooth layer of ice, it will slide farther still. Galileo reasoned that if all resisting effects could be removed, the object would continue in a steady state of motion indefinitely. He argued that a force is required not only to start an object moving from rest but also to slow down, stop, speed up, or change the direction of a moving object. You will appreciate this if you have ever tried to stop a rolling car by leaning against it, or a moving boat by tugging on a line.

\vspace{1em}

Galileo also studied the way objects accelerate---change their speed or direction of motion. Galileo watched objects as they fell freely or rolled down a ramp. He found that such objects accelerate uniformly; that is, in equal intervals of time they gain equal increments in speed. Galileo formulated these newly found laws in precise mathematical terms that enabled future experimenters to predict how far and how fast objects would move in various lengths of time.

\begin{mdframed}[backgroundcolor=black!10]
    \textbf{Link to Learning}

    \vspace{1ex}

    In theory, if Galileo is right, a feather and a hammer, dropped at the same time from a height, should land at the same moment. On Earth, this experiment is not possible because air resistance and air movements make the feather flutter, instead of falling straight down, accelerated only by the force of gravity. For generations, physics teachers had said that the place to try this experiment is somewhere where there is no air, such as the Moon. In 1971, Apollo 15 astronaut David Scott took a hammer and feather to the Moon and tried it, to the delight of physics nerds everywhere. NASA provides the \href{https://openstax.org/l/30HamVsFeath}{video of the hammer and feather} as well as a brief explanation.
\end{mdframed}

\vspace{1em}

Sometime in the 1590s, Galileo adopted the Copernican hypothesis of a heliocentric solar system. In Roman Catholic Italy, this was not a popular philosophy, for Church authorities still upheld the ideas of Aristotle and Ptolemy, and they had powerful political and economic reasons for insisting that Earth was the center of creation. Galileo not only challenged this thinking but also had the audacity to write in Italian rather than scholarly Latin, and to lecture publicly on those topics. For him, there was no contradiction between the authority of the Church in matters of religion and morality, and the authority of nature (revealed by experiments) in matters of science. It was primarily because of Galileo and his ``dangerous'' opinions that, in 1616, the Church issued a prohibition decree stating that the Copernican doctrine was ``false and absurd'' and not to be held or defended.

\subsubsection*{Galileo's Astronomical Observations}

It is not certain who first conceived of the idea of combining two or more pieces of glass to produce an instrument that enlarged images of distant objects, making them appear nearer. The first such ``spyglasses'' (now called telescopes) that attracted much notice were made in 1608 by the Dutch spectacle maker Hans Lippershey (1570--1619). Galileo heard of the discovery and, without ever having seen an assembled telescope, constructed one of his own with a three-power magnification ($3\times$), which made distant objects appear three times nearer and larger (Figure ?.??).

\vspace{1em} % Figure

On August 25, 1609, Galileo demonstrated a telescope with a magnification of 9× to government officials of the city-state of Venice. By a magnification of $9\times$, we mean the linear dimensions of the objects being viewed appeared nine times larger or, alternatively, the objects appeared nine times closer than they really were. There were obvious military advantages associated with a device for seeing distant objects. For his invention, Galileo's salary was nearly doubled, and he was granted lifetime tenure as a professor. (His university colleagues were outraged, particularly because the invention was not even original.)

\vspace{1em}

Others had used the telescope before Galileo to observe things on Earth. But in a flash of insight that changed the history of astronomy, Galileo realized that he could turn the power of the telescope toward the heavens. Before using his telescope for astronomical observations, Galileo had to devise a stable mount and improve the optics. He increased the magnification to $30\times$. Galileo also needed to acquire confidence in the telescope.

\vspace{1em}

At that time, human eyes were believed to be the final arbiter of truth about size, shape, and color. Lenses, mirrors, and prisms were known to distort distant images by enlarging, reducing, or inverting them, or spreading the light into a spectrum (rainbow of colors). Galileo undertook repeated experiments to convince himself that what he saw through the telescope was identical to what he saw up close. Only then could he begin to believe that the miraculous phenomena the telescope revealed in the heavens were real.

\vspace{1em}

Beginning his astronomical work late in 1609, Galileo found that many stars too faint to be seen with the unaided eye became visible with his telescope. In particular, he found that some nebulous blurs resolved into many stars, and that the Milky Way---the strip of whiteness across the night sky---was also made up of a multitude of individual stars.

\vspace{1em}

Examining the planets, Galileo found four moons revolving about Jupiter in times ranging from just under 2 days to about 17 days. This discovery was particularly important because it showed that not everything has to revolve around Earth. Furthermore, it demonstrated that there could be centers of motion that are themselves in motion. Defenders of the geocentric view had argued that if Earth was in motion, then the Moon would be left behind because it could hardly keep up with a rapidly moving planet. Yet, here were Jupiter's moons doing exactly that. (To recognize this discovery and honor his work, NASA named a spacecraft that explored the Jupiter system Galileo.)

\vspace{1em}

With his telescope, Galileo was able to carry out the test of the Copernican theory mentioned earlier, based on the phases of Venus. Within a few months, he had found that Venus goes through phases like the Moon, showing that it must revolve about the Sun, so that we see different parts of its daylight side at different times (see Figure ?.??.) These observations could not be reconciled with Ptolemy's model, in which Venus circled about Earth. In Ptolemy's model, Venus could also show phases, but they were the wrong phases in the wrong order from what Galileo observed.

\vspace{1em}

Galileo also observed the Moon and saw craters, mountain ranges, valleys, and flat, dark areas that he thought might be water. These discoveries showed that the Moon might be not so dissimilar to Earth---suggesting that Earth, too, could belong to the realm of celestial bodies.

\begin{mdframed}[backgroundcolor=black!10]
    \textbf{Link to Learning}

    \vspace{1ex}

    For more information about the life and work of Galileo, see the \href{https://openstax.org/l/30GalProj}{Galileo Project} at Rice University.
\end{mdframed}

After Galileo's work, it became increasingly difficult to deny the Copernican view, and Earth was slowly dethroned from its central position in the universe and given its rightful place as one of the planets attending the Sun. Initially, however, Galileo met with a great deal of opposition. The Roman Catholic Church, still reeling from the Protestant Reformation, was looking to assert its authority and chose to make an example of Galileo. He had to appear before the Inquisition to answer charges that his work was heretical, and he was ultimately condemned to house arrest. His books were on the Church's forbidden list until 1836, although in countries where the Roman Catholic Church held less sway, they were widely read and discussed. Not until 1992 did the Catholic Church admit publicly that it had erred in the matter of censoring Galileo's ideas.

\vspace{1em}

The new ideas of Copernicus and Galileo began a revolution in our conception of the cosmos. It eventually became evident that the universe is a vast place and that Earth's role in it is relatively unimportant. The idea that Earth moves around the Sun like the other planets raised the possibility that they might be worlds themselves, perhaps even supporting life. As Earth was demoted from its position at the center of the universe, so, too, was humanity. The universe, despite what we may wish, does not revolve around us.

\vspace{1em}

Most of us take these things for granted today, but four centuries ago such concepts were frightening and heretical for some, immensely stimulating for others. The pioneers of the Renaissance started the European world along the path toward science and technology that we still tread today. For them, nature was rational and ultimately knowable, and experiments and observations provided the means to reveal its secrets.

\begin{mdframed}[backgroundcolor=black!10]
    \textbf{Seeing for yourself: Observing the Planets}

    \vspace{1ex}

    At most any time of the night, and at any season, you can spot one or more bright planets in the sky. All five of the planets known to the ancients---Mercury, Venus, Mars, Jupiter, and Saturn---are more prominent than any but the brightest stars, and they can be seen even from urban locations if you know where and when to look. One way to tell planets from bright stars is that planets twinkle less.

    \vspace{1em}

    Venus, which stays close to the Sun from our perspective, appears either as an ``evening star'' in the west after sunset or as a ``morning star'' in the east before sunrise. It is the brightest object in the sky after the Sun and Moon. It far outshines any real star, and under the most favorable circumstances, it can even cast a visible shadow. Some young military recruits have tried to shoot Venus down as an approaching enemy craft or UFO.

    \vspace{1em}
    
    Mars, with its distinctive red color, can be nearly as bright as Venus is when close to Earth, but normally it remains much less conspicuous. Jupiter is most often the second-brightest planet, approximately equaling in brilliance the brightest stars. Saturn is dimmer, and it varies considerably in brightness, depending on whether its large rings are seen nearly edge-on (faint) or more widely opened (bright).

    \vspace{1em}
    
    Mercury is quite bright, but few people ever notice it because it never moves very far from the Sun (it's never more than \ang{28} away in the sky) and is always seen against bright twilight skies.

    \vspace{1em}
    
    True to their name, the planets ``wander'' against the background of the ``fixed'' stars. Although their apparent motions are complex, they reflect an underlying order upon which the heliocentric model of the solar system, as described in this chapter, was based. The positions of the planets are often listed in newspapers (sometimes on the weather page), and clear maps and guides to their locations can be found each month in such magazines as \textit{Sky \& Telescope} and \textit{Astronomy} (available at most libraries and online). There are also a number of computer programs and phone and tablet apps that allow you to display where the planets are on any night.
\end{mdframed}

















\end{document}