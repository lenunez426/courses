\documentclass[main-astronomy.tex]{subfiles}

\begin{document}

Kepler's three laws of planetary motion can be summarized as follows:

\begin{itemize}
    \item \gls{Kepler's first law}: Each planet moves around the Sun in an orbit that is an ellipse, with the Sun at one focus of the ellipse.
    \item \gls{Kepler's second law}: The straight line joining a planet and the Sun sweeps out equal areas in space in equal intervals of time.
    \item \gls{Kepler's third law}: The square of a planet's orbital period is directly proportional to the cube of the semimajor axis of its orbit.
\end{itemize}

Kepler's three laws provide a precise geometric description of planetary motion within the framework of the Copernican system. With these tools, it was possible to calculate planetary positions with greatly improved precision. Still, Kepler's laws are purely descriptive: they do not help us understand what forces of nature constrain the planets to follow this particular set of rules. That step was left to Isaac Newton.

\begin{example}
    \textit{Applying Kepler's Law}\\
    Using the orbital periods and semimajor axes for Venus and Earth that are provided here, calculate $P^2$ and $a^3$, and verify that they obey Kepler's third law. Venus' orbital period is \SI{0.62}{year}, and its semimajor axis is \SI{0.72}{AU}. Earth's orbital period is \SI{1.00}{year}, and its semimajor axis is \SI{1.00}{AU}.
\end{example}

\Solution We can use the equation for Kepler’s third law, $P^2 \propto a^3$. For Venus, $P^2= 0.62 \times 0.62 = 0.38$ and $a^3 = 0.72 \times 0.72 \times 0.72 =0.37$ (rounding numbers sometimes causes minor discrepancies like this). The square of the orbital period (0.38) approximates the cube of the semimajor axis (0.37). Therefore, Venus obeys Kepler's third law. For Earth,  $P^2= 1.00 \times 1.00 = 1.00$ and $a^3 = 1.00 \times 1.00 \times 1.00 = 1.00$. The square of the orbital period (1.00) approximates (in this case, equals) the cube of the semimajor axis (1.00). Therefore, Earth obeys Kepler's third law.

\endsolution

\begin{cfu}
    Using the orbital periods and semimajor axes for Saturn and Jupiter that are provided here, calculate $P^2$ and $a^3$, and verify that they obey Kepler's third law. Saturn's orbital period is 29.46 years, and its semimajor axis is \SI{9.54}{AU}. Jupiter's orbital period is 11.86 years, and its semimajor axis is \SI{5.20}{AU}.
\end{cfu}

\Solution For Saturn, $P^2 = 29.46 \times 29.46 = 867.9$ and $a^3 = 9.54 \times 9.54 \times 9.54 = 868.3$. The square of the orbital period (867.9) approximates the cube of the semimajor axis (868.3). Therefore, Saturn obeys Kepler's third law.

\endsolution

\begin{gradient}{LINK TO LEARNING}
    In honor of the scientist who first devised the laws that govern the motions of planets, the team that built the first spacecraft to search for planets orbiting other stars decided to name the probe ``Kepler.'' Visit NASA's Kepler website to learn more about Johannes Kepler's life and his laws of planetary motion. \href{https://openstax.org/l/30nasakepmiss}{NASA's Kepler website} and follow the links that interest you.
\end{gradient}








\end{document}