\documentclass{exam}
\usepackage[utf8]{inputenc}
\usepackage[english]{babel}
\usepackage{geometry}
\usepackage[T1]{fontenc}
\usepackage{graphicx}
\graphicspath{ {../Figures/} }
\setlength\parindent{0pt}
\usepackage{hyperref}
\hypersetup{colorlinks=true,linkcolor=blue,filecolor=magenta,urlcolor=cyan,}
\urlstyle{same}
\usepackage{amsthm}
\usepackage{amsmath}
\theoremstyle{definition}
\usepackage{pgfplots}
\usepackage{caption}
\usepackage{subcaption}
\usepackage{makecell}
\usepackage[table]{colortbl}
\usepackage{enumitem}
\usepackage{siunitx}
\usepackage{amssymb}
\usepackage{tikz-cd}
\tikzset{>=latex}
\usepackage{tkz-euclide}
\usepackage{tikz,bm}
\usepackage{mwe,tikz}
\usetikzlibrary{arrows}
\pgfplotsset{compat=1.11}
\usepackage{moresize}
\usepackage{bohr}
\usetikzlibrary{patterns}
\usepackage{wrapfig}
\usepackage{mdframed}
\usepackage{dashrule}
\usepackage{tikzsymbols}
\usepackage{fontawesome}
\usepackage{linearb} %for \BPwheel symbol in Unit 6
\usepackage{multicol}
\usepackage{glossaries}
\usepackage{cancel}
% \usepackage{circuitikz}
\sisetup{group-separator = {,}}
\usepgfplotslibrary{fillbetween}
\usetikzlibrary{math}
\numberwithin{equation}{section}
\numberwithin{figure}{section}




\DeclareSIUnit{\nothing}{\relax}
\def\mymu{\SI{}{\micro\nothing} }

\newtheorem{example}{Example}[section]
\newtheorem{exercise}{}[section]
\newtheorem{regla}{Rule}


\newcommand{\Solution}{{\footnotesize \color{cyan} SOLUTION }}


\def\endsolution{{\footnotesize \color{cyan} \hfill END OF SOLUTION }}

\newcommand{\cyanhrule}{{\color{cyan} \hrule }}

\def\redplus{\mathbin{\color{red} +}}
\def\redminus{\mathbin{\color{red} -}}
\def\redtimes{\mathbin{\color{red} \times}}


\def\openstax{https://openstax.org/books/physics/pages/1-introduction}
\def\openstaxfooter{\fancyfoot[C]{Access for free at \href{\openstax}{\openstax} \hfill \thepage}}


% The following needs to be added to each individual main.tex file so it doesn't interfere with exam headers:

%\usepackage{fancyhdr}
% \pagestyle{fancy}
% \renewcommand{\headrulewidth}{0pt}
% \renewcommand{\headruleskip}{0mm}
% \fancyhead{}
% \def\openstax{https://openstax.org/books/physics/pages/1-introduction}
% \def\openstaxfooter{\fancyfoot[C]{Access for free at \href{\openstax}{\openstax} \hfill \thepage}}

\newcommand\myboxa[2][]{\tikz[overlay]\node[fill=gray!20,inner sep=4pt, anchor=text, rectangle, rounded corners=1mm,#1] {#2};\phantom{#2}}
\newcommand{\hgraydashline}{{\color{lightgray} \hdashrule{0.99\textwidth}{1pt}{0.8mm}}}

\let\oldtexttt\texttt% Store \texttt
\renewcommand{\texttt}[2][black]{\textcolor{#1}{\ttfamily #2}}% 

\newcommand\mybox[2][]{\tikz[overlay]\node[fill=black!20,inner sep=2pt, anchor=text, rectangle, rounded corners=1mm,#1] {#2};\phantom{#2}}

\setlength{\columnsep}{1cm}
\setlength{\columnseprule}{1pt}
\def\columnseprulecolor{\color{cyan}}

\pgfdeclarehorizontalshading{visiblelight}{50bp}{
color(0.00000000000000bp)=(red);
color(8.33333333333333bp)=(orange);
color(16.66666666666670bp)=(yellow);
color(25.00000000000000bp)=(green);
color(33.33333333333330bp)=(cyan);
color(41.66666666666670bp)=(blue);
color(50.00000000000000bp)=(violet)
}

\def\myfillin{\rule{2cm}{0.15mm}}

\def\phet{\texttt[red]{PhET} }

\usepackage{circuitikz}

\usepackage{utfsym} %to get symbol of car
\def\mycar{\reflectbox{\huge\usym{1F697}} } %modiying symbol of car
\def\mycarleft{\huge\usym{1F697}}%modiying symbol of car

\def\mytrain{\reflectbox{\huge\usym{1F682}} } %modiying symbol of train
\def\mytrainleft{\huge\usym{1F682}}%modiying symbol of train


\usepackage{nameref}

\setenumerate{itemsep=-2pt,topsep=0pt,leftmargin=4em}




\usepackage{exam-randomizechoices}

\def\one{1}
\def\two{2}
\def\three{3}
\def\four{4}

\def\testVersion{\one} % Edit Version number here only.

\setrandomizerseed{\testVersion}

\pagestyle{headandfoot}
\firstpageheader{Physics}{Test on Unit 6: Momentum}{Version \testVersion}
\CorrectChoiceEmphasis{\color{red}\bfseries}
\SolutionEmphasis{\color{red}}

%\printanswers

\begin{document}

\begin{center}
\fbox{\fbox{\parbox{5.5in}{
%DO NOT WRITE ON THIS DOCUMENT. Answer the questions on the provided scantron. 
You may find the equations below useful.

\begin{equation*}
    p = mv \qquad
    \Delta p = p_f - p_i \qquad
    \Delta p = m\left(v_f - v_i\right) \qquad
\end{equation*}
\begin{equation*}
    \Delta p = F_{\text{net}}\,\Delta t \hspace{1em}
    \text{\footnotesize (impulse-momentum theorem)}
\end{equation*}
\begin{equation*}
    m_1 v_1 + m_2 v_2 = m_1 v_1^{\prime} + m_2 v_2^{\prime} \hspace{1em}
    \text{\footnotesize(law of conservation; elastic collisions)}
\end{equation*}
\begin{equation*}
    m_1 v_1 + m_2 v_2 = \left(m_1 + m_2\right)\,v' \hspace{1em} \text{\footnotesize (inelastic collisions)}
\end{equation*}

}}}
\end{center}

\begin{center}
\fbox{\fbox{\parbox{5.5in}{
\begin{center}
\vspace{-1em}
  \textbf{USEFUL EQUATIONS}  
\end{center}
\vspace{-1.5em}

\begin{equation*}
    W = F d \hspace{5mm} 
    \mathrm{weight} = mg \hspace{5mm}
    g = \SI{9.8}{m/s^2} \hspace{5mm}
    P = \frac{W}{t} \hspace{5mm}
\end{equation*}%
\begin{equation*}
    \mathrm{PE_g} = mgh \hspace{5mm}
    \mathrm{KE} = \frac{1}{2} m v^2 \hspace{5mm}
    \mathrm{ME} = \mathrm{KE} + \mathrm{PE} = \text{constant} \hspace{5mm}
    \mathrm{KE_i} + \mathrm{PE_i} = \mathrm{KE_f} + \mathrm{PE_f}
\end{equation*}
}}}
\end{center}
\begin{questions}

\question
What is momentum?

\begin{randomizechoices}
\choice The product of a system's net force and time.
\choice The product of a system's mass and change in velocity.
\correctchoice The product of system's mass and velocity.
\choice The product of a system's mass and acceleration.
\end{randomizechoices}

\question
A \SI{70}{kg} skier leaves a ski jump at a velocity of \SI{14}{m/s}. What is the skier's momentum at that instant?

\begin{randomizechoices}
\choice \SI{50}{kg\,m/s}
\choice \SI{9800}{kg\,m/s}
\correctchoice \SI{980}{kg\,m/s}
\choice \SI{5}{kg\,m/s}
\end{randomizechoices}

\question
A child on a sled moves down a hill at 20 meters per second. The combined mass of the child-sled system is 100 kilograms. What is the momentum of the system?

\begin{randomizechoices}
\choice \SI{20}{kg\,m/s}
\correctchoice \SI{2000}{kg\,m/s}
\choice \SI{5}{kg\,m/s}
\choice \SI{1000}{kg\,m/s}
\end{randomizechoices}

\question
Which of the following vehicles has the greatest momentum?

\begin{randomizechoices}
\correctchoice a car with a mass of \SI{1210}{kg} moving at a velocity of \SI{51}{m/s}
\choice a truck with a mass of \SI{6120}{kg} moving at a velocity of \SI{10}{m/s}
\choice a car with a mass of \SI{1540}{kg} moving at a velocity of \SI{38}{m/s}
\choice a truck with a mass of \SI{2250}{kg} moving at a velocity of \SI{25}{m/s}
\end{randomizechoices}

\question
A ball is hit with a bat. A student determines that the momentum of the ball is \SI{1.0}{kg\,m/s}. If the ball's velocity is 2.0 meters per second, what is the ball's mass?

\begin{randomizechoices}
\choice \SI{1.0}{kg}
\choice \SI{2.0}{kg}
\correctchoice \SI{0.5}{kg}
\choice \SI{0.0}{kg}
\end{randomizechoices}

\question
Freddy, the \SI{2.0}{kg} fish, travels through the Pacific Ocean at a velocity of \SI{5.0}{m/s} when a temporary force causes Freddy to slow down to a velocity of \SI{2.0}{m/s}. What is Freddy's change in momentum?

\begin{randomizechoices}
\choice \SI{6.0}{kg\,m/s}
\choice \SI{3.0}{kg\,m/s}
\correctchoice \SI{-6.0}{kg\,m/s}
\choice \SI{-3.0}{kg\,m/s}
\end{randomizechoices}

\question
Robbie Ray throws a \SI{0.145}{kg} baseball at a velocity of \SI{-43}{m/s}, and Yordan Alvarez hits the ball back at \SI{37}{m/s}. Assuming the ball travels in one dimension only, what is the baseball's change in momentum?

\begin{randomizechoices}
\choice \SI{5.36}{kg\,m/s}
\choice \SI{6.23}{kg\,m/s}
\correctchoice \SI{11.6}{kg\,m/s}
\choice \SI{0.87}{kg\,m/s}
\end{randomizechoices}

\question
Impulse is the product of \fillin\ .

\begin{randomizechoices}
\choice mass and velocity ($mv$)
\choice mass and acceleration ($ma$)
\correctchoice net force and time ($F_{\text{net}}\,\Delta t$)
\choice net force and velocity ($F_{\text{net}}\,v$)
\end{randomizechoices}

\question
According to the impulse-momentum theorem, the impulse experienced by an object is equivalent to the object's \fillin\ .

\begin{randomizechoices}
\choice momentum
\correctchoice change in momentum
\choice velocity
\choice force
\end{randomizechoices}

% \question
% What is the impulse-momentum theorem in equation form?

% \begin{randomizechoices}
% \choice $p=mv$
% \choice $F_{\text{net}} = ma$
% \choice $\Delta p = m (v_f - v_i)$
% \correctchoice $\Delta p = F_{\text{net}}\,\Delta t$
% \end{randomizechoices}

% \question
% Which two quantities can be expressed using the same units?

% \begin{randomizechoices}
% \choice momentum and energy
% \correctchoice impulse and momentum
% \choice energy and force
% \choice force and impulse
% \end{randomizechoices}

\question
According to the impulse-momentum theorem, why is it that, if you fall off your roof, you will experience a lesser net force ($F_{\text{net}}$) when you land on a trampoline compared to when you land on the ground?

\begin{randomizechoices}
\choice The trampoline decreases the elapsed time ($\Delta t$) during which you come to a stop.
\correctchoice The trampoline increases the elapsed time ($\Delta t$) during which you come to a stop.
\choice The trampoline increases your change in momentum ($\Delta p$).
\choice The trampoline decreases your change in momentum ($\Delta p$).
\end{randomizechoices}

% \question
% An egg dropped on the road usually breaks, while one dropped on the grass usually does not break. The impulse-momentum theorem suggests that for the egg dropped on the grass 

% \begin{randomizechoices}
% \choice the time interval for stopping is less.
% \correctchoice the time interval for stopping is greater.
% \choice the change in momentum is greater.
% \choice the change in momentum is less.
% \end{randomizechoices}

\question
A ball is hit with a force of \SI{40}{N}. It is in contact with the bat for 0.02 seconds. What is the change in momentum of the ball?

\begin{randomizechoices}
\choice \SI{2000}{kg\,m/s}
\choice \SI{0.08}{kg\,m/s}
\choice \SI{8.0}{kg\,m/s}
\correctchoice \SI{0.8}{kg\,m/s}
\end{randomizechoices}

% \question
% What impulse will a \SI{25}{kg} cart experience if it starts at rest and accelerates to a speed of \SI{12}{m/s}?

% \begin{randomizechoices}
% \choice \SI{2.10}{N\,s}
% \correctchoice \SI{300}{N\,s}
% \choice \SI{0.48}{N\,s}
% \choice \SI{13.0}{N\,s}
% \end{randomizechoices}

% \question
% A \SI{0.45}{kg} football traveling at a speed of \SI{22}{m/s} is caught by an \SI{84}{kg} stationary receiver. If the football comes to rest in the receiver's arms, what is the magnitude of the impulse imparted on the ball by the receiver's hands?

% \begin{randomizechoices}
% \choice \SI{3.8}{N\,s}
% \choice \SI{4.4}{N\,s}
% \correctchoice \SI{9.9}{N\,s}
% \choice \SI{1800}{N\,s}
% \end{randomizechoices}

\question
The law of conservation of momentum states that

\begin{randomizechoices}
    \choice the total momentum of all objects interacting with one another is zero
    \correctchoice the total initial momentum of all objects before an interaction equals the total final momentum after the interaction
    \choice the total initial momentum of all objects interacting with one another does not equal the total final momentum  
    \choice the total initial momentum of all objects is zero 
\end{randomizechoices}

\clearpage
\question
Soccer ball 1 collides with soccer ball 2, which was at rest. The total momentum of the two-object system after the collision \fillin\ .

\begin{randomizechoices}
    \choice is zero
    \choice increases
    \correctchoice remains constant
    \choice decreases
\end{randomizechoices}

\question
A \SI{60}{kg} student on ice skates stands at rest on a frictionless frozen pond holding a \SI{10}{kg} brick. He throws the brick eastward with a speed of \SI{18}{m/s}. What is the student's recoil velocity?

\begin{randomizechoices}
\choice \SI{3.0}{m/s} east
\correctchoice \SI{3.0}{m/s} west
\choice \SI{18}{m/s} west
\choice \SI{18}{m/s} east
\end{randomizechoices}

\question
A \SI{1000}{kg} cannon fires a \SI{10}{kg} projectile horizontally at a velocity of \SI{300}{m/s}. What is the recoil velocity of the cannon?

\begin{randomizechoices}
\choice \SI{0.3}{m/s}
\correctchoice \SI{3.0}{m/s}
\choice \SI{300}{m/s}
\choice \SI{30}{m/s}
\end{randomizechoices}

\question
Two shopping carts stick together and move with the same velocity after colliding. This collision is \fillin\ one.

\begin{randomizechoices}
\choice an impulse
\correctchoice an inelastic
\choice an elastic
\choice a momentum
\end{randomizechoices}

\question
After two billiard balls collide, they veer off in different directions. This is \fillin\ collision.

\begin{randomizechoices}
\choice an impulse
\choice an inelastic
\correctchoice an elastic
\choice a momentum
\end{randomizechoices}

\question
A car is hit by a train. The car sticks to the front of the train and is dragged down the tracks. What type of collision is this?

\begin{randomizechoices}
\choice impulse
\choice elastic collision
\correctchoice inelastic collision
\choice momentum
\end{randomizechoices}

\clearpage


\begin{EnvUplevel}
Read the statements below. Then answer questions \ref{SGv3JM} and \ref{jlNXIi}.

\textbf{Two medicine balls elastically collide. Their masses and velocities before the collision are shown in the figure below.}
\end{EnvUplevel}

\begin{center}
\def\xa{1.5}
\def\xb{8.5}
\def\y{3}
\def\rb{6mm}
\begin{tikzpicture}
\begin{axis}[width=8cm, height=3cm,
    ticks=none,
    axis line style={draw=none},
    clip=false,
    xmin=0,xmax=10,
    ymin=0,ymax=10,
]
    \draw[thick,->] (\xa,\y) -- ++(axis direction cs: 3,0) node[above] {$\SI{2.75}{m/s}$};
    \fill[black!15] (\xa-0.3,\y) circle (3mm);
    \fill[black!30] (\xa-0.15,\y) circle (3mm);
    \draw[fill=black] (\xa,\y) circle (3mm) node[below=3mm] {$\SI{2.00}{kg}$} node[above=3mm] {$m_1$};
    \fill[black!15] (\xb+0.3,\y) circle (\rb);
    \fill[black!30] (\xb+0.15,\y) circle (\rb);
    \draw[thick,->] (\xb,\y) -- ++(axis direction cs: -2.3,0) node[below] {$\SI{-1.00}{m/s}$};
    \draw[fill=black] (\xb,\y) circle (\rb) node[below=\rb] {$\SI{3.00}{kg}$} node[above=\rb] {$m_2$};
\end{axis}
\end{tikzpicture}
\end{center}

\question \label{SGv3JM}
If the first object's velocity after the collision is \SI{-1.75}{m/s}, how fast will the second ball move after the collision?

\begin{randomizechoices}
\choice \SI{1.00}{m/s}
\correctchoice \SI{2.00}{m/s} 
\choice \SI{1.75}{m/s}
\choice \SI{3.25}{m/s}
\end{randomizechoices}

\question \label{jlNXIi}
In which direction will the second ball travel after the collision?

\begin{randomizechoices}
\correctchoice right
\choice left
\choice no direction (it will stop moving)
\choice It's not possible to determine with the given information.
\end{randomizechoices}

\cyanhrule

\begin{EnvUplevel}
Read the description below. Then answer questions \ref{} and \ref{}.

\textbf{A \SI{32}{kg} object moving at \SI{6.7}{m/s} to the RIGHT collides with a \SI{220}{kg} object moving with a speed of \SI{5.9}{m/s} to the LEFT. The objects undergo an elastic collision.}
\end{EnvUplevel}

\question
 If after the collision the first object bounces backward with a speed of \SI{15.3}{m/s}, what is speed of the second object? (\textit{Hint}: Draw a sketch. Remember that velocity is speed AND direction, and speed does not indicate direction.)

\begin{randomizechoices}
\correctchoice \SI{2.7}{m/s}
\choice \SI{3.6}{m/s}
\choice \SI{0.0}{m/s}
\choice \SI{7.6}{m/s}
\end{randomizechoices}

 \question
 In which direction does the second object move after the collision?

\begin{randomizechoices}
\correctchoice Left
\choice Right
\choice No direction (it stops moving).
\choice Up
\end{randomizechoices}

\cyanhrule

\clearpage
\question
Each croquet ball in a set has a mass of \SI{0.50}{kg}. The \textbf{green} ball travels at \SI{10.5}{m/s} and strikes a stationary \textbf{red} ball. If the \textbf{green} ball stops moving after colliding with the \textbf{red} one, what is the final speed of the \textbf{red} ball after the collision?

\begin{randomizechoices}
\choice \SI{12.0}{m/s}
\choice \SI{9.6}{m/s}
\correctchoice \SI{10.5}{m/s}
\choice \SI{6.0}{m/s}
\end{randomizechoices}

\question
Jennifer loads a shopping cart with \SI{75}{kg} of groceries. She launches the cart at \SI{15}{m/s} into another \SI{315}{kg} shopping cart that is initially at rest. The two carts stick together after the collision. What is the final velocity of the two-object system?

\begin{randomizechoices}
\correctchoice \SI{2.9}{m/s}
\choice \SI{12}{m/s}
\choice \SI{19}{m/s}
\choice \SI{3.6}{m/s}
\end{randomizechoices}

\cyanhrule

\begin{EnvUplevel}
Read the description below. Then answer questions \ref{} and \ref{}.

\textbf{Two football players undergo an inelastic collision. The figure below shows their masses and their velocities before the collision.}
\end{EnvUplevel}
 



\begin{center}
\def\xa{1.5}
\def\xb{8.5}
\def\y{3}
\def\rb{6mm}
\begin{tikzpicture}
\begin{axis}[width=8cm, height=3cm,
    ticks=none,
    axis line style={draw=none},
    clip=false,
    xmin=0,xmax=10,
    ymin=0,ymax=10,
]
    \draw[thick,->] (\xa,\y) -- ++(axis direction cs: 3,0) node[above] {$\SI{7.23}{m/s}$};
    \fill[black!15] (\xa-0.3,\y) circle (3mm);
    \fill[black!30] (\xa-0.15,\y) circle (3mm);
    \draw[fill=black] (\xa,\y) circle (3mm) node[below=3mm] {$\SI{110}{kg}$} node[above=3mm] {$m_1$};
    \fill[black!15] (\xb+0.3,\y) circle (\rb);
    \fill[black!30] (\xb+0.15,\y) circle (\rb);
    \draw[thick,->] (\xb,\y) -- ++(axis direction cs: -2.2,0) node[below] {$\SI{-3.55}{m/s}$};
    \draw[fill=black] (\xb,\y) circle (\rb) node[below=\rb] {$\SI{135}{kg}$} node[above=\rb] {$m_2$};
\end{axis}
\end{tikzpicture}%
\end{center}

\question
At what speed is the combined system moving after the collision?

\begin{randomizechoices}
\correctchoice \SI{1.29}{m/s} 
\choice \SI{0.00}{m/s}
\choice \SI{3.68}{m/s}
\choice \SI{10.35}{m/s}
\end{randomizechoices}

\question
In what direction are the players moving after the collision?

\begin{randomizechoices}
\choice To the left.
\choice No direction: the players cancel each other's momentum.
\choice The law of conservation of momentum cannot indicate direction.
\correctchoice To the right.
\end{randomizechoices}

\question
A boy pushes his wagon at constant speed along a level sidewalk. The graph below represents the relationship between the horizontal force exerted by the boy and the
distance the wagon moves.

\textbf{What is the total work done by the boy in pushing the wagon 4.0 meters?}

\begin{center}
\begin{minipage}{0.2\textwidth}
\begin{randomizechoices}
\choice \SI{7.5}{J}
\correctchoice \SI{120}{J}
\choice \SI{5.0}{J}
\choice \SI{180}{J}
\end{randomizechoices}
\end{minipage}%
\hspace{2em}
\begin{minipage}{0.4\textwidth}
\centering
\begin{tikzpicture}
\begin{axis}[width=6cm,height=4cm,
    xmin=0,xmax=7,
    ymin=0,ymax=45,
    clip=false,
    axis lines=left,
    xlabel={Distance (m)},
    ylabel={Force (N)},
    xtick={0,1,...,6},
    ytick={0,10,...,40}
]
\addplot[very thick,
    color=black
    ]
    coordinates{
        (0,30)(6,30)
    };
\end{axis}
\end{tikzpicture}
\end{minipage}
\end{center}

\question
Imagine a ball on a track starts from rest at the position labeled \texttt{start} and moves down the track towards other positions. Assume no energy is transferred between the ball and the track or between the ball and the air around it. What is the highest position the ball will reach before stopping and going back down the track? 

\begin{center}
\begin{minipage}{0.3\textwidth}
    \begin{randomizechoices}[keeplast]
        \correctchoice 4
        \choice 3
        \choice 2
        \choice 5
        \choice None of the above
    \end{randomizechoices}
\end{minipage}%
\hspace{1em}
\begin{minipage}{0.4\textwidth}
\begin{center}
\begin{tikzpicture}
\begin{axis}[width=8cm,height=5cm,
    clip=false,
    xmin=-3,xmax=3,ymin=0,ymax=6,
    axis lines=right,
    axis line style={draw=none},
    %ticks=none,
    xtick style={draw=none},
    xticklabels={,,},
    ymajorgrids=true,
    ytick={0,1,...,6},
]

\addplot[ultra thick,samples=100,
    domain=-2.5:2.5
]{x^2}; 
\fill (-2*0.9,4) circle (2mm) node[left=2mm] {\texttt{start}};

\end{axis}
\end{tikzpicture}
\end{center}
\end{minipage}
\end{center}

\question
A \SI{12.5}{kg} glider is observed flying at an altitude of \SI{1510}{m} at a constant velocity of \SI{18}{m/s}. The glider dives to a new altitude of \SI{1250}{m}. Neglecting the effects of air resistance, what is the change in the glider's gravitational potential energy? (\textit{Hint}: first calculate the \textit{change} in height).

\begin{randomizeoneparchoices}
\choice \SI{338000}{J}
\choice \SI{185000}{J}
\correctchoice \SI{-31850}{J}
\choice \SI{-15300}{J}
\end{randomizeoneparchoices}

\question
A diver with a mass of \SI{80.0}{kg} dives off a \SI{10.0}{m} platform. His velocity just before striking the water is \SI{14.0}{m/s}. What is his kinetic energy at that moment?

\begin{randomizeoneparchoices}
\choice \SI{800}{m/s}
\choice \SI{8000}{m/s}
\correctchoice \SI{7840}{m/s}
\choice \SI{7500}{m/s}
\end{randomizeoneparchoices}

\question
As a ball falls freely (without air resistance) toward the ground, its total mechanical energy\ldots 

\begin{randomizeoneparchoices}
\choice increases.
\choice decreases.
\correctchoice remains the same.
\end{randomizeoneparchoices}

\question
Using ground level as the reference height with zero potential energy, which object has the GREATEST gravitational potential energy?

\begin{randomizechoices}
\choice a \SI{40}{kg} object at a \SI{2}{m} height
\choice a \SI{2}{kg} object at a \SI{60}{m} height
\choice a \SI{5}{kg} object at a \SI{5}{m} height
\correctchoice a \SI{20}{kg} object at a \SI{50}{m} height
\end{randomizechoices}

\question
What is the kinetic energy of a \SI{2.0}{kg} toy car moving at \SI{5}{m/s}?

\begin{randomizeoneparchoices}
\choice \SI{10}{J}
\correctchoice \SI{25}{J}
\choice \SI{50}{J}
\choice \SI{5}{J}
\end{randomizeoneparchoices}

\question
How high can a worker on a construction crane lift a \SI{40}{kg} bag of sand if the crane uses \SI{4,000}{J} of energy? Assume no energy is used to overcome friction.

\begin{randomizeoneparchoices}
\choice \SI{160000}{m}
\choice \SI{1020}{m}
\correctchoice \SI{10.2}{m}
\choice \SI{16.0}{m}
\end{randomizeoneparchoices}

\question
A piston, moving through a distance of \SI{0.15}{m}, pushes a box with a mass of \SI{8.0}{kg} onto a conveyor belt with a force of \SI{40}{N}. How much work is done by the piston on the box?

\begin{randomizeoneparchoices}
\correctchoice \SI{6.0}{J}
\choice \SI{320}{J}
\choice \SI{5.0}{J}
\choice \SI{120}{J}
\end{randomizeoneparchoices}

\question
Three hikers take three different paths to the top of a mountain, Paths 1, 2, and 3. The hikers are all have the same height and weight. When all of the hikers are at the finish point at the top of the mountain, which hiker will have the greatest amount of gravitational potential energy?

\begin{randomizechoices}
\choice The hiker who took Path 1. 
\choice The hiker who took Path 2.
\choice The hiker who took Path 3.
\choice The gravitational potential energy is the same for all the hikers.
\end{randomizechoices}

\question
A high diver steps off a diving platform that is 10 meters above the water. Assume there is no air resistance. During the fall, there will be a decrease in the diver's \fillin\ .

\begin{randomizechoices}
\choice kinetic energy
\choice total mechanical energy
\correctchoice gravitational potential energy
\choice momentum
\end{randomizechoices}

\question
The SI unit of power is the \fillin\ .

\begin{randomizeoneparchoices}
\choice meter (m)
\choice joule (J)
\correctchoice watt (W)
\choice second (s)
\choice newton (N)
\end{randomizeoneparchoices}

\question
The amount of potential energy possessed by an elevated object is equal to \fillin\ .

\begin{randomizechoices}
\choice the distance it is lifted
\choice the power used to lift it
\correctchoice the work done in lifting it
\choice the force needed to lift it
\choice the value of the acceleration due to gravity
\end{randomizechoices}

\question
A  ball is thrown into the air with \SI{100}{J} of kinetic energy, which is transformed to gravitational potential energy at the top of its trajectory. When it returns to its original level (neglecting air resistance), its kinetic energy is \fillin\ .

\begin{randomizeoneparchoices}
\choice less than \SI{100}{J}
\choice more than \SI{100}{J}
\correctchoice \SI{100}{J}
\choice not possible to calculate
\end{randomizeoneparchoices}

\question
An object that has kinetic energy must be \fillin\ .

\begin{randomizeoneparchoices}
\correctchoice moving
\choice at rest
\choice falling
\choice elevated
\end{randomizeoneparchoices}

\question
How many joules of work are done on a box when a force of \SI{25}{N} pushes it \SI{3}{m}?

\begin{randomizeoneparchoices}
\choice \SI{3}{J}
\choice \SI{8}{J}
\choice \SI{25}{J}
\correctchoice \SI{75}{J}
\choice \SI{1}{J}
\end{randomizeoneparchoices}

\question
How much power is required to do \SI{40}{J} of work on an object in 5 seconds?

\begin{randomizeoneparchoices}
\choice \SI{200}{W}
\choice \SI{5}{W}
\choice \SI{40}{W}
\choice \SI{0}{W}
\correctchoice \SI{8}{W}
\end{randomizeoneparchoices}

\question
How much power is expended if you lift a \SI{60}{N} crate 10 meters in 1 second?

\begin{randomizeoneparchoices}
\choice \SI{6}{W}
\choice \SI{0}{W}
\correctchoice \SI{600}{W}
\choice \SI{10}{W}
\choice \SI{60}{W}
\end{randomizeoneparchoices}

\question
Mechanical energy can be in the form of

\begin{randomizechoices}
\choice kinetic energy only
\choice neither kinetic nor potential energy
\correctchoice both kinetic and potential energy
\choice potential energy only
\end{randomizechoices}

\question
Gravitational potential energy is the energy an object has stored due to its \fillin\ .

\begin{randomizeoneparchoices}
\choice density
\correctchoice height above ground
\choice color
\choice speed and direction
\end{randomizeoneparchoices}




\clearpage
\ifprintanswers
    \printkeytable
\fi


\end{questions}
\end{document}