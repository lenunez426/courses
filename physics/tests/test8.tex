\documentclass{exam}
\usepackage[utf8]{inputenc}
\usepackage[english]{babel}
\usepackage{geometry}
\usepackage[T1]{fontenc}
\usepackage{graphicx}
\graphicspath{ {../Figures/} }
\setlength\parindent{0pt}
\usepackage{hyperref}
\hypersetup{colorlinks=true,linkcolor=blue,filecolor=magenta,urlcolor=cyan,}
\urlstyle{same}
\usepackage{amsthm}
\usepackage{amsmath}
\theoremstyle{definition}
\usepackage{pgfplots}
\usepackage{caption}
\usepackage{subcaption}
\usepackage{makecell}
\usepackage[table]{colortbl}
\usepackage{enumitem}
\usepackage{siunitx}
\usepackage{amssymb}
\usepackage{tikz-cd}
\tikzset{>=latex}
\usepackage{tkz-euclide}
\usepackage{tikz,bm}
\usepackage{mwe,tikz}
\usetikzlibrary{arrows}
\pgfplotsset{compat=1.11}
\usepackage{moresize}
\usepackage{bohr}
\usetikzlibrary{patterns}
\usepackage{wrapfig}
\usepackage{mdframed}
\usepackage{dashrule}
\usepackage{tikzsymbols}
\usepackage{fontawesome}
\usepackage{linearb} %for \BPwheel symbol in Unit 6
\usepackage{multicol}
\usepackage{glossaries}
\usepackage{cancel}
% \usepackage{circuitikz}
\sisetup{group-separator = {,}}
\usepgfplotslibrary{fillbetween}
\usetikzlibrary{math}
\numberwithin{equation}{section}
\numberwithin{figure}{section}




\DeclareSIUnit{\nothing}{\relax}
\def\mymu{\SI{}{\micro\nothing} }

\newtheorem{example}{Example}[section]
\newtheorem{exercise}{}[section]
\newtheorem{regla}{Rule}


\newcommand{\Solution}{{\footnotesize \color{cyan} SOLUTION }}


\def\endsolution{{\footnotesize \color{cyan} \hfill END OF SOLUTION }}

\newcommand{\cyanhrule}{{\color{cyan} \hrule }}

\def\redplus{\mathbin{\color{red} +}}
\def\redminus{\mathbin{\color{red} -}}
\def\redtimes{\mathbin{\color{red} \times}}


\def\openstax{https://openstax.org/books/physics/pages/1-introduction}
\def\openstaxfooter{\fancyfoot[C]{Access for free at \href{\openstax}{\openstax} \hfill \thepage}}


% The following needs to be added to each individual main.tex file so it doesn't interfere with exam headers:

%\usepackage{fancyhdr}
% \pagestyle{fancy}
% \renewcommand{\headrulewidth}{0pt}
% \renewcommand{\headruleskip}{0mm}
% \fancyhead{}
% \def\openstax{https://openstax.org/books/physics/pages/1-introduction}
% \def\openstaxfooter{\fancyfoot[C]{Access for free at \href{\openstax}{\openstax} \hfill \thepage}}

\newcommand\myboxa[2][]{\tikz[overlay]\node[fill=gray!20,inner sep=4pt, anchor=text, rectangle, rounded corners=1mm,#1] {#2};\phantom{#2}}
\newcommand{\hgraydashline}{{\color{lightgray} \hdashrule{0.99\textwidth}{1pt}{0.8mm}}}

\let\oldtexttt\texttt% Store \texttt
\renewcommand{\texttt}[2][black]{\textcolor{#1}{\ttfamily #2}}% 

\newcommand\mybox[2][]{\tikz[overlay]\node[fill=black!20,inner sep=2pt, anchor=text, rectangle, rounded corners=1mm,#1] {#2};\phantom{#2}}

\setlength{\columnsep}{1cm}
\setlength{\columnseprule}{1pt}
\def\columnseprulecolor{\color{cyan}}

\pgfdeclarehorizontalshading{visiblelight}{50bp}{
color(0.00000000000000bp)=(red);
color(8.33333333333333bp)=(orange);
color(16.66666666666670bp)=(yellow);
color(25.00000000000000bp)=(green);
color(33.33333333333330bp)=(cyan);
color(41.66666666666670bp)=(blue);
color(50.00000000000000bp)=(violet)
}

\def\myfillin{\rule{2cm}{0.15mm}}

\def\phet{\texttt[red]{PhET} }

\usepackage{circuitikz}

\usepackage{utfsym} %to get symbol of car
\def\mycar{\reflectbox{\huge\usym{1F697}} } %modiying symbol of car
\def\mycarleft{\huge\usym{1F697}}%modiying symbol of car

\def\mytrain{\reflectbox{\huge\usym{1F682}} } %modiying symbol of train
\def\mytrainleft{\huge\usym{1F682}}%modiying symbol of train


\usepackage{nameref}

\setenumerate{itemsep=-2pt,topsep=0pt,leftmargin=4em}




\usepackage{exam-randomizechoices}
\usepackage{circuitikz}

\def\one{1}
\def\two{2}
\def\three{3}
\def\four{4}

\def\testVersion{\four} % Edit Version number here only.

\setrandomizerseed{\testVersion}

\pagestyle{headandfoot}
\firstpageheader{Physics}{Test on Unit 8: Electrostatics}{Version \testVersion}
\CorrectChoiceEmphasis{\color{red}\bfseries}
\SolutionEmphasis{\color{red}}

%\printanswers

\begin{document}

\begin{center}
\fbox{\fbox{\parbox{5.5in}{
\begin{center}
\vspace{-1em}
  \textbf{EQUATIONS}  
\end{center}
\vspace{-2em}

\begin{equation*}
    e = \qty{1.6e-19}{C} \hspace{5mm} 
    q = -ne\ \text{(excess electrons)} \hspace{5mm}
    k = \num{8.99e9} \hspace{5mm}
    F = \frac{k \left|q_1 q_2 \right|}{r^2} 
\end{equation*}%
\begin{equation*}
    \mymu = \text{``micro''} = 10^{-6} \hspace{5mm}
    \text{n} = \text{``nano''} = 10^{-9} \hspace{5mm}
    \qty{1}{cm} = \qty{0.01}{m}
\end{equation*}

}}}
\end{center}

\begin{center}
\fbox{\fbox{\parbox{5.5in}{
\begin{center}
\vspace{-1em}
  \textbf{EQUATIONS}  
\end{center}
\vspace{-2em}

\begin{equation*}
    V = I R \hspace{1em} \text{\footnotesize (Ohm's Law)} \qquad
    R_{\text{eq}} = R_1 + R_2 \hspace{1em} \text{ \footnotesize (series)} \qquad
    R_{\text{eq}} = \left(\frac{1}{R_1} + \frac{1}{R_2}\right)^{-1} \hspace{1em} \text{\footnotesize (parallel)}
\end{equation*}


}}}
\end{center}

\begin{questions}

\question
J.~J.~Thompson discovered the electron, the negatively charged particle, in the year \fillin[1897].

\begin{randomizechoices}
    \correctchoice 1897
    \choice 1913
    \choice 2005
    \choice 1796
\end{randomizechoices}

\question
Millikan and Fletcher conducted the oil drop experiment 16 years after the discovery of the electron. What was the motivation for the experiment?

\begin{randomizechoices}
    \correctchoice To numerically calculate the charge on 1 electron.
    \choice To re-discover the electron independently of Thompson.
    \choice To measure the true size of 1 electron.
    \choice To conduct an electrical current through drops of oil.
\end{randomizechoices}

\question
What did Millikan conclude was the magnitude of the charge on 1 electron, in units of coulombs? (This value is called the \textbf{elementary charge} or the fundamental unit of electric charge.)

\begin{randomizechoices}
    \correctchoice \SI{1.60e-19}{C}
    \choice \SI{8.99e9}{C}
    \choice \SI{6.67e-11}{C}
    \choice \SI{9.805e-6}{C}
\end{randomizechoices}

\question
What is the SI unit of electric charge?

\begin{randomizechoices}
\correctchoice coulomb (C)
\choice joule (J)
\choice newton (N)
\choice meter per second (m/s)
\end{randomizechoices}

\question
A neutral plastic strip is rubbed with cotton and acquires a positive charge. Which of the following statements is true of the positively-charged strip?

\begin{randomizechoices}
\choice It gained some electrons during the rubbing process.
\correctchoice It lost some electrons to the cotton during the charging process.
\choice It gained some protons during the rubbing process.
\choice It lost some protons to the cotton during the charging process.
\end{randomizechoices}



\question
Two neutral balloons can make contact with each other. But if you rub the two balloons on a sweater and then try to place them next to one another, they will automatically pull away from each other. Why does this phenomenon occur?

\hspace{1cm}

\clearpage
\begin{randomizechoices}
\choice Both balloons lose electrons to the sweater, becoming negatively charged. Therefore, the balloons attract each other.
\choice Both balloons lose protons to the sweater, becoming negatively charged. Therefore, the balloons attract each other.
\correctchoice Both balloons gain electrons from the sweater, becoming negatively charged. Therefore, the balloons repel each other.
\choice Both balloons gain protons to the sweater, becoming negatively charged. Therefore, the balloons repel each other.
\end{randomizechoices}




\question
A neutral hydrogen atom has one proton and one electron. If you remove the electron, what will be the leftover sign of the charge?

\begin{minipage}{0.25\textwidth}
    \begin{randomizechoices}
    \choice Negative
    \CorrectChoice Positive
    \choice Zero
    \choice Neutral
    \end{randomizechoices}
\end{minipage}%
\hspace{5mm}%
\begin{minipage}{0.2\textwidth}
    \centering
    \setbohr{insert-missing,nucleus-radius=2em,electron-radius=3pt}
    \bohr{1}{}
\end{minipage}

\question
A balloon contains 80 million excess electrons. What is the net electric charge on the balloon?

\begin{randomizechoices}
\choice \SI{-8.99e-9}{\coulomb}
\choice \SI{-1.603e-19}{C}
\choice \SI{-20.8e-28}{C}
\correctchoice \SI{-1.28e-11}{C}
\end{randomizechoices}

\question
A speck of dust with excess electrons acquires a charge of \qty{-0.8}{nC}. How many excess electrons does it contain?

\begin{randomizechoices}
    \correctchoice \num{5000000000}
    \choice \num{5e12}
    \choice $0.000\,005$
    \choice 50
\end{randomizechoices}

\question
What is the value of the Coulomb's constant?

\begin{randomizechoices}
    \correctchoice \num{8.99e9}
    \choice \num{1.60e-19}
    \choice \num{6.67e-11}
    \choice \num{5.00e-6}
\end{randomizechoices}

\question
Coulomb's law\ldots

\begin{randomizechoices}
    \correctchoice describes the electrostatic force between charged objects.
    \choice describes the gravitational force between charged objects.
    \choice states that opposite charges repel.
    \choice states that like charges attract.
\end{randomizechoices}
\vspace{1em}
\cyanhrule

\clearpage
\begin{EnvUplevel}
\textbf{See the figure below, which shows two charged objects placed near each other. Then answer questions \ref{Vgjjc7}--\ref{Fxl8ta}.}
% \textit{Reminder}: \mymu means $10^{-6}$, and 1 centimeter is 0.01 meter.
\end{EnvUplevel}

\vspace{-1em}


\def\distanceA{5}
\def\distanceR{3}

\begin{center}
\begin{tikzpicture}
\begin{axis}[width=8cm,height=3.7cm,
    clip=false,
    xmin=-1.2,xmax=4,ymin=-0.6,ymax=1.2,
    axis line style={draw=none},
    ticks=none,    
]
    \fill (0,0) circle (2pt);
    \draw (0,0) circle (6mm) node[above=6mm] {\SI{3.6}{\micro C}};
    
    \begin{scope}[xshift=\distanceR*1.24 cm]
        \fill (0,0) circle (2pt);
        \draw (0,0) circle (6mm) node[above=6mm] {\SI{2.0}{\micro C}};
    \end{scope}
    
    \draw[<->,dashed] (0,0.2) -- ++(axis direction cs: \distanceR,0);
    \node[above=6pt] at (\distanceR/2,0) {\SI{4.7}{cm}};
\end{axis}
\end{tikzpicture}
\end{center}

\question \label{Vgjjc7}
Since the objects are charged, there exists \fillin[][5cm] between them.

\begin{randomizechoices}
    \correctchoice an electrostatic force
    \choice a gravitational force
    \choice a distance
    \choice a charge
\end{randomizechoices}

\question
Are the two charges attracted or repulsed?

\begin{randomizechoices}[keeplast]
    \choice attracted
    \correctchoice repulsed
    \choice both attracted and repulsed
    \choice neither attracted nor repulsed
    \choice none of the above
\end{randomizechoices}

\question
Which option explains your answer to the previous question?

\begin{randomizechoices}[keeplast]
    \choice opposite charges attract
    \correctchoice like charges repel
    \choice the positive charge causes attraction; the negative, repulsion
    \choice the positive and negative cancel each other, removing all attraction and repulsion
    \choice none of the above
\end{randomizechoices}

\question
What is the charge of the \qty{3.6}{\micro C} object in base units of coulombs?

\begin{randomizechoices}
    \correctchoice \qty{3.6e-6}{C}
    \choice \qty{3.6e-9}{C}
    \choice \qty{3.6e9}{C}
    \choice \qty{3.6e16}{C}
\end{randomizechoices}

\question \label{Fxl8ta}
Calculate the magnitude of the electrostatic force that the charged objects exert on each other.

\begin{randomizechoices}
    \correctchoice \qty{29.3}{N}
    \choice \qty{3.25e-9}{N}
    \choice \qty{1.37}{N}
    \choice \qty{1.53e-10}{N}
\end{randomizechoices}
\vspace{1em}
\cyanhrule

\clearpage
\begin{EnvUplevel}
     \textbf{Read the prompt below. Then answer questions \ref{yETfik}--\ref{b4zyju}.}

    Two charges are placed next to each other. The first has a charge of -7.5 microcoulombs, and the second, 1.2 microcoulombs. The magnitude of the electrostatic force is 0.005 newtons.  
\end{EnvUplevel}




\begin{center}
\begin{tikzpicture}
\begin{axis}[width=8cm,height=3.7cm,
    clip=false,
    xmin=-0.5,xmax=6,ymin=-0.6,ymax=1.2,
    axis line style={draw=none},
    ticks=none,    
]
    \fill (0,0) circle (3pt);
    \draw (0,0) circle (6mm) node[above=6mm] {$q_1$};
    %\draw[->,thick] (0,0) -- ++(axis direction cs: 2,0) node[below] {$F$};

    \begin{scope}[xshift=\distanceA*0.99 cm]
        \fill (0,0) circle (3pt);
        \draw (0,0) circle (6mm) node[above=6mm] {$q_2$};
        %\draw[->,thick] (0,0) -- ++(axis direction cs: -2,0) node[below] {$F$};
    \end{scope}

    \draw[<->,dashed] (0,0.2) -- ++(axis direction cs: \distanceA,0);
    \node[above=6pt] at (\distanceA/2,0) {$r$};
\end{axis}
\end{tikzpicture}
\end{center}

\question \label{yETfik}
Is the electrostatic force attractive or repulsive?

\begin{randomizechoices}[keeplast]
    \correctchoice attractive
    \choice repulsive
    \choice both
    \choice neither
    \choice none of the above
\end{randomizechoices}

\question
Which of the following justifies your answer to the previous question?

\begin{randomizechoices}[keeplast]
    \correctchoice opposite charges attract
    \choice like charges repel
    \choice opposite charges generation attraction and repulsion
    \choice like charges prevent attraction or repulsion
    \choice none of the above
\end{randomizechoices}

\question \label{b4zyju}
What is the distance between the charges?

\begin{randomizechoices}
    \correctchoice \qty{4.0}{m}
    \choice \qty{8.9}{m}
    \choice \qty{1.8e-9}{m}
    \choice \qty{3.6e-7}{m}
\end{randomizechoices}
\vspace{1em}

\cyanhrule

\question
A \qty{-8e-6}{C} charge and \qty{9e-6}{C} charge are separated by \SI{3.0}{cm}. What is the magnitude of the electrostatic force between them?

\begin{randomizechoices}
        \correctchoice \qty{719}{N}
        \choice \qty{21.5}{N}
        \choice \qty{0.64}{N}
        \choice \qty{8e-8}{N}
\end{randomizechoices}


\question
What is electric current?

\begin{randomizechoices}
    \correctchoice electric charge that is moving
    \choice a property of matter that causes objects to attract or repel
    \choice the electric potential energy per unit charge
    \choice schematic drawing of an electrical circuit 
\end{randomizechoices}

\question 
What is the SI unit of electric current?

\begin{randomizeoneparchoices}
    \correctchoice ampere (A)
    \choice coulomb (C)
    \choice ohm ($\Omega$)
    \choice volt (V)
\end{randomizeoneparchoices}

\question
What is the function of the circuit component shown below?

\begin{center}
    \begin{circuitikz}
        \draw (0,0) to[rmeter,t=A] (3,0);
    \end{circuitikz}
\end{center}

\begin{randomizechoices}
    \correctchoice to measure current through a wire
    \choice to measure voltage across a battery
    \choice to measure resistance of a resistor
    \choice to enable a circuit to be switched on or off
\end{randomizechoices}

\question
Which circuit element is depicted in the symbol below?

\begin{center}
    \begin{circuitikz}
        \draw (0,0) to[R] (3,0);
    \end{circuitikz}
\end{center}

\begin{randomizeoneparchoices}
    \correctchoice resistor
    \choice battery
    \choice lamp
    \choice switch
    \choice zig-zag transistor
\end{randomizeoneparchoices}

\question
What is the function of a resistor in a circuit?

\begin{randomizechoices}
    \correctchoice to oppose the passage of electric current
    \choice to increase the passage of electric current
    \choice to completely stop the passage of electric current
    \choice to measure the passage of electric current
\end{randomizechoices}

\question
What is current through the resistor in the circuit below? 

\begin{center}
\begin{circuitikz}
    \draw (0,0) -- (0,2) to[battery,l=$\SI{24}{V}$] (3,2)
                -- (3,0) to[R=$\SI{8.0}{\ohm}$] (0,0);
\end{circuitikz}
\end{center}



\begin{randomizechoices}
    \correctchoice \SI{3.0}{A}
    \choice \SI{0.3}{A}
    \choice \SI{32}{A}
    \choice \SI{16}{A}
\end{randomizechoices}

\question
Which set of resistors has the \textit{smallest} equivalent resistance?

% \fbox{
{\LARGE \mybox{\textbf{A}} \hspace{-1.4em}}
\begin{minipage}{0.45\textwidth}
    \centering
    \begin{circuitikz}
        \draw (0,0) -- (1,0) -- (1,1) to[R=$\SI{1.0}{\ohm}$] (4,1) -- (4,-1) to[R=$\SI{2.0}{\ohm}$] (1,-1) -- (1,0) (4,0) -- (5,0);
    \end{circuitikz}
\end{minipage}
% }%
% \fbox{
{\LARGE \mybox{\textbf{B}} \hspace{-1.4em}}
\begin{minipage}[c][3.5cm][c]{0.45\textwidth}
    \centering
    \begin{circuitikz}
    \draw (0,0) to[R=$\SI{1.0}{\ohm}$] (2.5,0) to[R=$\SI{2.0}{\ohm}$] (5,0);
\end{circuitikz}
\end{minipage}
% }

% \fbox{
{\LARGE \mybox{\textbf{C}} \hspace{-1.4em}}
\begin{minipage}[c][3.5cm][c]{0.45\textwidth}
    \centering
    \begin{circuitikz}
    \draw (0,0) to[R=$\SI{2.0}{\ohm}$] (2.5,0) to[R=$\SI{3.0}{\ohm}$] (5,0);
\end{circuitikz}
\end{minipage}
% }%
% \fbox{
{\LARGE \mybox{\textbf{D}} \hspace{-1.4em}}
\begin{minipage}{0.45\textwidth}
    \centering
    \begin{circuitikz}
    \draw (0,0) -- (1,0) -- (1,1) to[R=$\SI{2.0}{\ohm}$] (4,1) -- (4,-1) to[R=$\SI{3.0}{\ohm}$] (1,-1) -- (1,0) (4,0) -- (5,0);
\end{circuitikz}
\end{minipage}
% }

\begin{oneparchoices}
    \correctchoice \mybox{\textbf{A}}
    \choice \mybox{\textbf{B}}
    \choice \mybox{\textbf{C}}
    \choice \mybox{\textbf{D}}
\end{oneparchoices}

\vspace{1em}

\cyanhrule

\begin{EnvUplevel}
    \textbf{Consider the circuit below. Then answer questions \ref{zaf0rk} through \ref{MMCLG2}.}
\end{EnvUplevel}

\begin{center}
    \begin{circuitikz}
        \draw (0,0) -- (3,0) to[R=$\SI{5}{\ohm}$] (3,2) -- (0,2) to[battery,l=$\SI{15}{V}$] (0,0)
            (3,0) -- (5,0) to[R=$\SI{20}{\ohm}$] (5,2) -- (3,2);
    \end{circuitikz}
\end{center}


\question \label{zaf0rk}
What is the current through the \SI{5}{\ohm} resistor?

\begin{randomizechoices}
    \correctchoice \SI{3.00}{A}
    \choice \SI{0.80}{A}
    \choice \SI{50.0}{A}
    \choice \SI{0.20}{A}
\end{randomizechoices}

\question
What is the current through the 20-ohm resistor?

\begin{randomizechoices}
    \correctchoice \SI{0.75}{A}
    \choice \SI{60.0}{A}
    \choice \SI{0.06}{A}
    \choice \SI{12.0}{A}
\end{randomizechoices}

\question \label{MMCLG2}
What is the current through the wires that are directly connected to the battery?

\begin{randomizechoices}
    \correctchoice \SI{3.75}{A}
    \choice \SI{4.0}{A}
    \choice \SI{0.25}{A}
    \choice \SI{83.4}{A}
\end{randomizechoices}

\vspace{1em}

\cyanhrule


\question
An electric appliance draws 3.0 amperes of current when it is connected to a 21-volt source. What is the resistance of this appliance?

\begin{randomizechoices}
    \correctchoice \SI{7.0}{\ohm}
    \choice \SI{63}{\ohm}
    \choice \SI{24}{\ohm}
    \choice \SI{0.14}{\ohm}
\end{randomizechoices}

 \clearpage
\question
This diagram represents a circuit. What is the total resistance of the circuit?

\begin{center}
\begin{circuitikz}
    \draw (0,0) -- (3,0) to[R=\SI{41}{\ohm}] (3,3) to[R=\SI{15}{\ohm}] (0,3) to [battery,l=\SI{61}{V}] (0,0)
            (3,0) -- (5,0) to[R=\SI{31}{\ohm}] (5,3) -- (3,3);
\end{circuitikz}
\end{center}

\begin{randomizechoices}
    \correctchoice \SI{33}{\ohm}
    \choice \SI{87}{\ohm}
    \choice \SI{25}{\ohm}
    \choice \SI{28}{\ohm}
\end{randomizechoices}

\question
Three resistors---4 ohms, 6 ohms, and 8 ohms---are connected in parallel in an electric circuit. The equivalent resistance of the circuit is \fillin\ .

\begin{randomizechoices}
    \correctchoice less than 4 ohms
    \choice between 4 and 8 ohms
    \choice between 10 and 18 ohms
    \choice greater than 18 ohms
\end{randomizechoices}


\question 
In which circuit represented below are meters properly connected to measure the current through resistor $R_1$ and the potential difference across resistor $R_2$?


{\LARGE \mybox{\textbf{A}}}
\begin{minipage}[c][4cm][c]{0.45\textwidth}
    \centering 
    \begin{circuitikz}
    \draw (0,0) to[R=$R_1$] ++(2,0) to[rmeter,t=A] ++(2,0) to[R=$R_2$] ++(2,0) to[rmeter,t=V] ++(0,2) to[battery] ++(-6,0) -- ++(0,-2);
\end{circuitikz}
\end{minipage}%
\hspace{1em}
{\LARGE \mybox{\textbf{B}} \hspace{-1ex}}
\begin{minipage}{0.45\textwidth}
    \centering 
    \begin{circuitikz}
    \draw (0,0) to[R=$R_1$] ++(3,0) to[R=$R_2$] ++(3,0) -- ++(0,2) to[battery] ++(-6,0) -- ++(0,-2) 
    (0.5,0) -- ++(0,-1) to[rmeter,t=A] ++(2,0) -- ++(0,1)
    (3.5,0) -- ++(0,-1) to[rmeter,t=V] ++(2,0) -- ++(0,1);
\end{circuitikz}
\end{minipage}

{\LARGE \mybox{\textbf{C}} \hspace{-1em}}
\begin{minipage}{0.45\textwidth}
    \centering 
    \begin{circuitikz}
    \draw (0,0) to[R=$R_2$] ++(3,0) -- ++(0,2.5) to[R,a=$R_1$] ++(-3,0) -- ++(0,-2.5)
    (0.5,0) -- ++(0,-1) to[rmeter,t=V] ++(2,0) -- ++(0,1)
    (0.5,2.5) -- ++(0,-1) to[rmeter,t=A] ++(2,0) -- ++(0,1)
    (0,2.5) -- ++(0,1.5) to[battery] ++(3,0) -- ++(0,-1.5);
\end{circuitikz}
\end{minipage}%
\hspace{1em}
{\LARGE \mybox{\textbf{D}} \hspace{-1em}}
\begin{minipage}[c][6cm][c]{0.45\textwidth}
    \centering 
    \begin{circuitikz}
    \draw (0,0) to[R=$R_2$] ++(3,0) -- ++(1,0) -- ++(0,1.5) to[rmeter,t=A] ++(-2,0) to[R,a=$R_1$] ++(-2,0) -- ++(0,-1.5)
    (0.5,0) -- ++(0,-1) to[rmeter,t=V] ++(2,0) -- ++(0,1)
    (0,1.5) -- ++(0,1.5) to[battery] ++(4,0) -- ++(0,-1.5);
\end{circuitikz}
\end{minipage}

\begin{oneparchoices}
    \choice \mybox{\textbf{A}}
    \choice \mybox{\textbf{B}}
    \choice \mybox{\textbf{C}}
    \correctchoice \mybox{\textbf{D}}
\end{oneparchoices}


\question
The diagram below represents a circuit. What is the voltage of the battery when the ammeter reads 4.0 amperes?


\begin{center}
\begin{circuitikz}
    \draw (0,0) to[battery] (3,0)
            to[rmeter,t=A] ++(0,2) to[R=\SI{6.0}{\ohm}] ++(-3,0) -- ++(0,-2)
            (3,2) -- ++(0,1.5) to[R=\SI{12}{\ohm}] ++(-3,0) -- ++(0,-1.5); 
\end{circuitikz}
\end{center}

\begin{randomizechoices}
    \correctchoice \SI{16}{V}
    \choice \SI{0.22}{V}
    \choice \SI{1.0}{V}
    \choice \SI{4.5}{V}
\end{randomizechoices}



\question
A heater has a resistance of 10.0 ohms. It operates on \SI{120.0}{V}. What is the current through the resistance?

\begin{randomizechoices}
    \correctchoice \SI{12}{A}
    \choice \SI{24}{A}
    \choice \SI{120}{A}
    \choice \SI{80}{A}
\end{randomizechoices}

\clearpage
\question
What is the current through the battery?

\begin{center}
\begin{circuitikz}
    \draw (0,0) to[R=\SI{4}{\ohm}] (0,3) to[battery,l=\SI{8}{V}] ++(3,0) -- ++(0,-3) to[R=\SI{4}{\ohm}] ++(-3,0);
\end{circuitikz}
\end{center}

\begin{randomizechoices}
    \correctchoice \SI{1}{A}
    \choice \SI{4}{A}
    \choice \SI{8}{A}
    \choice \SI{2}{A}
\end{randomizechoices}




\question
The diagram below represents parts of an electric circuit containing three resistors. What is the equivalent resistance of this part of the circuit?

\begin{center}
\begin{circuitikz}
    \draw (0,0) -- ++(0,3) to[R=\SI{3.0}{\ohm}] ++(2,0) -- ++(0,-3) to[R=\SI{12}{\ohm}]  (0,0)
    (-1,1.5) to[R=\SI{4.0}{\ohm}] ++(4,0);
\end{circuitikz}
\end{center}

\begin{randomizechoices}
    \correctchoice \SI{1.5}{\ohm}
    \choice \SI{0.67}{\ohm}
    \choice \SI{6.3}{\ohm}
    \choice \SI{19}{\ohm}
\end{randomizechoices}

\clearpage
\question
In the circuit represented by the diagram below, what is the reading of the voltmeter V?

\begin{center}
\begin{circuitikz}
    \draw (0,0) -- (3,0) to[R=\SI{10}{\ohm}] ++(0,3) to[R=\SI{20}{\ohm}] ++(-3,0) to[battery,l=\SI{60}{V}] (0,0)
            (0.5,3) -- ++(0,1) to[rmeter,t=V] ++(2,0) -- ++(0,-1);
\end{circuitikz}
\end{center}

\begin{randomizechoices}
    \correctchoice \SI{40}{V}
    \choice \SI{2.0}{V}
    \choice \SI{30}{V}
    \choice \SI{20}{V}
\end{randomizechoices}

%\ifprintanswers
\keylistkeyname{Key Version \testVersion}
\printkeytable
%\fi

\end{questions}
\end{document}

\question
What happens to the electrostatic force between 2 charges if the distance between the charges is quadrupled?

\begin{choices}
\choice the force \textit{decreases} to $1/4$ its original strength
\choice the force \textit{increases} to 4 times its original strength
\choice the force \textit{increases} to 16 times its original strength
\correctchoice the force \textit{decreases} to $1/16$ its original strength
\end{choices}

\question
The magnitude of the attractive force between two charged objects is \SI{0.96}{N}. If the distance between the objects is quadrupled, what is the magnitude of the new force?

\begin{choices}
\choice \SI{16.67}{N}
\choice \SI{15.36}{N}
\choice \SI{0.24}{N}
\choice \SI{0.06}{N}
\end{choices}

\question
Charging a neutral body by bringing it near a charged object is charging by \fillin\ .

\begin{choices}
\choice friction
\choice conduction
\choice induction
\choice VISA
\end{choices}

\question
Which of the following is the best list of insulators?

\begin{choices}
\choice rubber, copper, plastic 
\choice rubber, wood, plastic
\choice rubber, copper, wood 
\choice human body, copper, plastic
\end{choices}

\question
The force exerted between two charged spheres is \SI{64}{N}. What is the magnitude of the force when the distance between the spheres is tripled?

\begin{choices}
\choice \SI{16.0}{N}
\choice \SI{192.0}{N}
\choice \SI{7.1}{N}
\choice \SI{21.3}{N}
\end{choices}

\question
Two identical spheres contain a charge of $q$. If each sphere then acquires a new charge of $3q$, and are held at a constant separation distance, how does the electric force between the spheres change?

\begin{choices}
\choice It increases by a factor of 3.
\choice It decreases to one-ninth its
original value.
\choice It increases by a factor of 9.
\choice It decreases to one-third its
original value.
\choice It increases by a factor of 6.
\end{choices}

\question
True or false? Although both are inverse square laws, the electric force is significantly stronger than gravitational force.

\begin{choices}
\choice True
\choice False
\end{choices}

\question
True or false? Since the electric force and the gravitational force are inverse square laws, both types of forces can be either attractive or repulsive.

\begin{choices}
\choice True
\choice False
\end{choices}

\question
Skip for now.

\question
Skip for now.

\question
Skip for now.

\question
Skip for now.





\question
If two-point charges that are 2000 meters apart have charge values of \SI{40}{C} and \SI{-40}{C}, respectively, what is the
value of the electrostatic force between them? Coulomb's constant is  \SI{8.99e9}{N\,\cdot\,m^2/C^2}).