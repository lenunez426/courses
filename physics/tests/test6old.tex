\documentclass{exam}
\usepackage[utf8]{inputenc}
\usepackage[english]{babel}
\usepackage{geometry}
\usepackage[T1]{fontenc}
\usepackage{graphicx}
\graphicspath{ {../Figures/} }
\setlength\parindent{0pt}
\usepackage{hyperref}
\hypersetup{colorlinks=true,linkcolor=blue,filecolor=magenta,urlcolor=cyan,}
\urlstyle{same}
\usepackage{amsthm}
\usepackage{amsmath}
\theoremstyle{definition}
\usepackage{pgfplots}
\usepackage{caption}
\usepackage{subcaption}
\usepackage{makecell}
\usepackage[table]{colortbl}
\usepackage{enumitem}
\usepackage{siunitx}
\usepackage{amssymb}
\usepackage{tikz-cd}
\tikzset{>=latex}
\usepackage{tkz-euclide}
\usepackage{tikz,bm}
\usepackage{mwe,tikz}
\usetikzlibrary{arrows}
\pgfplotsset{compat=1.11}
\usepackage{moresize}
\usepackage{bohr}
\usetikzlibrary{patterns}
\usepackage{wrapfig}
\usepackage{mdframed}
\usepackage{dashrule}
\usepackage{tikzsymbols}
\usepackage{fontawesome}
\usepackage{linearb} %for \BPwheel symbol in Unit 6
\usepackage{multicol}
\usepackage{glossaries}
\usepackage{cancel}
% \usepackage{circuitikz}
\sisetup{group-separator = {,}}
\usepgfplotslibrary{fillbetween}
\usetikzlibrary{math}
\numberwithin{equation}{section}
\numberwithin{figure}{section}




\DeclareSIUnit{\nothing}{\relax}
\def\mymu{\SI{}{\micro\nothing} }

\newtheorem{example}{Example}[section]
\newtheorem{exercise}{}[section]
\newtheorem{regla}{Rule}


\newcommand{\Solution}{{\footnotesize \color{cyan} SOLUTION }}


\def\endsolution{{\footnotesize \color{cyan} \hfill END OF SOLUTION }}

\newcommand{\cyanhrule}{{\color{cyan} \hrule }}

\def\redplus{\mathbin{\color{red} +}}
\def\redminus{\mathbin{\color{red} -}}
\def\redtimes{\mathbin{\color{red} \times}}


\def\openstax{https://openstax.org/books/physics/pages/1-introduction}
\def\openstaxfooter{\fancyfoot[C]{Access for free at \href{\openstax}{\openstax} \hfill \thepage}}


% The following needs to be added to each individual main.tex file so it doesn't interfere with exam headers:

%\usepackage{fancyhdr}
% \pagestyle{fancy}
% \renewcommand{\headrulewidth}{0pt}
% \renewcommand{\headruleskip}{0mm}
% \fancyhead{}
% \def\openstax{https://openstax.org/books/physics/pages/1-introduction}
% \def\openstaxfooter{\fancyfoot[C]{Access for free at \href{\openstax}{\openstax} \hfill \thepage}}

\newcommand\myboxa[2][]{\tikz[overlay]\node[fill=gray!20,inner sep=4pt, anchor=text, rectangle, rounded corners=1mm,#1] {#2};\phantom{#2}}
\newcommand{\hgraydashline}{{\color{lightgray} \hdashrule{0.99\textwidth}{1pt}{0.8mm}}}

\let\oldtexttt\texttt% Store \texttt
\renewcommand{\texttt}[2][black]{\textcolor{#1}{\ttfamily #2}}% 

\newcommand\mybox[2][]{\tikz[overlay]\node[fill=black!20,inner sep=2pt, anchor=text, rectangle, rounded corners=1mm,#1] {#2};\phantom{#2}}

\setlength{\columnsep}{1cm}
\setlength{\columnseprule}{1pt}
\def\columnseprulecolor{\color{cyan}}

\pgfdeclarehorizontalshading{visiblelight}{50bp}{
color(0.00000000000000bp)=(red);
color(8.33333333333333bp)=(orange);
color(16.66666666666670bp)=(yellow);
color(25.00000000000000bp)=(green);
color(33.33333333333330bp)=(cyan);
color(41.66666666666670bp)=(blue);
color(50.00000000000000bp)=(violet)
}

\def\myfillin{\rule{2cm}{0.15mm}}

\def\phet{\texttt[red]{PhET} }

\usepackage{circuitikz}

\usepackage{utfsym} %to get symbol of car
\def\mycar{\reflectbox{\huge\usym{1F697}} } %modiying symbol of car
\def\mycarleft{\huge\usym{1F697}}%modiying symbol of car

\def\mytrain{\reflectbox{\huge\usym{1F682}} } %modiying symbol of train
\def\mytrainleft{\huge\usym{1F682}}%modiying symbol of train


\usepackage{nameref}

\setenumerate{itemsep=-2pt,topsep=0pt,leftmargin=4em}




\pagestyle{headandfoot}
\firstpageheader{Physics}{Test}{Unit 3: Projectiles}
\CorrectChoiceEmphasis{\color{red}\bfseries}
\SolutionEmphasis{\color{red}}


%\printanswers

\begin{document}
\begin{questions}

\question
The $x$-component is the \fillin\ component of a vector.

\begin{choices}
    \correctchoice horizontal
    \choice vertical
    \choice direction
    \choice magnitude
\end{choices}

\question
The $y$-component is the \fillin\ component of a vector?

\begin{choices}
    \choice horizontal
    \choice direction
    \choice magnitude
    \correctchoice vertical
\end{choices}


\begin{EnvUplevel}
\textbf{Refer to the figure below and answer the following questions.}
\end{EnvUplevel}

\begin{figure}[h!]
    \centering
\def\Ax{2}
\def\Ay{6}
\begin{tikzpicture}
\pgfplotsset{compat=1.11}
    \begin{axis}[width=8cm,height=8cm,
        axis lines = middle,
        xlabel = $x$, x label style={anchor=west},
        ylabel = $y$, y label style={anchor=south},
        ymin=-8, ymax=8,
        xmin=-8, xmax=8,
        xtick={-8,-6,...,8},
        ytick={-8,-6,...,8},
        ymajorgrids=true,
        xmajorgrids=true,
        clip=false,
        ]
        \draw[ultra thick,red,->] (axis cs: 0,0) -- (\Ax,\Ay) node[red,above] {$\vec{A}$};
    \end{axis}
\end{tikzpicture}
\end{figure}

\question
What is the $x$-component of vector $\vec{A}$?

\begin{choices}
    \choice $-6$
    \choice $-2$
    \choice 6
    \correctchoice 2
\end{choices}

\question
What is the $y$-component of $\vec{A}$?

\begin{choices}
    \choice 2
    \correctchoice 6
    \choice $-2$
    \choice $-6$
\end{choices}
\vspace{1em}
\hrule



\clearpage
\begin{EnvUplevel}
\textbf{Read the passage below. Then answer the following questions.}

The fox travels 48 meters west, then makes a left and travels 55 meters south.
\end{EnvUplevel}

\question
 What is the magnitude of the fox's resultant displacement? (\textit{Tip}: Draw a sketch.)

\begin{choices}
    \choice \SI{48}{m}
    \choice \SI{55}{m}
    \choice \SI{103}{m}
    \correctchoice \SI{73}{m}
\end{choices}

\question
What is the direction of the resultant displacement as measured counter-clockwise from \SI{0}{\degree} (i.e., from the east axis)?

\begin{choices}
    \correctchoice \SI{229}{\degree}
    \choice \SI{194}{\degree}
    \choice \SI{48.9}{\degree}
    \choice \SI{304}{\degree}
\end{choices}
\vspace{1em}
\hrule

\begin{EnvUplevel}
    \textbf{Read the passage below. Then answer the remaining questions.}
    
    Sam Houston fires a cannon horizontally at \SI{63}{m/s} from the top of a tall cliff. The cliff is 754 meters tall.
\end{EnvUplevel}

\begin{figure}[h!]
    \centering
    \begin{tikzpicture} 
    \tikzmath{
        \gravity = 9.8;
        \vi = 28.6;
        \yi = 60;
        \thetai = 0.0;
        \ys = 1.3;
        \x1 = 20; \y1 = 57.60;
        \x2 = 40; \y2 = 50.42;
        \x3 = 60; \y3 = 38.43;
        \x4 = 80; \y4 = 21.66;
        \x5 = \vi*sqrt(2*\yi/\gravity); \y5 = 0;
    }
    \pgfplotsset{compat=1.11}
        \begin{axis}[width=8cm,height=8cm,ticks=none,
        axis line style = {black!50},
        axis lines = middle,
        clip=false,
        ylabel style= {anchor=south},
        xlabel style= {anchor=west},
        ylabel = $y$,
        xlabel = $x$,
        xmin=0, xmax=120,
        ymin=0, ymax=80,
        ]
        \addplot [dashed,
            domain=0:100,
            samples=100, 
            color=black!50,
        ]
        {\yi + tan(\thetai)*x - \gravity*x^2/(2*(\vi*cos(\thetai))^2)}; %equation of path
        \fill (0,\yi) circle (3pt) node[left] {$y_0$};
        \fill (\x1,\y1) circle (3pt); %Calculated using equator of path
        \fill (\x2,\y2) circle (3pt);
        \fill (\x3,\y3) circle (3pt);
        \fill (\x4,\y4) circle (3pt);
        \fill (\x5,\y5) circle (3pt);
        \end{axis}
    \end{tikzpicture}
\end{figure}

\question
What is the horizontal acceleration of the cannon?

\begin{choices}
    \correctchoice \SI{0}{m/s^2}
    \choice \SI{63}{m/s^2}
    \choice \SI{9.8}{m/s^2}
    \choice Unknown without time $t$
\end{choices}

\question
What is the cannon's vertical acceleration?

\begin{choices}
    \choice \SI{0}{m/s^2}
    \choice \SI{-63}{m/s^2}
    \correctchoice \SI{-9.8}{m/s^2}
    \choice Unknown without time $t$
\end{choices}

\clearpage
\question
What is the horizontal velocity of the cannon 4 seconds after launch?

\begin{choices}
    \correctchoice \SI{63}{m/s}
    \choice \SI{252}{m/s}
    \choice \SI{164}{m/s}
    \choice \SI{0}{m/s}
\end{choices}

\question 
What is the cannon's vertical velocity 4 seconds after launch?

\begin{choices}
    \choice \SI{-63.0}{m/s}
    \correctchoice \SI{-39.2}{m/s}
    \choice \SI{-105}{m/s}
    \choice \SI{-44.1}{m/s}
\end{choices}

\question
What is the cannon's horizontal displacement after 7 seconds of travel?

\begin{choices}
    \correctchoice \SI{441}{m}
    \choice \SI{63}{m}
    \choice \SI{867}{m}
    \choice \SI{225}{m}
\end{choices}

\question
What is the vertical displacement of the cannon 7 seconds after launch?

\begin{choices}
    \choice \SI{-34.4}{m}
    \choice \SI{-96}{m}
    \correctchoice \SI{-240}{m}
    \choice \SI{-390}{m}
\end{choices}


\question
How much time does it take the cannon to impact the ground below?

\begin{choices}
    \correctchoice \SI{12.4}{s}
    \choice \SI{13.0}{s}
    \choice \SI{14.1}{s}
    \choice \SI{15.2}{s}
\end{choices}

\question
What is the magnitude of the cannon's velocity as the cannon strikes the ground? (This is known as the ``entry speed.'')

\begin{choices}
    \choice \SI{58.6}{m/s}
    \correctchoice \SI{136.9}{m/s}
    \choice \SI{184.6}{m/s}
    \choice \SI{7658}{m/s}
\end{choices}

\end{questions}
\end{document}

\clearpage
\question
After takeoff, an airplane travels \SI{120}{m} at \SI{33}{\degree} above the horizontal. What is the horizontal component of its displacement?

\begin{choices}
    \choice \SI{65.4}{m}
    \choice \SI{98.0}{m}
    \correctchoice \SI{101}{m}
    \choice \SI{3960}{m}
\end{choices}

\question
For the airplane in the previous problem, what is the vertical displacement?

\begin{choices}
    \choice \SI{101}{m}
    \choice \SI{75.4}{m}
    \choice \SI{33.0}{m}
    \correctchoice \SI{65.4}{m}
\end{choices}


\begin{EnvUplevel}
\textbf{Refer to the figure below and answer the following questions.}
\end{EnvUplevel}

\begin{figure}[h!]
    \centering
\def\A{5}
\def\angle{142}
\begin{tikzpicture}
\pgfplotsset{compat=1.11}
\begin{axis}[width=8cm,height=8cm,
    axis lines=middle,
    axis line style={black!20},
    xmin=-5,xmax=5,ymin=-5,ymax=5,
    axis equal,
    ticks = none,
    xlabel = $x$, x label style={anchor=west},
    ylabel = $y$, y label style={anchor=south},
        ticks=none,
        clip=false,
        ]
        \node[left] at (axis cs: -5,0) {$-x$};
        \node[below] at (axis cs: 0,-5) {$-y$};
        \draw[thick] (axis cs: -1,0) arc (180:\angle:1);
        \node at (axis cs: -1.8,0.5) {\SI{38}{\degree}}; %180 - 142 = 38
        \draw[ultra thick,black,->] (axis cs: 0,0) -- ++(axis direction cs: {\A*cos(\angle)},{\A*sin(\angle)}) node[above] {$\vec{A}=16$};
    \end{axis}
\end{tikzpicture}
\end{figure}

\question
What is the $x$-component of the vector?

\begin{choices}
    \choice $-9.85$
    \choice 12.6
    \choice 9.85
    \correctchoice $-12.6$
\end{choices}

\question
What is the $y$-component of the vector?

\begin{choices}
    \choice 12.6
    \correctchoice 9.85
    \choice $-12.6$
    \choice $-9.85$
\end{choices}

\vspace{1em}
\hrule
