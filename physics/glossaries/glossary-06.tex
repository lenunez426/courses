\makenoidxglossaries

\newglossaryentry{air resistance}{
    name=air resistance,
    description={a frictional force that slows the motion of objects as they travel through the air; when solving basic physics problems, air resistance is assumed to be zero}
}

\newglossaryentry{maximum height}{
    name=maximum height,
    description={the highest altitude, or maximum displacement in the vertical position reached in the path of a projectile}
}

\newglossaryentry{projectile}{
    name=projectile,
    description={an object that travels through the air and experiences only acceleration due to gravity}
}

\newglossaryentry{projectile motion}{
    name=projectile motion,
    description={the motion of an object that is subject only to the acceleration of gravity}
}

\newglossaryentry{range}{
    name=range,
    description={the maximum horizontal distance that a projectile travels}
}

\newglossaryentry{trajectory}{
    name=trajectory,
    description={the path of a projectile through the air}
}

\newglossaryentry{vector}{
    name=vector,
    description={a quantity that has both magnitude and direction; an arrow used to represent quantities with both magnitude and direction}
}

\newglossaryentry{component}{
    name=component,
    description={a piece of a vector that points in either the vertical or the horizontal direction; every 2-d vector can be expressed as a sum of two vertical and horizontal vector components}
}

\newglossaryentry{uniform circular motion}{
    name={uniform circular motion},
    description={the motion of an object in a circular path at constant speed}
}

\newglossaryentry{centrifugal force}{
    name=centrifugal force,
    description={a fictitious force that acts in the direction opposite the centripetal acceleration}
}

\newglossaryentry{centripetal acceleration}{
    name=centripetal acceleration,
    description={the acceleration of an object moving in a circle, directed toward the center of the circle}
}

\newglossaryentry{radius of curvature}{
    name=radius of curvature,
    description={the distance between the center of a circular path and the path}
}

\newglossaryentry{centripetal force}{
    name=centripetal force,
    description={any force causing uniform circular motion}
}

\newglossaryentry{NLUG}
{
    name=Newton's Law of Universal Gravitation,
    description={a law that states that every particle in the universe attracts every other particle with a force along a line joining them. The force is directly proportional to the product of their masses and inversely proportional to the square of the distance between them}
}

\newglossaryentry{GravConst}
{
    name=gravitational constant,
    description={a proportionality factor used in the equation for Newton’s Law of Universal Gravitation. It is a universal constant: it is thought to be the same everywhere in the universe}
}

\newglossaryentry{CenterMass}
{
    name=center of mass,
    description={the point where the entire mass of an object can be thought to be concentrated}
}

\newglossaryentry{tangential velocity}{
    name=tangential velocity,
    description={the instantaneous linear velocity of an object in circular motion}
}

\newglossaryentry{tangential speed}{
    name={tangential speed},
    description={magnitude of tangential velocity}
}

\newglossaryentry{horizontally launched projectile}{
    name=horizontally launched projectile,
    description={a projectile, launched from some height, whose initial velocity is entirely in the horizontal direction}
}