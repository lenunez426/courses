\documentclass{article}
\usepackage[landscape,margin=0.5in]{geometry}
\usepackage{hyperref}
\hypersetup{
    colorlinks=true,
    linkcolor=blue,
    filecolor=magenta,      
    urlcolor=cyan,
    }

\begin{document}
\thispagestyle{empty}

\begin{center}
    \textbf{Final Project: Making a Balloon-Powered Car}
\end{center}

\textbf{Objective}: Students will be able to design and make a balloon-powered car using ordinary materials found at home. For more information, \href{https://www.sciencebuddies.org/science-fair-projects/project-ideas/Phys_p099/physics/balloon-powered-car-challenge}{click here}.
\vspace{1em}

\textbf{Car name}: \rule{2in}{0.15mm}
\vspace{1em}

Teammate \#1: \rule{2in}{0.15mm} \hspace{2em}
Teammate \#2 (optional): \rule{2in}{0.15mm}
\vspace{1em}

\begin{center}
\textbf{RUBRIC}
\vspace{1ex}

\scriptsize
    \begin{tabular}{|p{2cm}|p{5cm}|p{5cm}|p{5cm}|p{3cm}|p{1.5cm}|}
    \hline
         & \textbf{Excellent} (10 points) & {\centering \textbf{Meets Expectations} (7.5 points)} & {\centering \textbf{Almost there} (5 points)} & {\centering \textbf{Needs improvement} (2 points)} & {\centering \textbf{Score}}\\
         \hline
        Sketch of balloon car design
        & Sketch is a thorough and detailed outline of car. Sketch created with advanced artistic ability. May be done digitally. Contains, at minimum, car body, wheels, axles, wheel-axle connectors, straw(s), and balloon(s). & Sketch contains a complete picture of car. Contains, at minimum, car body, wheels, axles, wheel-axle connectors, straw(s), and balloon(s). & Sketch is an attempt at car outline. Although it contains some parts, it's missing other key parts, labels, and/or coherence. & Something was drawn on paper. Appears hastily drawn. &  \hspace{2em} \textbf{/10}\\ \hline
        Car body & 
        Car body is a rigid and light material, e.g., cardboard, empty water bottle, paper towel roll. Body has been modified for increased aerodynamics. Teammates' names are written in permanent ink on body. & 
        Car body is a rigid and light material, e.g., cardboard, empty water bottle, paper towel roll. Teammates' names are written in permanent ink on body. & 
        Car body is there. It's heavy and/or non-aerodynamic. Although it holds everything together, it's slow and/or travels short distances. Missing teammates' names on body. & 
        Some car body is there. Car falls apart when put in motion. Body doesn't hold car together. Missing teammates' names on body. & \hspace{2em} \textbf{/10}\\ \hline
        Wheels (4) & Four wheels are present. They are smooth and symmetrical, and are modified to increase the speed and acceleration of the car. & Four wheels are present. Although they contain minor imperfections, they are can travel distances with few problems. & 3--4 wheels are present. The car is able to move, but the shape of the wheels causes the car to wobble. & 3 or less wheels are present. The car is unable to travel very far. & \hspace{2em} \textbf{/10}\\ \hline
        Axles (2) & Axles are present. They consist of rigid but light material. They allow for smooth and efficient wheel rotation. No signs of wobbling present. & Axles are present. They enable the wheels to rotate with minimal wobbling. & Axles are present. Their shape, placement, and/or material prevents the wheels from spinning efficiently. & Axles are damaged, weak, bent, not rigid, and/or can barely hold the wheels in place. Wheels may be unable to rotate. & \hspace{2em} \textbf{/10} \\ \hline
        Wheel-axle connector & Excellent connectors present. They are light but reliably hold the wheels in place. They are lubricated to facilitate fastest possible wheel rotation. & Adequate connectors are present. They hold the wheels in place and do not appear to be slowing down the car. & Some connectors are present. They enable the car to move some distance. They may be addition counter-productive friction to the wheels, causing deceleration of the car. & Some connectors are present. They comes off when car is placed into motion, or they severely prevents the rotation of wheels. & \hspace{2em} \textbf{/10} \\ \hline
        Straw & Pro Tip: Use a boba straw for maximum airflow power. & A straight straw is used here. This works but makes it difficult to blow air into the balloon. & An old and used straw is used to connect the balloon. & Something that looks like a straw is connected to the balloon. &  \hspace{2em} \textbf{/10}\\ \hline
        Balloon & This is provided by the teacher. You get this one ``on the house.''& n/a & n/a & n/a &  \hspace{2em} \textbf{/10}\\ \hline
        Other materials & Other materials improve speed and/or maximum travel distance. & Other materials fit well with design but do not contribute to speed or maximum distance. & Other materials used are too heavy for the car to move. & Using electricity to power the car (e.g., a motor) is not allowed. &  \hspace{2em} \textbf{/10}\\ \hline
        Attendance & You showed up to finals day with a completed car. & You showed up to finals day with a car that needs minor last-minute adjustments. & You showed up to finals day with a severely incompleted car. & You showed up to finals day without a car. You may be wondering if you can do in 30 minutes what you had 10 days to complete.& \hspace{2em} \textbf{/10} \\ \hline
        Final tests & Your car travels in tact in a straight line on one full ``tank'' of air. & Your car travels in tact with minimal wobbling and/or deviations on one full ``tank'' of air. & Your car stays in tact on one full ``tank'' but it severely wobbles and/or swerves off course. & Your car falls apart and cannot travel on one full ``tank'' of air. & \hspace{2em} \textbf{/10}\\
        \hline
    \hline
    \end{tabular}
\end{center}
\vspace{1em}

{\Large \textbf{Final Grade}:} \rule{2cm}{0.15mm}
\end{document}